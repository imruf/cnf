\documentclass[12pt]{article}
\usepackage[legalpaper,
            lmargin=1in,rmargin=1in,
            bmargin=1in,tmargin=1in,includefoot]{geometry}
\usepackage{fontspec}
\usepackage{titlesec}
\usepackage{multirow}
\usepackage[colorlinks=true,urlcolor=Blue]{hyperref}
\usepackage{graphicx}
\usepackage{array}
\usepackage{makecell}
\usepackage{fancyhdr}
\usepackage[none]{hyphenat}
\usepackage{longtable}
\usepackage[dvipsnames]{xcolor}
\usepackage[banglamainfont=Kalpurush,
            banglattfont=SolaimanLipi,
            % feature=1,
            % changecounternumbering=0
           ]{latexbangla}

\renewcommand{\thepage}{\ifnum\value{page}<10 ০\fi\arabic{page}}
\pagestyle{fancy}
\fancyhf{}
\renewcommand{\headrulewidth}{0pt}
% header
\chead{
\underline{{পৃষ্ঠা নং - ০৩}}
}

% footer
\rfoot{ চলমান পৃষ্ঠা-\thepage}

\fancypagestyle{laststyle}
{
   \fancyfoot[R]{}
}

\begin{document}
\begin{minipage}[t]{0.59\linewidth}
\hspace{0.5em}
\end{minipage}
\begin{minipage}[t]{1\textwidth}
% fileno
\underline{নথি নং-১৮/বিবিধ/সেকশন-৭(বি)/২০২১-২০২২}
\end{minipage}
\\
% ns8
\begin{minipage}[t]{0.05\linewidth}
০৮।
\end{minipage}
\begin{minipage}[t]{1\linewidth}
% fileno
নথি নং-১৮/বিবিধ/সেকশন-৭(বি)/২০২১-২০২২
তে আলোচ্য বিষয়ে জয়েন্ট কমিশনার মহোদয়
সদয় হয়ে আলোচনা করেছেন।
নথির পত্রাংশে রক্ষিত আমদানিকারক
Jamunua Electronics \& Automobiles Ltd
এর আবেদন পত্র এবং আবেদন পত্রের সাথে সংযুক্ত
জাতীয় রাজস্ব বোর্ডের পত্র
% nbr
নং-০৮.০১.০০০০.০৮০.০৫.০০১.১২(অংশ-১)/১৩২(১),
তারিখ-২৩.০৭.২০১৯ খ্রি:
ও মাননীয় সুপ্রীম কোর্ট কর্তৃক প্রচারিত
Contempt Petition No:07/2017
সদয় দেখা যেতে পারে।
\end{minipage}
\\
\\
\\
% ns9
\begin{minipage}[t]{0.05\linewidth}
০৯।
\end{minipage}
\begin{minipage}[t]{1\linewidth}
আমদানিকারক
Jamunua Electronics \& Automobiles Ltd
বিজ্ঞ কমিশনার মহোদয় বরাবর
একখানা আবেদন পত্র দাখিল
করেছেন। আবেদনে তারা উল্লেখ করেন যে,
জাতীয় রাজস্ব বোর্ড পত্র
% nbr
নং-০৮.০১.০০০০.০৮০.০৫.০০১.১২(অংশ-১)/১৩২(১),
তারিখ-২৩.০৭.২০১৯ খ্রি:
এর মাধ্যমে
Contempt Petition No:07/2017
এর আদেশের প্রেক্ষিতে মূল্য সংযোজন কর
বাবদ আদায়কৃত
% total
১৪,০১,২৯,৯২৪.২৪
(চৌদ্দ কোটি একলক্ষ ঊনত্রিশ হাজার নয়শত চব্বিশ টাকা চব্বিশ পয়সা) টাকা
ফেরত প্রদানের আবেদন জানান।
আমদানিকারকের পত্রটি নথির যোগাযোগ
অংশে রক্ষিত আছে।
সদয় দেখা যেতে পারে।
\end{minipage}
\\
\\
\\
% ns10
\begin{minipage}[t]{0.05\linewidth}
১০।
\end{minipage}
\begin{minipage}[t]{1\linewidth}
Contempt Petition No. 07/2017
এর বিপরীতে মহামান্য সুপ্রীম কোর্টের
আপীলাত ডিভিশন কর্তৃক প্রদত্ত রায়ের
মাধ্যমে নিম্নোক্ত নির্দেশন প্রদান করেন-


\hspace{1em}``We direct the NBR
to give effect to the direction
of this Division
dated 13.02.2017
at once and refund the VAT
and Advance Income Tax
of the aforesaid amount already
collected/received from the
petitioners, as claimed, during
the period up to 30.06.2017,
which was exempted by the
High Court Division, as per
aforesaid SROs, as well as by this
Division, to the  petitioners
forthwith. With these observations
and directions this contempt
petition is disposed of''
\end{minipage}
\\
\\
\\
% ns11
\begin{minipage}[t]{0.05\linewidth}
১১।
\end{minipage}
\begin{minipage}[t]{1\linewidth}
জাতীয় রাজস্ব বোর্ড পত্র
% nbr
নং-০৮.০১.০০০০.০৮০.০৫.০০১.১২(অংশ-১)/১৩২(১),
তারিখ-২৩.০৭.২০১৯ খ্রি:
তে সময়
০১/০৭/২০১৬ খ্রি: হতে ০৩/০৬/২০১৭ খ্রি:
এর মাধ্যমে মূল্য সংযোজন কর বাবদ
% total
১৪,০১,২৯,৯২৪.২৪
ফেরতের বিষয়ে পরবর্তী
আইনানুগ কার্যক্রম গ্রহণের জন্য
নির্দেশক্রমে অনুরোধ করা হয়েছে।
আমদানিকারকের আবেদনের প্রেক্ষিতে শুল্কায়ন
% section
সেকশন-৭(বি) তে দাখিলকৃত/শুল্কায়িত
বি/ই এবং আমদানিকারকের
পত্রের সাথে সংযুক্ত বিল অব এন্ট্রি
পর্যালোচনায় দেখা যায়,
পত্রে উল্লিখিত বি/ই সমূহের
মধ্যে আলোচ্য চালানের
% beno
\textbf{বিল অব এন্ট্রি নং সি-৩৫০৯৩১ তারিখ ১৩/০৩/২০১৭ খ্রি:}
অন্তর্ভূক্ত রয়েছে।
\end{minipage}
\\
\\
\\
% ns12
\begin{minipage}[t]{0.05\linewidth}
১২।
\end{minipage}
\begin{minipage}[t]{1\linewidth}
Contempt Petition No. 07/2017
এর আদেশের প্রক্ষিতে পরবর্তী
করণীয় বিষয়ে এ দপ্তরের আইন
উপদেষ্ঠার মতামত কামনা করা হলে এ দপ্তরের
বিজ্ঞ আইন উপদেষ্ঠা মহোদয়
% ref file
\textbf{নথি নং-এস২-৮০/বিবিধ/সেকশন-৮(এ)/১৯-২০}
-এর নোট অনুচ্ছেদ-৮ এ নিম্নোক্ত মতামত প্রদান করেন-


\hspace{1em}``মাননীয় আপীল বিভাগের
Contempt Petition No. 07/2017
নং মামলার ০৮/০৬/২০১৭ খ্রি: তারিখের
আদেশ/রায়ের আলোকে আহরিত ভ্যাট বাবদ
% total
১৪,০১,২৯,৯২৪.২৪
পিটিশনার
Jamunua Electronics \& Automobiles Ltd
এন্ড অন্যান্য বরাবর রিফান্ড
দ্রুত করা আবশ্যক।''
\end{minipage}
\\
\\
\\
% ns13
\begin{minipage}[t]{0.05\linewidth}
১৩।
\end{minipage}
\begin{minipage}[t]{1\linewidth}
Contempt Petition No. 07/2017
এর প্রেক্ষিতে বিজ্ঞ আইন উপদেষ্ঠা মহোদয়ের
মতামতের প্রেক্ষিতে
% ref file
\textbf{নথি নং-এস২-৮০/বিবিধ/সেকশন-৮(এ)/১৯-২০}
এবং সমজাতীয় অন্যান্য নথির
রিফান্ড প্রদানের প্রয়োজনীয় ব্যবস্থা গ্রহণের
জন্য নথির ছায়ানথি সেকশনে
সংরক্ষণপূর্বক মূল নথি রিফান্ড
শাখায় প্রেরণের প্রস্তাব করা হলে উল্লিখিত নথির
নোট অনুচ্ছেদ-১৩ এ বিজ্ঞ কমিশনার মহোদয় অনুমোদন
করেন (নোটাংশের ফটোকপি নথির পত্রাংশে রক্ষিত)।
\end{minipage}
\\
\\
\\
% ns14
\begin{minipage}[t]{0.05\linewidth}
১৪।
\end{minipage}
\begin{minipage}[t]{1\linewidth}
আলোচ্য চালানের
% beno
\textbf{(বিল অব এন্ট্রি নং সি-৩৫০৯৩১ তারিখ ১৩/০৩/২০১৭ খ্রি:)}
মূল্য সংযোজন কর বাবদ
% vat
২২৪৮০৪৯.৩০
% ref file
\textbf{নথি নং-এস২-৮০/বিবিধ/সেকশন-৮(এ)/১৯-২০
এর নোট অনুচ্ছেদ-১৩} এ বিজ্ঞ কমিশনার মহোদয়ের
অনুমোদন মোতাবেক আলোচ্য চালানের
বিপরীতে
% vat
২২৪৮০৪৯.৩০
রিফান্ড প্রদানের প্রয়োজনীয় ব্যবস্থা গ্রহণের জন্য
নথির ছায়ানথি সেকশনে সংরক্ষন পূর্বক
মূল নথি রিফান্ড শাখায় প্রেরণ করা যেতে পারে।
\\
\\
সদয় অবগতি ও আদেশার্থে।
\end{minipage}

\thispagestyle{laststyle}
\end{document}
