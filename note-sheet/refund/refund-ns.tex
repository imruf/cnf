\documentclass[12pt]{article}
\usepackage[legalpaper,
            lmargin=1in,rmargin=1in,
            bmargin=1in,tmargin=1in,includefoot]{geometry}
\usepackage{fontspec}
\usepackage{titlesec}
\usepackage{multirow}
\usepackage[colorlinks=true,urlcolor=Blue]{hyperref}
\usepackage{graphicx}
\usepackage{array}
\usepackage{makecell}
\usepackage{fancyhdr}
\usepackage[none]{hyphenat}
\usepackage{longtable}
\usepackage[dvipsnames]{xcolor}
\usepackage[banglamainfont=Kalpurush,
            banglattfont=SolaimanLipi,
            % feature=1,
            % changecounternumbering=0
           ]{latexbangla}

\renewcommand{\thepage}{\ifnum\value{page}<10 ০\fi\arabic{page}}
\pagestyle{fancy}
\fancyhf{}
\renewcommand{\headrulewidth}{0pt}
% header
\chead{
\underline{{পৃষ্ঠা নং - \thepage}}
}

% footer
\rfoot{ চলমান পৃষ্ঠা-\thepage}

\fancypagestyle{laststyle}
{
   \fancyfoot[R]{}
}

\begin{document}
\begin{minipage}[t]{0.59\linewidth}
\hspace{0.5em}
\end{minipage}
\begin{minipage}[t]{1\textwidth}
% fileno
\underline{নথি নং-২০/বিবিধ/সেকশন-৭(বি)/২০২১-২০২২}
\end{minipage}
\\
% ns1
\begin{minipage}[t]{0.05\linewidth}
০১।
\end{minipage}
\begin{minipage}[t]{1\linewidth}
Jamunua Electronics \& Automobiles Ltd
এর আবেদন পত্র নং-যমুনা/কাস-চট্ট/মূ.অ/২০১৯/০৯,
তারিখ: ২৪/০৭/২০১৯ খ্রিঃ নথির অপর পাশে রক্ষিত আছে।
দয়া করে দেখা যেতে পারে।
প্রতিষ্ঠানটির আবেদনের প্রেক্ষিতে পরবর্তী কার্যক্রমের জন্য একজন এআরও
এর নামে বন্টন করা যেতে পারে।
\end{minipage}
\\
\\
\begin{minipage}[t]{0.05\linewidth}
\hspace{1em}
\end{minipage}
\begin{minipage}[t]{.5\linewidth}
সহকারী
\end{minipage}
\begin{minipage}[t]{1\linewidth}
আরও
\end{minipage}
\\
\\
\\
\\
\\
\\
% ns2
\begin{minipage}[t]{0.05\linewidth}
০২।
\end{minipage}
\begin{minipage}[t]{1\linewidth}
নোটানুচ্ছেদ-১ এর আলোকে নথির ডানপাশে রক্ষিত আমদানিকারক
% acn
Jamunua Electronics \& Automobiles Ltd
কর্তৃক
% lcno
L/C No. 0000088817020026
এর মাধ্যমে
% con
Korea
হতে
% pkg
80,000.00 KGS,
% item
POLYETHER POLYOL ELASTOCOOL KH 2030/116 C-A
আমদানি করে
% beno
বি/ই নং-সি-৫০১১০৪,
% bedt
তাং-১৬.০৪.২০১৭ ইং
দাখিল করে পণ্যচালানটি শুল্কায়ন করা হয়।
শুল্কায়ন পর্যায়ে আমদানিকারক নিমোক্ত ছকে
শুল্ককরাদি পরিশোধ করে পণ্যচালানটি খালাস গ্রহণ করেন-
\\
\\
% table
\begin{minipage}[t]{.08\linewidth}
\hspace{1em}
\end{minipage}
\begin{minipage}[t]{1\linewidth}
\begin{tabular}{|c|c|c|}
\hline
\textbf{
\makecell{
SL No.
}
}
&
\textbf{
\makecell{
Type
}
}
&
\textbf{
\makecell{
Tax Amount
}
}\\
\hline
\makecell{
1
}
&
\makecell{
CD
}
&
\makecell{
618285.04
}\\
\hline
\makecell{
2
}
&
\makecell{
RD
}
&
\makecell{
0.00
}\\
\hline
\makecell{
3
}
&
\makecell{
SD
}
&
\makecell{
0.00
}\\
\hline
\makecell{
4
}
&
\makecell{
VAT
}
&
\makecell{
1947597.88
}\\
\hline
\makecell{
4
}
&
\makecell{
AIT
}
&
\makecell{
618285.04
}\\
\hline
\makecell{
6
}
&
\makecell{
ATV
}
&
\makecell{
0.00
}\\
\hline
\multicolumn{2}{|c|}{Total} &
\makecell{
3184167.96
}\\
\hline
\end{tabular}
\end{minipage}
\end{minipage}
\\
\\
\\
% ns3
\begin{minipage}[t]{0.05\linewidth}
০৩।
\end{minipage}
\begin{minipage}[t]{1\linewidth}
পরবর্তীতে আমদানিকারক কর্তৃক দায়েরকৃত
Civil Petition for Leave to Appeal
No. 248 of 2017
এর
Contempt Petition NO. 07/2017
এর রায় ও আদেশ
পর্যালোচনা করে দেখা যায় যে,
মাননীয় আদালত হতে নিম্নরুপ
আদেশ প্রদান করেন-

\hspace{1em}``If that be so,
we direct the NBR
to give effect to the direction
of this Division
dated 13.02.2017
at once and refund the VAT
and Advance Income Tax
of the aforesaid amount already
collected/received from the
petitioners, as claimed, during
the period up to 30.06.2017,
which was exempted by the
High Court Division, as per
aforesaid SROs, as well as by this
Division, to the  petitioners
forthwith. With these observations
and directions this contempt
petition is disposed of''
\end{minipage}
\\
\\
\\
% ns4
\begin{minipage}[t]{0.05\linewidth}
০৪।
\end{minipage}
\begin{minipage}[t]{1\linewidth}
উক্ত আদেশের বিরুদ্ধে
In the Supreme Court of Bangladesh
Appellate Division
এর
Civil Review Petition
No. 415 of 2017
দায়ের করেন। মাননীয় সুপ্রীম কোর্টের
আপীলাত ডিভিশন শুনানী শেষে নিম্নোক্ত
আদেশ প্রদান করেন-

\hspace{1em}``The petition is our of time
by 59 days but the explanation
offered seeking condonation of
delay is not at all satisfactory.


\hspace{1em} Accordingly, the civil review petition
is dismissed as barred by limitation''
\end{minipage}
\\
\\
\\
% ns5
\begin{minipage}[t]{0.05\linewidth}
০৫।
\end{minipage}
\begin{minipage}[t]{1\linewidth}
উপরোক্ত আদেশের বিষয়ে
জাতীয় রাজস্ব বোর্ডের পত্র
% nbr
নং-০৮.০১.০০০০.০৮০.০৫.০০১.১২(অংশ-১)/১৩২(১),
তারিখ-২৩.০৭.২০১৯ খ্রি: এর মাধ্যমে জানান যে,
``মাননীয় সুপ্রীম কোর্টের আপীল বিভাগের
Contempt Petition No. 07/2017
এর রায় ও আদেশ মোতাবেক
Jamunua Electronics \& Automobiles Ltd.
০১/০৭/২০১৬ খ্রি: তারিখ হতে ৩০/০৬/২০১৭ খ্রি:
পর্যন্ত সময়ের জন্য ভ্যাট বাবদ
% vat
১৪,০১,২৯,৯২৪.২৪
(চৌদ্দ কোটি একলক্ষ ঊনত্রিশ হাজার নয়শত চব্বিশ টাকা চব্বিশ পয়সা) টাকা
এবং AIT বাবদ
% ait
৮,১১,২৯,৪৬৮.১৯
(আট কোটি এগারো লক্ষ ঊনত্রিশ হাজার চারশত আটষট্টি টাকা ঊনিশ পয়সা) টাকা
ফেরতের জন্য জাতীয় রাজস্ব বোর্ডে আবেদন করে। উক্ত আবেদনের
প্রেক্ষিতে সার্বিক বিষয় উল্লেখ করে মাননীয় অর্থমন্ত্রী বরাবরে
সার-সংক্ষেপ প্রেরণ করা হয়, যা মাননীয় অর্থমন্ত্রী কর্তৃক অনুমোদন
করা হয়েছে। মহামান্য আদালতের আদেশ মোতাবেক ভ্যাট বাবদ
% vat
১৪,০১,২৯,৯২৪.২৪
(চৌদ্দ কোটি একলক্ষ ঊনত্রিশ হাজার নয়শত চব্বিশ টাকা চব্বিশ পয়সা) টাকা
ফেরতের বিষয়ে পরবর্তী আইনানুগ কার্যক্রম গ্রহণের জন্য
নির্দেশক্রমে অনুরোধ করা হয়েছে''।
\end{minipage}
\newpage
\begin{minipage}[t]{0.59\linewidth}
\hspace{0.5em}
\end{minipage}
\begin{minipage}[t]{1\textwidth}
% fileno
\underline{নথি নং-২০/বিবিধ/সেকশন-৭(বি)/২০২১-২০২২}
\end{minipage}
% ns6
\begin{minipage}[t]{0.05\linewidth}
০৬।
\end{minipage}
\begin{minipage}[t]{1\linewidth}
জাতীয় রাজস্ব বোর্ড কর্তৃক জারীকৃত পত্র
পর্যালোচনায় দেখা যায়, আলোচ্য বি/ই টি জাতীয়
রাজস্ব বোর্ড থেকে প্রাপ্ত তালিকায় অন্তর্ভুক্ত রয়েছে
(পৃ: \hspace{3em})।
উল্লেখ্য যে, মহামান্য সুপ্রীম কোর্টের আপীল বিভাগের
রায়ের আইনগত করনীয় সম্পর্কে
নথি নং- ১০১৩/এপি/সেকশন-৮(এ)/১৬-১৭
এর নোটানুচ্ছেদ ২১ অনুযায়ী বিজ্ঞ আইন উপদেষ্টা
মহোদয়ের মতামত চাওয়া হলে তিনি দ্রুত
আমদানিকারকের ভ্যাটের টাকা ফেরত ও ব্যাংক
গ্যারান্টি অবমুক্ত করা আবশ্যক মর্মে মতামত প্রদান করেন।
\end{minipage}
\\
\\
\\
% ns7
\begin{minipage}[t]{0.05\linewidth}
০৭।
\end{minipage}
\begin{minipage}[t]{1\linewidth}
এমতাবস্থায়, আলোচ্য আমদানিকারক
(Jamunua Electronics \& Automobiles Ltd)
কর্তৃক দাখিলকৃত
% beno
(বি/ই নং- সি-৫০১১০৪, তাং- ১৬.০৪.২০১৭ ইং)
০১/০৭/২০১৬ খ্রি: তারিখ হতে ৩০/০৬/২০১৭ খ্রি:
তারিখের মধ্যে দাখিল হওয়ায় নিম্নোক্ত প্রস্তাব
অনুমোদন করা যেতে পারে।
\\
প্রস্তাব:
\\
(ক) জাতীয় রাজস্ব বোর্ডের সংযুক্তি মোতাবেক শুল্কায়ন
সেকশন-৭(বি) এর মোট বিল অব এন্ট্রি ৩০টি এবং
টাকার পরিমাণ
৪,৬০,৩১,২৬৮.০৭ টাকা।
রিফান্ড শাখা হতে প্রাপ্ত নথি যাচাই করে দেখা যায় এ
সেকশনের ৯টি বিল অব এন্ট্রি এর বিপরীতে মোট
১,১০,০১,৪৯৬.৭১ টাকা
ইতোমধ্যে রিফান্ড করা হয় এবং
অবশিষ্ট ২১টি বিল অব এন্ট্রি এর বিপরীতে
৩,৫০,২৯,৭৭১.৩৬ টাকা
রিফান্ড পরিশোধযোগ্য।
সংযুক্ত (পৃ \hspace{3em})।
\\
(খ) আলোচ্য বি/ই এর বিপরীতে আদায়কৃত ভ্যাট বাবদ
১৯,৪৭,৫৯৭.৮৮ টাকা
এ দপ্তর থেকে রিফান্ড প্রদান করা যেতে পারে।
\\
(গ) অগ্রিম আয়কর বাবদ
৬,১৮,২৮৫.০৪ টাকা ফেরত প্রদানের লক্ষ্যে আমদানিকারককে
সংশ্লিষ্ট কর অঞ্চলে যোগাযোগ করার জন্য বলা যেতে পারে। এবং
\\
(ঘ) রিফান্ড আবেদনের বিষয়ে সিদ্ধান্তের জন্য নথিটি রিফান্ড
কমিটি বরাবর পেশ করা যেতে পারে।
\\
\\
সদয় অবগতি ও আদেশার্থে।

\end{minipage}




\thispagestyle{laststyle}

\end{document}
