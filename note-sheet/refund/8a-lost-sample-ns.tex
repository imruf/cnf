\documentclass[12pt]{article}
\usepackage[legalpaper,lmargin=1in,rmargin=1in,tmargin=1in,bmargin=1in]{geometry}
\usepackage{fontspec}
\usepackage{titlesec}
\usepackage{multirow}
\usepackage[colorlinks=true,urlcolor=Blue]{hyperref}
\usepackage{graphicx}
\usepackage{array}
\usepackage{makecell}
\usepackage{fancyhdr}
\usepackage{longtable}
\usepackage[none]{hyphenat}
\usepackage[dvipsnames]{xcolor}
\usepackage[banglamainfont=Kalpurush,
            banglattfont=SolaimanLipi,
            % feature=1,
            % changecounternumbering=0
           ]{latexbangla}

\pagestyle{fancy}
\fancyhf{}
\renewcommand{\headrulewidth}{0pt}
\chead{
\underline{{=১১=}}
}

\fancypagestyle{laststyle}
{
   \fancyfoot[R]{}
}

\begin{document}
\begin{flushright}
{
\underline{নথি নং-১২৭৭/এপি/সেকশন-৮(এ)/২০২০-২০২১}
}
\end{flushright}
২৬। নোটানুচ্ছেদ-২৫ এর আলোকে নথিতে রক্ষিত দলিলাদি ও
আমদানিকারকের আবেদন পর্যালোচনায় দেখা যায়, আমদানিকারক
Jamuna Spinning Mills Ltd.
কর্তৃক বি/এল নং-TRA-16416 এর মাধ্যমে Turkey হতে
আলোচ্য পণ্যচালানের পণ্যগুলো আমদানি করেন, যা খালাসের উদ্যেশ্যে
সিএন্ডএফ এজেন্ট Shameem Spinning Mills Ltd.
এর মাধ্যমে বি/ই নং সি-১৬০৪০৩৪, তারিখ-২১.১১.২০২০
দাখিল করেন।
\\
\\
২৭। আলোচ্য পণ্যচালানের ১৯টি পণ্যের মধ্যে আইটেম নং-০২
এর পণ্য Seamless Steel Pipe-এর ঘোষিত ও প্রস্তাবিত
HS Code এর পার্থক্যজনিত শুল্ককরাদির সমপরিমাণ অর্থের
নিঃশর্ত ও অব্যাহত
(ব্যাংক গ্যারান্টি নং-৩৪৬/২০২০, তারিখ-১৯.১১.২০২০ টাকা-৬,৯০,৮৮৯,২৯)
জমা রেখে খালাস প্রদান করা হয়। একইসাথে খালাস পর্যায়ে
আইটেম নং-০২ এর নমুনা উত্তোলন করে
আমদানিকারকের নিজ খরচে বহিঃল্যাবে রাসায়নিক পরীক্ষার শর্তে
সাময়িক শুল্কায়ন করা হয়, যা বুয়েট হতে রাসায়নিক পরীক্ষার প্রতিবেদন
প্রাপ্তির পর সঠিক HS Code এ পন্য চালানটি চুড়ান্ত শুল্কায়নপূর্বক
(ব্যাংক গ্যারান্টি নং-৩৪৬/২০২০, তারিখ-১৯.১১.২০২০ টাকা-৬,৯০,৮৮৯,২৯)
নিষ্পত্তিযোগ্য।
\\
\\
২৮। ইতোমধ্যে আমদানিকারক কমিশনার মহোদয় বরাবর আবেদন করে
ব্যাংক গ্যারান্টির সমপরিমাণ টাকার পে-অর্ডার গ্রহণ করে চূড়ান্ত
শুল্কায়ন পূর্বক ব্যাংক গ্যারান্টি অবমুক্ত করার আবেদন করেছেন।
আবেদনে আরও উল্লেখ করেন যে, সাময়িক শুল্কায়নের শর্ত মোতাবেক
নমুনা বুয়েট টেস্ট করে ফলাফলের ভিত্তিতে পণ্যচালানটি চূড়ান্ত শুল্কায়ন
করার কথা। কিন্তু নমুনা হারিয়ে যাওয়ার কারণে বুয়েট টেস্ট করা সম্ভব হয়নি।
আমদানিকারক
(ব্যাংক গ্যারান্টি নং-৩৪৬/২০২০, তারিখ-১৯.১১.২০২০) সমপরিমাণ অর্থ
৬,৯০,৮৮৯,২৯/- পে-অর্ডার প্রদান করে চূড়ান্ত শুল্কায়ন করতে ইচ্ছুক।
ইতোমধ্যে আমদানিকারক পে-অর্ডার নং- ২৮৪৮৮৮৭, তারিখ-২৭.০৫.২০২১ খ্রিঃ
প্রেরণ করেছেন। যা গ্রহণ করে চূড়ান্ত শুল্কায়নপূর্বক ব্যাংক গ্যারান্টি ফেরতের অনুরোধ
করেছেন।
\\
\\
২৯। নথি পর্যালোচনায় দেখা যায়, আলোচ্য পণ্যচালানটি
শুল্কায়নের পূর্বে শতভাগ কায়িক পরীক্ষা করা হয়। কায়িক পরীক্ষা
পূর্বক নমুনা এ দপ্তরে প্রেরণ করা হয়েছিল। নোটানুচ্ছেদ ২২ ও ২৩
এর আলোকে দেখা যায় নমুনা শাখা কর্তৃক ০১ (এক) ব্যাগ নমুনা
অত্র শুল্কায়ন সেকশন-৮(এ) প্রেরণ করা হয়েছিল।
কিন্তু ২নং আইটেম Seamless Steel Pipe-এর নমুনা
জেটি কাস্টম হতে প্রেরিত নমুনার ব্যাগে ছিল না। ২নং আইটেমের নমুনা
পণ্য খালাসকালে সংগ্রহ করে কাস্টম হাউজ চট্টগ্রামে প্রেরণের জন্য
বলা হয়েছিল। পরবর্তীতে খালাসকালে নমুনা সংগ্রহ করা হলেও কাস্টম
হাউজে প্রেরণ করা হয়নি।
\\
\\
৩০। পরবর্তীতে আমদানিকারকের আবেদনের মাধ্যমে জানা যায় নমুনা
হারিয়ে গেছে। একইসাথে ব্যাংক গ্যারান্টির সমপরিমাণ টাকা পে-অর্ডার করে
মূল ব্যাংক গ্যারান্টি ফেরতের অনুরোধ করেছেন।
\\
\\
এমতাবস্থায়, নমুনা প্রেরিত না হওয়ায় আমদানিকারক শর্ত লঙ্ঘন করেছেন।
আমদানিকারক নিজে অপরাধ স্বীকার করে ব্যাংক গ্যারান্টির সমপরিমাণ পে-অর্ডার
প্রদান করেছেন। এমতাবস্থায় রাজস্ব হানির কোন সম্ভবনা না থাকায় ব্যাংক
গ্যারান্টির সমপরিমাণ টাকা পুনরায় গ্রহণ করে ব্যাংক গ্যারান্টি অবমুক্ত করে চূড়ান্ত
শুল্কায়ন করা যেতে পারে।
\\
\\
আলোচ্য পণ্যচালানের বিষয়ে পরবর্তী কার্যক্রম গ্রহণের
লক্ষ্যে সদয় সিদ্ধান্তের জন্য নথি উপস্থাপন করা হলো।






\thispagestyle{laststyle}

\end{document}

