\documentclass[12pt]{article}
\usepackage[legalpaper,
            lmargin=1in,rmargin=0.5in,
            bmargin=0.5in,tmargin=1in,includefoot]{geometry}
\usepackage{fontspec}
\usepackage{titlesec}
\usepackage{multirow}
\usepackage[colorlinks=true,urlcolor=Blue]{hyperref}
\usepackage{graphicx}
\usepackage{array}
\usepackage{makecell}
\usepackage{fancyhdr}
\usepackage[none]{hyphenat}
\usepackage{longtable}
\usepackage[dvipsnames]{xcolor}
\usepackage[banglamainfont=Kalpurush,
            banglattfont=SolaimanLipi,
            % feature=1,
            % changecounternumbering=0
           ]{latexbangla}

\pagestyle{fancy}
\fancyhf{}
\renewcommand{\headrulewidth}{0pt}
% header
\chead{
\underline{{পৃষ্ঠা - \thepage}}
\\
}
\rhead{{\filenou}}
% footer
\rfoot{চলমান পৃষ্ঠা-\thepage}

\fancypagestyle{laststyle}
{
   \fancyfoot[R]{}
}

\newcommand{\fileno}{নথি নং- ২০১৮/সেকশন-৫(এ)/এপি/২০২১-২০২২}
\newcommand{\filenou}{\underline{\fileno}}
\newcommand{\good}{COMPRESSOR}
\newcommand{\pkg}{217}
\newcommand{\co}{CHINA}
\newcommand{\coship}{CHINA}
\newcommand{\vessel}{PEARL RIVER BRIDGE}
\newcommand{\rotno}{2021/4907}
\newcommand{\blno}{SHWWSE210917216}
\newcommand{\bldt}{30.09.2021}
\newcommand{\beno}{C-1691243}
\newcommand{\bedt}{21.10.2021}
\newcommand{\lcno}{0000174121020222}
\newcommand{\lcdt}{24.08.2021}
\newcommand{\lcano}{265479}
\newcommand{\lcadt}{24.08.2021}
\newcommand{\lienbank}{MERCANTILE BANK LIMITED}
\newcommand{\invno}{BST2021092201}
\newcommand{\invdt}{22.09.2021}
\newcommand{\impn}{JAMUNA ELECTRONICS \& AUTOMOBILES LTD.}
\newcommand{\cnfn}{SHAMEEM SPINNING MILLS LTD.}
\newcommand{\cnfadd}{92,HIGH LEVEL ROAD
\newline
LALKHAN BAZAR, CHITTAGONG}
\newcommand{\cnfain}{301083417}
\newcommand{\impadd}{SINABHA, KALIAKAIR
\newline
PS:GAZIPUR-1750, BANGLADESH}
\newcommand{\impbin}{000146478-0103}
\newcommand{\crf}{NON CRF}
\newcommand{\crfdt}{}
\newcommand{\hscode}{7210.70.10}
\newcommand{\price}{US\$ 358,092.06}
\newcommand{\sro}{এস.আর.ও নং ১১৪ - আইন/২০২১/০৩/কাস্টমস}
\newcommand{\srodt}{তারিখ: ২৪ মে, ২০২১ খ্রি:}
\newcommand{\cpcfzn}{সিপিসি ৪০০০/৪০৯}
\newcommand{\cpcsso}{সিপিসি ৪০০০/৬৬১}
\newcommand{\taxtab}{
\noindent
\begin{longtable}{|l|l|l|l|l|l|l|l|}
\hline
\textbf{
\makecell{
ক্রঃ \\ নং
}
}
&
\textbf{
\makecell{
পণ্যের বর্ণনা
}
}
&
\textbf{
\makecell{
পরিমাণ
}
}
& \textbf{
\makecell{
ইনভয়েজ
\\
ঘোষিত
\\
এইচএসকোড
\\
(শুল্কহারসহ)
}
}
&
\textbf{
\makecell{
প্রকৃত
\\
এইচএসকোড
\\
(শুল্কহারসহ)
}
}
&
\textbf{
\makecell{
ইনভয়েজ
\\
প্রত্যায়িত
\\
একক মূল্য
\\
(US\$)
}
}
&
\textbf{
\makecell{
সাময়িক
\\
কালের
\\
অভিন্ন/অনুরুপ
\\
পণ্যের
\\
একক মূল্য
(US\$)
}
}
&
\textbf{
\makecell{
প্রস্তাবিত
\\
একক মূল্য
\\
(US\$)
}
} \\
\hline
\end{longtable}
}

\begin{document}
\noindent
\begin{minipage}[t]{0.05\linewidth}
% ns1
০১।
\end{minipage}
\begin{minipage}[t]{0.95\linewidth}
আলোচ্য বি/ই রেজি: নং- {\beno}, তারিখ: {\bedt}।
আমদানিকারকের প্রতিনিধি কমিশনার মহোদয় বরাবর একটি
আবেদন করেছেন।
আবেদন নথির ডান পার্শ্বে রক্ষিত আছে,
দয়া করে দেখা যেতে পারে।
পরবর্তী কার্যক্রমের জন্য নথিখানা উপস্থাপন করা হলো।
\\
\\
\\
\end{minipage}
\begin{minipage}[t]{0.05\linewidth}
\hspace*{1em}
\end{minipage}
\begin{minipage}[t]{0.45\linewidth}
\hspace*{1em}
\end{minipage}
\begin{minipage}[t]{0.50\linewidth}
\textbf{আর.ও}
\\
\end{minipage}
\begin{minipage}[t]{0.05\linewidth}
% ns2
০২।
\end{minipage}
\begin{minipage}[t]{0.95\linewidth}
নোট অনুচ্ছেদ-১ সদয় দ্রষ্টব্য। আমদানিকারকের
প্রতিনিধি কমিশনার মহোদয় বরাবরে একটি
আবেদন করেছেন।
আবেদন পত্র নং নাই, তারিখ: ২৬/১০/২১ ইং।
আবেদনপত্রে আমদানিকারকের প্রতিনিধি উল্লেখ
করেন যে,
তাদের আমদানিকারক {\impn} {\co}
হতে {\pkg}
{\good} আমদানি করেন।
জাতীয় রাজস্ব বোর্ড প্রদত্ত
VAT ও AIT
সুবিধায় পণ্যচালানটি ২১/১০/২০২১ ইং
তারিখে শুল্কায়ন করা হয়।
কিন্তু আমদানিকারক {\sro} {\srodt}
-এর আওতায় {\cpcfzn} -তে শুল্কায়ন
করতে ইচ্ছুক।
\\
\end{minipage}
\begin{minipage}[t]{0.05\linewidth}
% ns3
০৩।
\end{minipage}
\begin{minipage}[t]{0.95\linewidth}
\underline{\textbf{শুল্কায়ন সেকশনের পর্যালোচনা:}}
\end{minipage}
\begin{minipage}[t]{0.05\linewidth}
\hspace{1em}
\end{minipage}
\begin{minipage}[t]{0.05\linewidth}
% ka
(ক)
\end{minipage}
\begin{minipage}[t]{0.90\linewidth}
\underline{\textbf{আমদানি দলিল পত্র যাচাই:}}
পণ্য চালান খালাসের জন্য নিম্নবর্ণিত দলিলাদিসহ বি/ই দাখিল করা
হয়েছে যা সংশ্লিষ্ট ব্যাংকের অথরাইজ কর্মকর্তা কর্তৃক সত্যায়িত।
পরীক্ষান্তে এসব কাগজপত্র যথাযথ আছে বলে প্রতীয়মান হয়।
\\
(১) এল.সি এবং এল.সি.এ ফরম।
\\
(২) ইনভয়েস।
\\
(৩) প্যাকিং লিস্ট।
\\
(৪) বি/এল।
\\
(৫) কান্ট্রি অব অরিজিন সনদ।
\\
\end{minipage}
\begin{minipage}[t]{0.05\linewidth}
\hspace{1em}
\end{minipage}
\begin{minipage}[t]{0.05\linewidth}
% kha
(খ)
\end{minipage}
\begin{minipage}[t]{0.90\linewidth}
\underline{\textbf{আমদানি যোগ্যতা যাচাই:}}
প্রচলিত আমদানিনীতি আদেশ পর্যালোচনায় দেখা যায় যে, পণ্যগুলি অবাধে আমদানিযোগ্য।
আলোচ্য চালানের ক্ষেত্রে আমদানিনীতি আদেশের প্রযোজ্য অন্যান্য শর্ত (কান্ট্রি অব অরিজিন, রেজিঃ
সার্টিফিকেট ইত্যাদি) প্রতিপালিত হয়েছে।
\\
\end{minipage}
\begin{minipage}[t]{0.05\linewidth}
\hspace{1em}
\end{minipage}
\begin{minipage}[t]{0.05\linewidth}
% ga
(গ)
\end{minipage}
\begin{minipage}[t]{0.90\linewidth}
\underline{\textbf{কায়িক পরীক্ষার প্রতিবেদন পর্যালোচনা:}}
আলোচ্য পণ্য চালানটি ১০০\% কায়িক পরীক্ষার জন্য নির্বাচিত
নয়।
\\
\end{minipage}
\begin{minipage}[t]{0.05\linewidth}
\hspace{1em}
\end{minipage}
\begin{minipage}[t]{0.05\linewidth}
% gha
(ঘ)
\end{minipage}
\begin{minipage}[t]{0.90\linewidth}
\underline{\textbf{রাসায়নিক পরীক্ষা সংক্রান্ত মন্তব্য:}}
প্রযোজ্য নয়।
\\
\end{minipage}
\begin{minipage}[t]{0.05\linewidth}
\hspace{1em}
\end{minipage}
\begin{minipage}[t]{0.05\linewidth}
% uma
(ঙ)
\end{minipage}
\begin{minipage}[t]{0.90\linewidth}
\underline{\textbf{এইচ.এস.কোড সঠিকতা যাচাই:}}
ইনভয়েস প্রত্যায়িত এইচ.এস.কোড দি কাস্টমস্ এ্যাক্ট ১৯৬৯ এর FIRST SCHEDULE ও
EXPLANATORY NOTES প্রচলিত এসআরও/স্থায়ী আদেশ ইত্যাদির আলোকে পরীক্ষা করা হলো।
প্রত্যায়িত এইচ.এস.কোড যথাযথ আছে।
\\
\end{minipage}
\begin{minipage}[t]{0.05\linewidth}
\hspace{1em}
\end{minipage}
\begin{minipage}[t]{0.05\linewidth}
% ch
(চ)
\end{minipage}
\begin{minipage}[t]{0.90\linewidth}
\underline{\textbf{রেয়াতী হার বা বিশেষ মওকুফ সংক্রান্ত:}}
\\
(১) জাতীয় রাজস্ব বোর্ড পত্র নং-০৮.০১.০০০০.০৬৮.১৮.০০৪.১৭/১৬৯(৩),
তারিখ-২৭/০৬/২০২১ খ্রিঃ এর মাধ্যমে আলোচ্য প্রতিষ্ঠান কর্তৃক উপকরণ ও যন্ত্রাংশ
আমদানির ক্ষেত্রে আমদানি পর্যায়ে আরোপনীয় সমুদয় মূল্য সংযোজন কর (আগাম কর ব্যতিত)
ও সম্পূরক শুল্ক (প্রযোজ্য ক্ষেত্রে) বোর্ডের সিদ্ধান্তক্রমে অব্যাহতি প্রদান করা হয়েছে।
যার মেয়াদ আগামী ৩০/০৬/২০২২ ইং পর্যন্ত বলবৎ
রয়েছে।
\\
(২) জাতীয় রাজস্ব বোর্ড পত্র নং-০৮.০১.০০০০.০৩৪.০২.৩০১.১৯-৪৭
তারিখ: ১২/০৭/২০২১ খ্রি: এর মাধ্যমে
আলোচ্য প্রতিষ্ঠান কর্তৃক উপকরণ ও যন্ত্রাংশ
আমদানির ক্ষেত্রে
পরিশোধযোগ্য মূল্যের উপর ০.৮৩\%
উৎসে অগ্রিম আয়কর কর্তন হবে।
জাতীয় রাজস্ব বোর্ডের পত্র
(১) ও (২)
এর আলোকে পণ্যটির ক্ষেত্রে {\cpcsso} প্রযোজ্য।
\\
\end{minipage}
\begin{minipage}[t]{0.05\linewidth}
\hspace{1em}
\end{minipage}
\begin{minipage}[t]{0.05\linewidth}
% cha
(ছ)
\end{minipage}
\begin{minipage}[t]{0.90\linewidth}
\underline{\textbf{অভিযোগ সংক্রান্ত:}} এ চালানে জাতীয় রাজস্ব বোর্ডের/গোয়েন্দা
সংস্থার অভিযোগ পাওয়া যায় নাই।
\\
\end{minipage}
\begin{minipage}[t]{0.05\linewidth}
\hspace{1em}
\end{minipage}
\begin{minipage}[t]{0.05\linewidth}
% ja
(জ)
\end{minipage}
\begin{minipage}[t]{0.90\linewidth}
\underline{\textbf{ন্যায় নির্ণয় সংক্রান্ত:}} প্রযোজ্য নয়।
\\
\end{minipage}
\begin{minipage}[t]{0.05\linewidth}
\hspace{1em}
\end{minipage}
\begin{minipage}[t]{0.05\linewidth}
% jha
(ঝ)
\end{minipage}
\begin{minipage}[t]{0.90\linewidth}
\underline{\textbf{উল্লেখ করার মত প্রাসঙ্গিক অন্যান্য বিষয়:}} আমদানিকারক
মূসক ২.৩, ৪.৩ অন-লাইন আইআরসি কপি, ভ্যাট রিটার্ণ অনলাইন কপি দাখিল
করেছেন।
\\
\end{minipage}
\begin{minipage}[t]{0.05\linewidth}
\hspace{1em}
\end{minipage}
\begin{minipage}[t]{0.05\linewidth}
% nio
(ঞ)
\end{minipage}
\begin{minipage}[t]{0.90\linewidth}
উপরোল্লেখিত নোট অনুচ্ছেদ (চ) ১ ও ২ এর আলোকে
{\cpcsso} এর মাধ্যমে পণ্য চালানটি শুল্কায়ন করা হয়।
শুল্কহার যথাক্রমে
সিডি-১০\%,
আরডি-০\%,
ভ্যাট-০\%,
এআইটি-০.৮৩\%,
এটি-০\%।
\\
\end{minipage}
\begin{minipage}[t]{0.05\linewidth}
% ns4
০৪।
\end{minipage}
\begin{minipage}[t]{0.95\linewidth}
\underline{\textbf{শুল্কায়ন সম্পর্কিত প্রস্তাব:}}
পণ্য চালান সংক্রান্ত আমদানি দলিলপত্র
ইনভয়েস পর্যালোচনা পূর্বক পণ্যের বর্ণনা, পরিমাণ, এইচ.এস.কোড, ঘোষিত মূল্য,
সমসাময়িককালের অনুরূপ পণ্যের একক মূল্যসহ নিম্নের ছকে শুল্কায়নের প্রস্তাব উপস্থাপন
করা হলো:
\end{minipage}
\scriptsize
\begin{minipage}[t]{1\linewidth}
{\taxtab}
\smallskip
\end{minipage}
\normalsize
\begin{minipage}[t]{0.05\linewidth}
% ns5
০৫।
\end{minipage}
\begin{minipage}[t]{0.95\linewidth}
এমতাবস্থায়, জাতীয় রাজস্ব বোর্ড পত্র নং-০৮.০১.০০০০.০৬৮.১৮.০০৪.১৭/১৬৯(৩),
তারিখ-২৭/০৬/২০২১ খ্রিঃ এর মাধ্যমে আলোচ্য প্রতিষ্ঠান কর্তৃক উপকরণ ও যন্ত্রাংশ
আমদানির ক্ষেত্রে আমদানি পর্যায়ে আরোপনীয় সমুদয় মূল্য সংযোজন কর (আগাম কর ব্যতিত)
ও সম্পূরক শুল্ক (প্রযোজ্য ক্ষেত্রে) বোর্ডের সিদ্ধান্তক্রমে অব্যাহতি প্রদান করা হয়েছে।
যার মেয়াদ আগামী ৩০/০৬/২০২২ ইং পর্যন্ত বলবৎ
রয়েছে।
{\sro} -এর আওতায় {\cpcfzn}
নোট অনুচ্ছেদ-৪ এর ছকের প্রকৃত এইচ.এস.কোড ও
প্রস্তাবিত মূল্যে শুল্কায়ন অনুমোদন দেয়া যেতে পারে।
\\
\end{minipage}
\begin{minipage}[t]{0.05\linewidth}
% ns6
০৬।
\end{minipage}
\begin{minipage}[t]{0.95\linewidth}
উল্লেখিত ছকের প্রস্তাবিত মূল্য ও এইচ.এস.কোড
এ চালানটি শুল্ক মূল্যায়ন বিধিমালা ২০০০ এর
বিধি ০৫ মোতাবেক পুন:শুল্কায়ন প্রস্তাব অনুমোদনের
জন্য পেশ করা হলো।
\\
\end{minipage}
\begin{minipage}[t]{0.05\linewidth}
% ns7
০৭।
\end{minipage}
\begin{minipage}[t]{0.95\linewidth}
এই দপ্তরের অফিস আদেশ নং-৩৫,
তারিখ: ১৮/০৪/২০২১ ইং এর
নির্দেশনা মোতাবেক পণ্য চালানটি শুল্কায়নের
লক্ষ্যে অনুমোদনের নিমিত্তে এ্যাসিসটেন্ট/ডেপুটি কমিশনার
মহোদয় পর্যায়ে শুল্কায়ন কার্যক্রম নিষ্পত্তি হবে।
\\
সদয় অবগতি ও আদেশার্থে উপস্থাপন করা হলো।

\end{minipage}

\thispagestyle{laststyle}

\end{document}
