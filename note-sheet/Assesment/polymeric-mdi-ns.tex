\documentclass[12pt]{article}
\usepackage[legalpaper,
            lmargin=1in,rmargin=0.5in,
            bmargin=0.5in,tmargin=0.8in,includefoot]{geometry}
\usepackage{fontspec}
\usepackage{titlesec}
\usepackage{multirow}
\usepackage[colorlinks=true,urlcolor=Blue]{hyperref}
\usepackage{graphicx}
\usepackage{array}
\usepackage{makecell}
\usepackage{fancyhdr}
\usepackage[none]{hyphenat}
\usepackage{longtable}
\usepackage[dvipsnames]{xcolor}
\usepackage[banglamainfont=Kalpurush,
            banglattfont=SolaimanLipi,
            % feature=1,
            % changecounternumbering=0
           ]{latexbangla}

\pagestyle{fancy}
\fancyhf{}
\renewcommand{\headrulewidth}{0pt}
% header
\chead{
\underline{{পৃষ্ঠা - \thepage}}
\\
}
\rhead{{\filenou}}
% footer
% \rfoot{চলমান পৃষ্ঠা-\thepage}

\fancypagestyle{laststyle}
{
   \fancyfoot[R]{}
}

\newcommand{\fileno}{নথি নং - ২২২৪/এপি/সেকশন-৭(বি)/২০২১-২০২২}
\newcommand{\filenou}{\underline{\fileno}}
\newcommand{\product}{}
\newcommand{\good}{POLYMERIC MDI (WANNATE PM 2010)}
\newcommand{\pkg}{400 DRUM=100,000.00 KGS}
\newcommand{\co}{CHINA}
\newcommand{\coship}{CHINA}
\newcommand{\vessel}{MV. FILOTIMO}
\newcommand{\blno}{239893829}
\newcommand{\bldt}{09.11.2021}
\newcommand{\beno}{C-1927163}
\newcommand{\bedt}{30.11.2021}
\newcommand{\lcno}{0000088821020097}
\newcommand{\lcdt}{16.09.2021}
\newcommand{\lcano}{332646}
\newcommand{\lcadt}{{\lcdt}}
\newcommand{\lienbank}{ISLAMI BANK BANGLADESH LTD.}
\newcommand{\invno}{SG1355155BD}
\newcommand{\invdt}{09.11.2021}
\newcommand{\ssml}{SHAMEEM SPINNING MILLS LTD.}
\newcommand{\ssmla}{SHAFIPUR, KALIAKAIR
\\
GAZIPUR-1751, BANGLADESH.}
\newcommand{\jsml}{JAMUNA SPINNING MILLS LTD.}
\newcommand{\jsmla}{SHAFIPUR, KALIAKAIR
\\
GAZIPUR-1751, BANGLADESH.}
\newcommand{\jdl}{JAMUNA DENIMS LTD.}
\newcommand{\jdla}{SHAFIPUR, KALIAKAIR
\\
GAZIPUR-1751, BANGLADESH.}
\newcommand{\scml}{SHAMEEM COMPOSITE MILLS LTD.}
\newcommand{\jeal}{JAMUNA ELECTRONICS \& AUTOMOBILES LTD.}
\newcommand{\jdwl}{JAMUNA DENIMS \& WEAVING LTD.}
\newcommand{\htf}{HOORAIN HTF LTD.}
\newcommand{\tdj}{M/S THE DAILY JUGANTOR}
\newcommand{\tdja}{Ka-244, Kuril, Progati Sarani, Vatara PS
\\
Dhaka-1229,Bangladesh}

\newcommand{\impn}{\jeal}
\newcommand{\impadd}{SINABHA, KALIAKAIR
\newline
PS:GAZIPUR-1750, BANGLADESH}
\newcommand{\impbin}{000146478-0103}
\newcommand{\impoldbin}{000146478}
\newcommand{\cnfn}{SHAMEEM SPINNING MILLS LTD.}
\newcommand{\cnfadd}{92, HIGH LEVEL ROAD
\newline
LALKHAN BAZAR, CHITTAGONG}
\newcommand{\cnfain}{301 08 3417}
\newcommand{\crf}{NON CRF}
\newcommand{\crfdt}{}
\newcommand{\ircno}{আইআরসি নং - ২৬০৩২৬১২০৪২৬৭২০}
\newcommand{\ircrenewdt}{৩০/০৬/২০২২ ইং}
\newcommand{\musokr}{সেপ্টেম্বর-২১}
\newcommand{\hscode}{3909.31.10}
\newcommand{\price}{US\$ 250,000.00}
\newcommand{\menifest}{2021/5595}
\newcommand{\srooof}{এসআরও নং -১১৪-আইন/২০২১/০৩/কাস্টমস}
\newcommand{\srooofd}{তারিখ: ২৪/০৫/২০২১ খ্রি:}
\newcommand{\srotsz}{এসআরও নং - ৩৬০-আইন/২০১৩/২৪৬৮/কাস্টমস}
\newcommand{\srotszd}{তারিখ: ২৫/১১/২০১৩ খ্রি:}
\newcommand{\srofs}{এসআরও নং - ৫৭-আইন/২০০০/১৮২১/শুল্ক}
\newcommand{\srofsd}{তারিখ: ২৩/০২/২০০০ খ্রি:}
\newcommand{\nbrosnt}{জাতীয় রাজস্ব বোর্ডের পত্র নং - ০৮.০১.০০০০.০৬৮.১৮.০০৪.১৭/১৬৯(৩)}
\newcommand{\nbrosntd}{তারিখ: ২৭/০৬/২০২১ খ্রি:}
\newcommand{\nbrfs}{জাতীয় রাজস্ব বোর্ডের পত্র নং - ০৮.০১.০০০০.০৩৪.০২.৩০১.১৯-৪৭}
\newcommand{\nbrfsd}{তারিখ: ১২/০৭/২০২১ খ্রি:}
\newcommand{\cpc}{(সিপিসি ৪০০০/৪০৯)}
\newcommand{\taxtab}{
\begin{longtable}{|c|c|c|c|c|c|c|c|}
\hline
\textbf{
\makecell{
ক্রঃ \\ নং
}
}
&
\textbf{
\makecell{
পণ্যের বর্ণনা
}
}
&
\textbf{
\makecell{
পরিমাণ
}
}
& \textbf{
\makecell{
ইনভয়েস
ঘোষিত
\\
এইচএসকোড
\\
ও শুল্কহার
}
}
&
\textbf{
\makecell{
প্রকৃত
\\
এইচএসকোড
\\
ও শুল্কহার
}
}
&
\textbf{
\makecell{
ইনভয়েস
\\
প্রত্যায়িত
\\
একক মূল্য
\\
(US\$)
}
}
&
\textbf{
\makecell{
সমসাময়িক
কালের
\\
অভিন্ন/অনুরুপ
\\
পণ্যের
\\
একক মূল্য
(US\$)
}
}
&
\textbf{
\makecell{
প্রস্তাবিত
\\
একক মূল্য
\\
(US\$)
}
} \\
\hline
% row
\makecell{
01
}
&
\makecell{
POLYMERIC MDI
\\
(WANNATE PM 2010)
\\
C/O. CHINA
}
&
\makecell{
100,000.00
\\
KGS
}
&
\makecell{
3909.31.10
\\
CD-5\%
\\
AIT-0.83\%
\\
NBR LETTER
\\
SRO-114/2021
\\
CPC 4000/409
}
&
\makecell{
3909.31.10
\\
CD-5\%
\\
AIT-0.83\%
\\
NBR LETTER
\\
SRO-114/2021
\\
CPC 4000/409
}
&
\makecell{
US\$
\\
2.50/KG
}
&
\makecell{
US\$
\\
2.58/KG
\\
DATABASE
VALUE
\\
C-1529560
\\
DT:19.09.21
}
&
\makecell{
US\$
\\
2.58/KG
} \\
\hline
\end{longtable}
}
\begin{document}
\noindent
\begin{minipage}[t]{0.05\linewidth}
% ek
০১।
\end{minipage}
\begin{minipage}[t]{0.95\linewidth}
বি/ই রেজি: নং- {\beno}, তারিখ: {\bedt}
নথিভূক্ত করে
পরবর্তী কার্যক্রমের জন্য উপস্থাপন করা হলো।
\\
\\
\\
\end{minipage}
\begin{minipage}[t]{0.05\linewidth}
\hspace*{1em}
\end{minipage}
\begin{minipage}[t]{0.05\linewidth}
সহকারী
\end{minipage}
\begin{minipage}[t]{0.40\linewidth}
\hspace{1em}
\end{minipage}
\begin{minipage}[t]{0.50\linewidth}
\textbf{শুল্কায়ন কর্মকর্তা}
\\
\end{minipage}
\begin{minipage}[t]{0.05\linewidth}
% dui
০২।
\end{minipage}
\begin{minipage}[t]{0.95\linewidth}
\underline{\textbf {আমদানিকৃত পণ্য চালানের
মৌলিক তথ্য:}}
\\
\end{minipage}
\footnotesize
\begin{minipage}[t]{0.05\linewidth}
\hspace*{1em}
\end{minipage}
\begin{minipage}[t]{0.45\linewidth}
(ক) বি/ই রেজি: নং ও তারিখ
\end{minipage}
\begin{minipage}[t]{0.02\linewidth}
:
\end{minipage}
\begin{minipage}[t]{0.50\linewidth}
\textbf{{\beno}} \hspace{2em} DT: {\bedt}
\\
\end{minipage}
\begin{minipage}[t]{0.05\linewidth}
\hspace*{1em}
\end{minipage}
\begin{minipage}[t]{0.45\linewidth}
(খ) আমদানিকারকের নাম, ঠিকানা
ও BIN নম্বর
\end{minipage}
\begin{minipage}[t]{0.02\linewidth}
:
\end{minipage}
\begin{minipage}[t]{0.50\linewidth}
\textbf{{\impn}}
\\
{\impadd}
\\
BIN NO. {\impbin}
\\
OLD BIN NO. {\impoldbin}
\\
\end{minipage}
\begin{minipage}[t]{0.05\linewidth}
\hspace*{1em}
\end{minipage}
\begin{minipage}[t]{0.45\linewidth}
(গ) সিএন্ডএফ এজেন্টের নাম, ঠিকানা
ও AIN নম্বর
\end{minipage}
\begin{minipage}[t]{0.02\linewidth}
:
\end{minipage}
\begin{minipage}[t]{0.50\linewidth}
\textbf{{\cnfn}}
\\
{\cnfadd}
\\
AIN NO. {\cnfain}
\\
\end{minipage}
\begin{minipage}[t]{0.05\linewidth}
\hspace*{1em}
\end{minipage}
\begin{minipage}[t]{0.45\linewidth}
(ঘ) এল/সি নং ও তারিখ
\end{minipage}
\begin{minipage}[t]{0.02\linewidth}
:
\end{minipage}
\begin{minipage}[t]{0.50\linewidth}
{\lcno} \hspace{2em} DT: {\lcdt}
\\
\end{minipage}
\begin{minipage}[t]{0.05\linewidth}
\hspace*{1em}
\end{minipage}
\begin{minipage}[t]{0.45\linewidth}
(ঙ) লিয়েন ব্যাংকের নাম
\end{minipage}
\begin{minipage}[t]{0.02\linewidth}
:
\end{minipage}
\begin{minipage}[t]{0.50\linewidth}
{\lienbank}
\\
\end{minipage}
\begin{minipage}[t]{0.05\linewidth}
\hspace*{1em}
\end{minipage}
\begin{minipage}[t]{0.45\linewidth}
(চ) এলসিএ নং ও তারিখ
\end{minipage}
\begin{minipage}[t]{0.02\linewidth}
:
\end{minipage}
\begin{minipage}[t]{0.50\linewidth}
{\lcano} \hspace{2em} DT: {\lcadt}
\\
\end{minipage}
\begin{minipage}[t]{0.05\linewidth}
\hspace*{1em}
\end{minipage}
\begin{minipage}[t]{0.45\linewidth}
(ছ) বি/এল নং ও তারিখ
\end{minipage}
\begin{minipage}[t]{0.02\linewidth}
:
\end{minipage}
\begin{minipage}[t]{0.50\linewidth}
{\blno} \hspace{2em} DT: {\bldt}
\\
\end{minipage}
\begin{minipage}[t]{0.05\linewidth}
\hspace*{1em}
\end{minipage}
\begin{minipage}[t]{0.45\linewidth}
(জ) বাণিজ্যিক ইনভয়েস নং ও তারিখ
\end{minipage}
\begin{minipage}[t]{0.02\linewidth}
:
\end{minipage}
\begin{minipage}[t]{0.50\linewidth}
{\invno} \hspace{2em} DT: {\invdt}
\\
\end{minipage}
\begin{minipage}[t]{0.05\linewidth}
\hspace*{1em}
\end{minipage}
\begin{minipage}[t]{0.45\linewidth}
(ঝ) সিআরএফ নং ও ইস্যুর তারিখ
\end{minipage}
\begin{minipage}[t]{0.02\linewidth}
:
\end{minipage}
\begin{minipage}[t]{0.50\linewidth}
{\crf} \hspace{2em} {\crfdt}
\\
\end{minipage}
\begin{minipage}[t]{0.05\linewidth}
\hspace*{1em}
\end{minipage}
\begin{minipage}[t]{0.45\linewidth}
(ঞ) পণ্যের বিবরণ
\end{minipage}
\begin{minipage}[t]{0.02\linewidth}
:
\end{minipage}
\begin{minipage}[t]{0.50\linewidth}
{\good}
\\
\end{minipage}
\begin{minipage}[t]{0.05\linewidth}
\hspace*{1em}
\end{minipage}
\begin{minipage}[t]{0.45\linewidth}
(ট) পণ্যের পরিমাণ (একক সহ)
\end{minipage}
\begin{minipage}[t]{0.02\linewidth}
:
\end{minipage}
\begin{minipage}[t]{0.50\linewidth}
{\pkg}
\\
\end{minipage}
\begin{minipage}[t]{0.05\linewidth}
\hspace*{1em}
\end{minipage}
\begin{minipage}[t]{0.45\linewidth}
(ঠ) পণ্যের এইচ.এস.কোড
\end{minipage}
\begin{minipage}[t]{0.02\linewidth}
:
\end{minipage}
\begin{minipage}[t]{0.50\linewidth}
{\hscode}
\\
\end{minipage}
\begin{minipage}[t]{0.05\linewidth}
\hspace*{1em}
\end{minipage}
\begin{minipage}[t]{0.45\linewidth}
(ড) পণ্যের মূল্য (ইনভয়েস অনুযায়ী)
\end{minipage}
\begin{minipage}[t]{0.02\linewidth}
:
\end{minipage}
\begin{minipage}[t]{0.50\linewidth}
{\price}
\\
\end{minipage}
\begin{minipage}[t]{0.05\linewidth}
\hspace*{1em}
\end{minipage}
\begin{minipage}[t]{0.45\linewidth}
(ঢ) কান্ট্রি অব অরিজিন
\end{minipage}
\begin{minipage}[t]{0.02\linewidth}
:
\end{minipage}
\begin{minipage}[t]{0.50\linewidth}
{\co}
\\
\end{minipage}
\begin{minipage}[t]{0.05\linewidth}
\hspace*{1em}
\end{minipage}
\begin{minipage}[t]{0.45\linewidth}
(ণ) কান্ট্রি অব শিপমেন্ট
\end{minipage}
\begin{minipage}[t]{0.02\linewidth}
:
\end{minipage}
\begin{minipage}[t]{0.50\linewidth}
{\coship}
\\
\end{minipage}
\begin{minipage}[t]{0.05\linewidth}
\hspace*{1em}
\end{minipage}
\begin{minipage}[t]{0.45\linewidth}
(ত) জাহাজের নাম
\end{minipage}
\begin{minipage}[t]{0.02\linewidth}
:
\end{minipage}
\begin{minipage}[t]{0.50\linewidth}
{\vessel}
\end{minipage}
\begin{minipage}[t]{0.05\linewidth}
\hspace*{1em}
\end{minipage}
\begin{minipage}[t]{0.45\linewidth}
\hspace*{1.8em}পালা নং বি/এল নং
\end{minipage}
\begin{minipage}[t]{0.02\linewidth}
\hspace{1em}
\end{minipage}
\begin{minipage}[t]{0.50\linewidth}
{\menifest}, B/L {\blno}
\\
\end{minipage}
\begin{minipage}[t]{0.05\linewidth}
\hspace*{1em}
\end{minipage}
\begin{minipage}[t]{0.45\linewidth}
(থ) মেনিফিস্ট নং
\end{minipage}
\begin{minipage}[t]{0.02\linewidth}
:
\end{minipage}
\begin{minipage}[t]{0.50\linewidth}
{\menifest}
\\
\end{minipage}
\normalsize
\begin{minipage}[t]{0.05\linewidth}
% tin
০৩।
\end{minipage}
\begin{minipage}[t]{0.95\linewidth}
\underline{\textbf{শুল্কায়ন সেকশনের পর্যালোচনা:}}
\end{minipage}
\begin{minipage}[t]{0.05\linewidth}
\hspace{1em}
\end{minipage}
\begin{minipage}[t]{0.05\linewidth}
% tina
(ক)
\end{minipage}
\begin{minipage}[t]{0.90\linewidth}
\underline{\textbf{আমদানি দলিল পত্র যাচাই:}}
পণ্যচালান খালাসের জন্য নিম্নবর্ণিত দলিলাদিসহ বি/ই দাখিল করা
হয়েছে যা সংশ্লিষ্ট ব্যাংকের অথরাইজড কর্মকর্তা কর্তৃক সত্যায়িত।
\\
(১) এল.সি এবং এল.সি.এ ফরম।
\\
(২) ইনভয়েস।
\\
(৩) প্যাকিং লিস্ট।
\\
(৪) বি/এল।
\\
(৫) কান্ট্রি অব অরিজিন সনদ।
\\
দলিলাদি পর্যালোচনায় এগুলো
সঠিক পাওয়া যায়।
\\
\end{minipage}
\begin{minipage}[t]{0.05\linewidth}
\hspace{1em}
\end{minipage}
\begin{minipage}[t]{0.05\linewidth}
% tinb
(খ)
\end{minipage}
\begin{minipage}[t]{0.90\linewidth}
\underline{\textbf{আমদানি যোগ্যতা যাচাই:}}
প্রচলিত আমদানিনীতি আদেশ পর্যালোচনা করে দেখা যায় যে, পণ্যগুলি অবাধে আমদানিযোগ্য।
আলোচ্য চালানের ক্ষেত্রে আমদানিনীতি আদেশের প্রযোজ্য অন্যান্য শর্ত (কান্ট্রি অব অরিজিন, রেজিঃ
সার্টিফিকেট ইত্যাদি) প্রতিপালিত হয়েছে।
\\
\end{minipage}
\begin{minipage}[t]{0.05\linewidth}
\hspace{1em}
\end{minipage}
\begin{minipage}[t]{0.05\linewidth}
% tinc
(গ)
\end{minipage}
\begin{minipage}[t]{0.90\linewidth}
\underline{\textbf{কায়িক পরীক্ষার প্রতিবেদন পর্যালোচনা:}}
আলোচ্য পণ্যচালানটির ক্ষেত্রে প্রযোজ্য নয়।
\\
\end{minipage}
\begin{minipage}[t]{0.05\linewidth}
\hspace{1em}
\end{minipage}
\begin{minipage}[t]{0.05\linewidth}
% tind
(ঘ)
\end{minipage}
\begin{minipage}[t]{0.90\linewidth}
\underline{\textbf{রাসায়নিক পরীক্ষা সংক্রান্ত মন্তব্য:}}
আলোচ্য পণ্যচালানটির ক্ষেত্রে প্রযোজ্য নয়।
\\
\end{minipage}
\begin{minipage}[t]{0.05\linewidth}
\hspace{1em}
\end{minipage}
\begin{minipage}[t]{0.05\linewidth}
% tine
(ঙ)
\end{minipage}
\begin{minipage}[t]{0.90\linewidth}
\underline{\textbf{এইচ.এস.কোড সঠিকতা যাচাই:}}
আমদানিকারক কর্তৃক ঘোষিত এইচ.এস.কোড The Customs Act, 1969 এর FIRST SCHEDULE ও
EXPLANATORY NOTES প্রচলিত এসআরও/স্থায়ী আদেশ ইত্যাদির আলোকে পরীক্ষা করা হলো।
প্রত্যায়িত এইচ.এস.কোড যথাযথ আছে।
\\
\end{minipage}
\begin{minipage}[t]{0.05\linewidth}
\hspace{1em}
\end{minipage}
\begin{minipage}[t]{0.05\linewidth}
% tinf
(চ)
\end{minipage}
\begin{minipage}[t]{0.90\linewidth}
\underline{\textbf{রেয়াতী হার বা বিশেষ মওকুফ সংক্রান্ত মন্তব্য:}}
\end{minipage}
\begin{minipage}[t]{0.1\linewidth}
\hspace{1em}
\end{minipage}
\begin{minipage}[t]{0.05\linewidth}
(১)
\end{minipage}
\begin{minipage}[t]{0.85\linewidth}
{\nbrosnt}, {\nbrosntd} এর মাধ্যমে আলোচ্য
প্রতিষ্ঠান কর্তৃক উপকরণ ও যন্ত্রাংশ আমদানির ক্ষেত্রে
আমদানি পর্যায়ে আরোপনীয় সমুদয় মূল্য সংযোজন কর
(আগাম কর ব্যতিত) ও সম্পূরক শুল্ক (প্রযোজ্য ক্ষেত্রে) বোর্ডের
সিদ্ধান্তক্রমে অব্যাহতি প্রদান করা হয়েছে।
\end{minipage}
\begin{minipage}[t]{0.1\linewidth}
\hspace{1em}
\end{minipage}
\begin{minipage}[t]{0.05\linewidth}
(২)
\end{minipage}
\begin{minipage}[t]{0.85\linewidth}
{\nbrfs}, {\nbrfsd} এর মাধ্যমে আলোচ্য
প্রতিষ্ঠান কর্তৃক উপকরণ ও যন্ত্রাংশ আমদানির ক্ষেত্রে
পরিশোধযোগ্য মূল্যের উপর ০.৮৩\% হারে উৎসে অগ্রিম
আয়কর কর্তন হবে।
\end{minipage}
\begin{minipage}[t]{0.1\linewidth}
\hspace{1em}
\end{minipage}
\begin{minipage}[t]{0.05\linewidth}
(৩)
\end{minipage}
\begin{minipage}[t]{0.85\linewidth}
{\srooof}, {\srooofd}
\\
\end{minipage}
\begin{minipage}[t]{0.05\linewidth}
\hspace{1em}
\end{minipage}
\begin{minipage}[t]{0.05\linewidth}
% ting
(ছ)
\end{minipage}
\begin{minipage}[t]{0.90\linewidth}
\underline{\textbf{অভিযোগ সংক্রান্ত:}} আলোচ্য পণ্যচালানে
গোপন সংবাদ দাতা, শুল্ক গোযেন্দা বা
জাতীয় রাজস্ব বোর্ডের কিংবা অন্য দপ্তর থেকে
কোন অভিযোগ পাওয়া যায় নাই।
\\
\end{minipage}
\begin{minipage}[t]{0.05\linewidth}
\hspace{1em}
\end{minipage}
\begin{minipage}[t]{0.05\linewidth}
% tinh
(জ)
\end{minipage}
\begin{minipage}[t]{0.90\linewidth}
\underline{\textbf{ন্যায় নির্ণয় সংক্রান্ত:}} প্রযোজ্য নয়।
\\
\end{minipage}
\begin{minipage}[t]{0.05\linewidth}
\hspace{1em}
\end{minipage}
\begin{minipage}[t]{0.05\linewidth}
% tinh
(ঝ)
\end{minipage}
\begin{minipage}[t]{0.90\linewidth}
\underline{\textbf{এসআরও শর্ত পূরণের লক্ষ্যে দাখিলকৃত দলিলাদি:}}
\end{minipage}
\begin{minipage}[t]{0.10\linewidth}
\hspace{0em}
\end{minipage}
\begin{minipage}[t]{0.05\linewidth}
% chard
(১)
\end{minipage}
\begin{minipage}[t]{0.85\linewidth}
আলোচ্য আমদানিকারক হালনাগাদ Industrial IRC
দাখিল করেছেন যার {\ircno} এবং
{\ircrenewdt} তারিখ পর্যন্ত নবায়নকৃত আছে।
\end{minipage}
\begin{minipage}[t]{0.10\linewidth}
\hspace{0em}
\end{minipage}
\begin{minipage}[t]{0.05\linewidth}
% chard
(২)
\end{minipage}
\begin{minipage}[t]{0.85\linewidth}
আমদানিকারক প্রতিষ্ঠানটি শিল্প প্রতিষ্ঠান হিসেবে
মূসক-২.৩, ৪.৩ দাখিল করেছেন।
\end{minipage}
\begin{minipage}[t]{0.10\linewidth}
\hspace{0em}
\end{minipage}
\begin{minipage}[t]{0.05\linewidth}
% chard
(৩)
\end{minipage}
\begin{minipage}[t]{0.85\linewidth}
অক্টোবর ২০২১ পর্যন্ত ভ্যাট রিটার্ন দাখিল করেছেন।
\\
\end{minipage}
\begin{minipage}[t]{0.05\linewidth}
০৪।
\end{minipage}
\begin{minipage}[t]{0.95\linewidth}
\underline{\textbf{শুল্কায়ন সম্পর্কিত প্রস্তাব:}}
পণ্য চালান সংক্রান্ত আমদানি দলিলপত্র
ইনভয়েস পর্যালোচনা পূর্বক পণ্যের বর্ণনা, পরিমাণ, এইচ.এস.কোড, ঘোষিত মূল্য,
সমসাময়িককালের অভিন্ন/অনুরূপ পণ্যের একক মূল্যসহ নিম্নের ছকে শুল্কায়নের প্রস্তাব উপস্থাপন
করা হলো:
\end{minipage}
\scriptsize
{\taxtab}
\vspace*{1mm}
\normalsize
\noindent
\begin{minipage}[t]{0.05\linewidth}
% sat
০৫।
\end{minipage}
\begin{minipage}[t]{0.95\linewidth}
\underline{\textbf{শুল্কায়নযোগ্য মূল্য নিরুপন:}} ASYCUDA WORLD SYSTEM
-এ আলোচ্য পণ্য চালানের বি/ই দাখিলের পূর্ববর্তী ৯০ (নব্বই) দিনের
মূল্য তথ্য পর্যালোচনা করে দেখা যায়, হুবহু বাণিজ্যিক বর্ণনার পণ্যের যে
শুল্কায়িত মূল্য পাওয়া যায় তার সাথে সামঞ্জস্য রেখে উপরের ছকে শুল্ক
মূল্যায়ন (আমদানি পণ্যের মূল্য নির্ধারণ) বিধিমালা, ২০০০
({\srofs}, {\srofsd})
এর বিধি-৪ এর আওতায় ঘোষিত মূল্যে শুল্কায়ন প্রস্তাব রাখা হলো।
\\
\end{minipage}
\begin{minipage}[t]{0.05\linewidth}
% at
০৬।
\end{minipage}
\begin{minipage}[t]{0.95\linewidth}
{\srooof}, {\srooofd} - এর শর্তাবলী পালন করায় নোট অনুচ্ছেদ-৪ এর ছকে প্রস্তাবিত মূল্যে CPC-4000/409 তে শুল্কায়নের অনুমোদন দেয়া যেতে
পারে।
\\
\\
সদয় অবগতি ও আদেশার্থে উপস্থাপন করা
হলো।
\end{minipage}



\end{document}
