\documentclass[11pt]{article}
\usepackage[legalpaper,
            lmargin=1in,rmargin=1in,
            bmargin=1in,tmargin=1in]{geometry}
\usepackage{fontspec}
\usepackage{titlesec}
\usepackage{multirow}
\usepackage[colorlinks=true,urlcolor=Blue]{hyperref}
\usepackage{graphicx}
\usepackage{array}
\usepackage{makecell}
\usepackage{fancyhdr}
\usepackage[none]{hyphenat}
\usepackage{longtable}
\usepackage[dvipsnames]{xcolor}
\usepackage[banglamainfont=Kalpurush,
            banglattfont=SolaimanLipi,
            % feature=1,
            % changecounternumbering=0
           ]{latexbangla}

\pagestyle{fancy}
\fancyhf{}
\renewcommand{\headrulewidth}{0pt}
% header
\chead{
\underline{{পৃষ্ঠা - \thepage}}
}
% footer
\rfoot{চলমান পৃষ্ঠা-\thepage}

\fancypagestyle{laststyle}
{
   \fancyfoot[R]{}
}

\newcommand{\fileno}{নথি নং- এস-৫-১৬৩/স্টাফ/কাস/০৬-০৭/অংশ-২}
\newcommand{\filenou}{\underline{নথি নং- \hspace{5em}/স্টাফ/কাস/০৬-০৭/অংশ-২}}
\newcommand{\fdt}{\hspace*{3em} তারিখ: \hspace{2.4em} /১০/২১}
\newcommand{\frbdt}{তারিখ: \hspace{2.4em} /১০/২১}
\newcommand{\pack}{STC:220 BALES}
\newcommand{\good}{RAW COTTON}
\newcommand{\co}{IVORY COAST}
\newcommand{\depo}{EBIL, DEPOT, CHITTAGONG}
\newcommand{\rdepo}{....................................................................., DEPOT, CHITTAGONG}
\newcommand{\vessel}{MV TZINI}
\newcommand{\rotno}{2021/4696}
\newcommand{\blno}{OOLU2119927931}
\newcommand{\beno}{C-1635889}
\newcommand{\bedt}{11.10.2021}
\newcommand{\rno}{\hspace{5em} / \hspace{14.3em}}
\newcommand{\rdt}{\hspace{3em}-10-21}
\newcommand{\conno}{$\boldsymbol{02}$}
\newcommand{\conn}{(HCL)] FCL Container}
\newcommand{\cond}{$\boldsymbol{\times 40^\prime}$}
% \newcommand{\cond}{$\boldsymbol{\times 40^\prime\hspace{1em}\&\hspace{1em} 05\times 40^\prime}$}
\newcommand{\impn}{SHAMEEM SPINNING MILLS LTD.}
\newcommand{\cnfn}{SHAMEEM SPINNING MILLS LTD.}
\newcommand{\cnfadd}{92,HIGH LEVEL ROAD, LALKHAN BAZAR, CHITTAGONG}
\newcommand{\impadd}{Shafipur, Kaliakoir, Gazipur, 1751, BD}
% smml
% \newcommand{\impreg}{২৩১৩৯৫০৬১৮০-এইচ, তারিখ: ০৯/০১/২০১৩}
%jml
\newcommand{\impreg}{এল-২৯৩০১৩০১৩৪৩১-এইচ, তারিখ: ০৯/০১/২০১৩}
\newcommand{\san}{CONTINENTAL TRADERS (BD) LIMITED}
\newcommand{\sad}{IIUC Tower (9th floor), Plot \#09, Agrabad C/A, Sk. Mujib Road, Chittagong-4100.}
\newcommand{\rodt}{\hspace*{3.0em} তারিখ: \hspace{3.5em}/১০/২০২১ ইং}
\newcommand{\robdt}{\hspace*{3.0em} \hspace{3.5em}/১০/২০২১ ইং}
\newcommand{\consh}{
\noindent{
\begin{tabular}{|c|c|}
\hline
(01) \textbf{OOCU} {\cond}
&
(02) \textbf{OOCU} {\cond}
\\
\hline
\end{tabular}
}
}

\begin{document}
\begin{minipage}[t]{0.56\linewidth}
\hspace{1em}
\end{minipage}
\begin{minipage}[t]{0.44\linewidth}
{\filenou}
\end{minipage}
\begin{minipage}[t]{0.05\linewidth}
০১।
\end{minipage}
\begin{minipage}[t]{1\linewidth}
বি/ই রেজি: নং- {\beno}, {\bedt}
নথিভূক্ত করে পরবর্তী কার্যক্রমের জন্য উপস্থাপন করা হলো।
\\
\\
\\
\\
\\
\end{minipage}
\begin{minipage}[t]{0.40\linewidth}
\hspace*{1em}
\end{minipage}
\begin{minipage}[t]{.60\linewidth}
\textbf{এ.আর.ও}
\\
\end{minipage}
\begin{minipage}[t]{0.05\linewidth}
০২।
\end{minipage}
\begin{minipage}[t]{0.35\linewidth}
\underline{\textbf{আমদানিকৃত পণ্য চালানের
মৌলিক তথ্যঃ}}
\end{minipage}
\begin{tabular}{lll}
& \textbf{এ.আর.ও} \\
% ns2
০২ । \underline{\textbf {আমদানিকৃত পণ্য চালানের
মৌলিক তথ্যঃ}} & \\
\hspace{2em} (ক) বি/ই রেজি: নং ও তারিখ
% cno
& : \textbf{C-1270625}
\hspace{1em} DT:07-082021 \\
\hspace{2em} (খ) আমদানিকারকের নাম, ঠিকানা
ও BIN নম্বর
% ac
& : \textbf{JAMUNUA ELECTRONICS
\& AUTOMOBILES LTD} \\
& \hspace{4pt} SINABHA, KALIAKAIR PS; GAZIPUR \\
& \hspace{4pt} 1750, BANGLADESH \\
& \hspace{4pt} BIN NO. 000146478-0103 \\
\hspace{2em} (গ) সিএন্ডএফ এজেন্টের নাম, ঠিকানা
ও AIN নম্বর
% ca
& : \textbf{SHAMEEM SPINNING MILLS LIMITED} \\
& \hspace{4pt} 92, HIGH LAVEL ROAD, LALKHAN BAZAR \\
& \hspace{4pt} CHITTAGONG \\
%c ain
& \hspace{4pt} AIN NO. 301 08 3417 \\
% lc
\hspace{2em} (ঘ) এল/সি নং ও তারিখ
& : 00000088821020076
\hspace{1em} DT:07-082021 \\
\hspace{2em} (ঙ) লিয়েন ব্যাংকের নাম
% bank
& : ISLAMI BANK BANGLADESH LIMITED \\
\hspace{2em} (চ) এলসিএ নং ও তারিখ
% lca
& : 332636
\hspace{1em} DT:07-082021 \\
\hspace{2em} (ছ) বি/এল নং ও তারিখ
% bl
& : SHWWSE210714406
\hspace{1em} DT:07-082021 \\
\hspace{2em} (জ) বাণিজ্যিক ইনভয়েজ নং ও তারিখ
% invoice
\hspace{2em} (ঝ) সিআরএফ নং ও ইস্যুর তারিখ
% crf
& : NON CRF \\
\hspace{2em} (ঞ) পণ্যের বিবরণ
% goods
& : ALUMINIUM PIPE COPPER PIPE CAPILLARY PIPE \\
\hspace{2em} (ট) পণ্যের পরিমাণ (একক সহ)
% wt
& : 35PKGS = 19605.30 KGS \\
\hspace{2em} (ঠ) পণ্যের এইচ.এস.কোড
% hs
& : 7608.10.00 \\
\hspace{2em} (ড) পণ্যের মূল্য
% usd
& : US\$: 1,13,716.76 \\
\hspace{2em} (ঢ) কান্ট্রি অব অরিজিন
% co
& : CHINA \\
\hspace{2em} (ণ) কান্ট্রি অব শিপমেন্ট
% cos
& : CHINA \\
\hspace{2em} (ত) জাহাজের নাম
% carrier
& : ULANGA \\
\hspace{3.8em} পালা নং, বি/এল নং
% rotation
& : ROT.NO.2021/3477,B/L NO. SHWWSE210714406 \\
\hspace{2em} (থ) মেনিফিস্ট নং
% manifest
& : 2021 / 3477 \\
\end{tabular}
\\
\\
% ns3
০৩ । \underline{\textbf{শুল্কায়ন সেকশনের পর্যালোচনাঃ-}}

\begin{description}
 \item \hspace{1em} (ক)
 \underline{\textbf{আমদানি দলিল পত্র যাচাইঃ-}}
পণ্য চালান খালাসের জন্য নিম্নবর্ণিত দলিলাদিসহ বি/ই দাখিল করা
হয়েছে যা সংশ্লিষ্ট ব্যাংকের অথরাইজ কর্মকর্তা কর্তৃক সত্যায়িত।
পরীক্ষান্তে এসব কাগজপত্র যথাযথ আছে বলে প্রতীয়মান হয়।
\\
(১) এল.সি এবং এল.সি.এ ফরম।
\\
(২) ইনভয়েস।
\\
(৩) প্যাকিং লিস্ট।
\\
(৪) বি/এল।
\\
(৫) কান্ট্রি অব অরিজিন সনদ।

\item \hspace{1em} (খ)
\underline{\textbf{আমদানি যোগ্যতা যাচাইঃ-}}
প্রচলিত আমদানিনীতি আদেশ পর্যালোচনায় দেখা যায় যে, পণ্যগুলি অবাধে আমদানিযোগ্য।
আলোচ্য চালানের ক্ষেত্রে আমদানিনীতি আদেশের প্রযোজ্য অন্যান্য শর্ত (কান্ট্রি অব অরিজিন, রেজিঃ
সার্টিফিকেট ইত্যাদি) প্রতিপালিত হয়েছে।
\item \hspace{1em} (গ)
\underline{\textbf{কায়িক পরীক্ষার প্রতিবেদন পর্যালোচনাঃ-}}
আলোচ্য পণ্য চালানটি ১০০\% কায়িক পরীক্ষার জন্য নির্বাচিত। কায়িক পরীক্ষায় ইনভয়েস,
প্যাকিং লিস্ট মোতাবেক সঠিক পাওয়া যায়। কায়িক পরীক্ষার প্রতিবেদন নথির যোগাযোগ অংশে
রক্ষিত আছে, দয়া করে দেখা যেতে পারে।
\item \hspace{1em} (ঘ)
\underline{\textbf{রাসায়নিক পরীক্ষা সংক্রান্ত মন্তব্যঃ-}}
প্রযোজ্য নয়।
\item \hspace{1em} (ঙ)
\underline{\textbf{এইচ.এস.কোড সঠিকতা যাচাইঃ-}}
ইনভয়েস প্রত্যায়িত এইচ.এস.কোড দি কাস্টমস্ এ্যাক্ট ১৯৬৯ এর FIRST SCHEDULE ও
EXPLANATORY NOTES প্রচলিত এসআরও/স্থায়ী আদেশ ইত্যাদির আলোকে পরীক্ষা করা হলো।
প্রত্যায়িত এইচ.এস.কোড যথাযথ আছে।

\item \hspace{1em} (চ)
\underline{\textbf{রেয়াতী হার বা বিশেষ মওকুফ সংক্রান্তঃ-}}
\\
(১) জাতীয় রাজস্ব বোর্ড পত্র নং-০৮.০১.০০০০.০৬৮.১৮.০০৪.১৭/১৬৯(৩),
তারিখ-২৭/০৬/২০২১ খ্রিঃ এর মাধ্যমে আলোচ্য প্রতিষ্ঠান কর্তৃক উপকরণ ও যন্ত্রাংশ
আমদানির ক্ষেত্রে আমদানি পর্যায়ে আরোপনীয় সমুদয় মূল্য সংযোজন কর (আগাম কর ব্যতিত)
ও সম্পূরক শুল্ক (প্রযোজ্য ক্ষেত্রে) বোর্ডের সিদ্ধান্তক্রমে অব্যাহতি প্রদান করা হয়েছে।
\\
(২) জাতীয় রাজস্ব বোর্ডের পত্র নথি নং-০৮.০১.০০০০.০৩৪.০২.৩০১.১৯-৪৭
তাং-১২/০৬/২০২১ খ্রিঃ এর মাধ্যমে আলোচ্য প্রতিষ্ঠান কর্তৃক উপকরণ ও যন্ত্রাংশ
আমদানির ক্ষেত্রে পরিশোধযোগ্য মূল্যের উপর ০.৮৩\% হারে উৎসে অগ্রিম আয়কর
কর্তন হবে।
\\
(৩) মূসক-৪.৩ অনুযায়ি প্রতিষ্ঠানটি Refrigerator উৎপাদনকারি প্রতিষ্ঠান হওয়ায়
এসআরও-১১৪-আইন/২০২১/০৩/কাস্টমস, তারিখ: ২৪/০৫/২০২১ ইং এর
আওতায় রেয়াতী সুবিধা প্রাপ্ত।
\\
জাতীয় রাজস্ব বোর্ডের পত্র (১) ও (২) এর আলোকে পণ্য চালানটির প্রথম আইটেম
এর ক্ষেত্রে সিপিসি ৪০০০-৬৬১ প্রযোজ্য এবং জাতীয় রাজস্ব বোর্ডের পত্র নং-(১)
এবং পরবর্তী আইটেম গুলোতে এসআরও-১১৪-আইন/২০২১/০৩/কাস্টমস,
তারিখ: ২৪/০৫/২০২১ ইং এর আওতায় সিপিসি ৪০০০-৪০৯ প্রযোজ্য
\item \hspace{1em} (ছ)
\underline{\textbf{অভিযোগ সংক্রান্তঃ-}} এ চালানে জাতীয় রাজস্ব বোর্ডের/গোয়েন্দা
সংস্থার অভিযোগ পাওয়া যায় নাই।
\item \hspace{1em} (জ)
\underline{\textbf{ন্যায় নির্ণয় সংক্রান্তঃ-}} প্রযোজ্য নয়।
\item \hspace{1em} (ঝ)
\underline{\textbf{উল্লেখ করার মত প্রাসঙ্গিক অন্যান্য বিষয়ঃ-}} আমদানিকারক
মূসক ২.৩, ৪.৩ অন-লাইন আইআরসি কপি, ভ্যাট রিটার্ণ এবং ভ্যাট প্রত্যয়নপত্র দাখিল
করেছেন।
\end{description}
%ns4
০৪ । \underline{\textbf{শুল্কায়ন সম্পর্কিত প্রস্তাবঃ-}}
শতভাগ কায়িক পরীক্ষার প্রতিবেদন এবং পণ্য চালান সংক্রান্ত আমদানি দলিলপত্র
ইনভয়েস পর্যালোচনা পূর্বক পণ্যের বর্ণনা, পরিমাণ, এইচ.এস.কোড, ঘোষিত মূল্য,
সমসাময়িককালের অনুরূপ পণ্যের একক মূল্যসহ নিম্নের ছকে শুল্কায়নের প্রস্তাব উপস্থাপন
করা হলো:
\\
% tax table
\noindent
\begin{longtable}{|c|c|c|c|c|c|c|c|}

\hline
\textbf{
\scriptsize{
\makecell{
SL \\ NO
}
}
}
&
\textbf{
\scriptsize{
\makecell{
DESCRIPTION OF GOODS
}
}
}
&
\textbf{
\scriptsize{
\makecell{
QTY
}
}
}
& \textbf{
\scriptsize{
\makecell{
DECLARED \\ H.S CODE
}
}
}
&
\textbf{
\scriptsize{
\makecell{
PROPOSED \\ H.S CODE
}
}
}
&
\textbf{
\scriptsize{
\makecell{
 DECLARED \\ VALUE \\ (USD/KG)
}
}
}
&
\textbf{
\scriptsize{
\makecell{
 DATABASE \\ VALUE \\ (USD/KG) \\ OF \\
 ASYCUDA \\ WORLD \\ SYSTEM
}
}
}
&
\textbf{
\scriptsize{
\makecell{
 PROPOSED \\ VALUE \\ (USD/KG)
}
}
} \\
\hline
% row
\scriptsize{
\makecell{
01
}
}
&
\scriptsize{
\makecell{
02
}
}
&
\scriptsize{
\makecell{
03
}
}
&
\scriptsize{
\makecell{
04
}
}
&
\scriptsize{
\makecell{
05
}
}
&
\scriptsize{
\makecell{
06
}
}
&
\scriptsize{
\makecell{
07
}
}
&
\scriptsize{
\makecell{
08
}
} \\
\hline
\end{longtable}
%ns5
\underline{\textbf{শুল্কায়নযোগ্য মূল্য নিরুপণঃ-}} (ক) পন্য চালানটির
দাখিলকৃত দলিলাদি বিশ্লেষণ করে দেখা যায় পণ্যের ঘোষিত মূল্য সমসাময়িককালের
শুল্কায়ন মূল্য হতে অনেক কম। আমদানিকারক তার মনোনীত সি এন্ড এফ এজেন্ট
বিল অব এন্ট্রির সাথে শুল্ক মূল্যায়ন (আমদানি পন্যের মূল্য নির্ধারণ) বিধিমালা,
২০০০ (এসআর নং-৫৭/২০০০) এর বিধি-১১(১) (ক) অনুযায়ী কোন মূল্য ঘোষনা
ফরম দাখিল করেননি। কিন্তু উক্ত বিধিমালার বিধি ১১(১) (খ) অনুযায়ি প্রস্তুতকারি
উৎপাদনকারির নিকট হতে সংগৃহীত ইনভয়েসসহ অন্য কোন বিবৃতি, তথ্য অথবা দলিল
যা আমদানিকৃত পণ্যের মূল্য নির্ধারণের জন্য আবশ্যক ছিল তা দাখিল করেননি।
এছাড়া ইনভয়েস, প্যাকিং লিস্ট, এলসি, এলসিএ ফরম, প্রোফরমা ইনভয়েস ব্যতীত নিরপেক্ষ
উৎস হতে পণ্যের মূল্যের স্বপক্ষে অন্য কোন দলিলও দাখিল করেননি বিধায় ঘোষিত মূল্যকে
বিনিময় মূল্য (Transaction Value) হিসেবে গ্রহন করার কোন সুযোগ নেই।
\\
\\
%ns5
০৫ । আমদানিকারক Industrial IRC Holder উৎপাদনকারি (Manufacturing)
হিসাবে হালনাগাদ IRC মূসক-৪.৩; মূসক-২.৩ ও ১৩ ডিজিটের নিবন্ধন পত্র
দাখিল করেছেন। মূসক-২.৩ অনুযায়ী প্রতিষ্ঠানের কার্যক্রম:
Manufacturing Services Import Export
উৎপাদনকারি প্রতিষ্ঠান হিসাবে মূসক-৪.৩ তে আমদানি পণ্যগুলো
অন্তর্ভূক্ত রয়েছে। মূসক-২.৩ অনুযায়ি প্রতিষ্ঠানটি উৎপাদনকারি হিসেবে
নিবন্ধিত।
\\
\\
%ns6
০৬ । (১) জাতীয় রাজস্ব বোর্ড পত্র নং-০৮.০১.০০০০.০৬৮.১৮.০০৪.১৭/১৬৯(৩),
তারিখ-২৭/০৬/২০২১ খ্রিঃ এর মাধ্যমে আলোচ্য প্রতিষ্ঠান কর্তৃক উপকরণ ও যন্ত্রাংশ
আমদানির ক্ষেত্রে আমদানি পর্যায়ে আরোপনীয় সমুদয় মূল্য সংযোজন কর (আগাম কর ব্যতিত)
ও সম্পূরক শুল্ক (প্রযোজ্য ক্ষেত্রে) বোর্ডের সিদ্ধান্তক্রমে অব্যাহতি প্রদান করা হয়েছে।
\\
(২) জাতীয় রাজস্ব বোর্ডের পত্র নথি নং-০৮.০১.০০০০.০৩৪.০২.৩০১.১৯-৪৭
তাং-১২/০৬/২০২১ খ্রিঃ এর মাধ্যমে আলোচ্য প্রতিষ্ঠান কর্তৃক উপকরণ ও যন্ত্রাংশ
আমদানির ক্ষেত্রে পরিশোধযোগ্য মূল্যের উপর ০.৮৩\% হারে উৎসে অগ্রিম আয়কর
কর্তন হবে।
\\
\\
%ns7
০৭ । এমতাবস্থায়, নোটানুচ্ছেদ-৪ এ উল্লেখিত ছকের প্রস্তাবিত মূল্যে,
এইচ.এস.কোড ও সিপিসি তে চালানটি শুল্ক মূল্যায়ন বিধিমালা ২০০০
এর বিধি ০৪ ও ০৫ মোতাবেক শুল্কায়ন প্রস্তাব অনুমোদনের জন্য পেশ
করা হলো।
\\
\\
\\
\\
সদয় অবগতি ও আদেশার্থে উপস্থাপন করা হলো।

\thispagestyle{laststyle}

\end{document}

