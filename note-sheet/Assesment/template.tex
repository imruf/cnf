\documentclass[10pt]{article}
\usepackage[a4paper,hmargin=0.5in,vmargin=0.25in,noheadfoot]{geometry}
\usepackage[none]{hyphenat}
\usepackage{fontspec}
\usepackage[]{amsmath}
\usepackage{chemfig}
\renewcommand*\printatom[1]{\ensuremath{\mathsf{#1}}}
\usepackage{modiagram}
\usepackage{titlesec}
\usepackage[banglamainfont=Kalpurush,
            banglattfont=SolaimanLipi,
            % feature=1,
            changecounternumbering=0
           ]{latexbangla}
\newfontfamily\bn[Language=Bengali,Script=Bengali,Scale=0.8]{ABShaplaBeta}

\author{ Masud Parvez }
\title{Basic \LaTeX{}  Template}
\date{\today}

\begin{document}

\maketitle

\begin{abstract}
Don't need paper's abstract \ldots
\end{abstract}

\section{Introduction}
This is a basic \LaTeX{} template.\\

\begin{itemize}
\item \ref{Language} Language/ভাষা
\item \ref{Formatting} Formatting.
\item \ref{Lists} Lists.
\item \ref{Reference} Reference.
\end{itemize}

\section{Bengali Text Writing\label{Language}}
\subsection{Latexbangla Package }
\LaTeX{} Bangla has complete bangla support
with some issues.
fix numerals issue by editing latexbangla.sty
line 139 \verb|\setmainlanguage[numerals=Bengali, changecounternumbering=true]{bengali}|

\begin{verbatim}
\usepackage[banglamainfont=Kalpurush,
            banglattfont=SolaimanLipi,
            % feature=1,
            % changecounternumbering=0
           ]{latexbangla}
\end{verbatim}

\subsection{Using begin/end Environment}
\begin{verbatim}
\usepackage{xunicode}
\usepackage{polyglossia}
\setmainlanguage{english}
\setmainfont[Language=English,Script=Latin]{Liberation Sans}
\setsansfont[Language=English,Script=Latin]{Liberation Serif}
\setmonofont[Language=English,Script=Latin]{Liberation Mono}
\setotherlanguages{sanskrit,bengali}
\newfontfamily\bengalifont[Language=Bengali,Script=Bengali,Scale=0.8]{Lohit Bengali}
\newfontfamily\bengalifontsf[Language=Bengali,Script=Bengali,Scale=0.8]{Lohit Bengali}
\newfontfamily\bengalifonttt[Language=Bengali,Script=Bengali,Scale=0.8]{Lohit Bengali}
\end{verbatim}
\verb|\begin{bengali}|
\\
এই ব্লকে বাংলা লিখতে হবে
\\
\verb|\end{bengali}|

\subsection{Using Bracket Block}
\verb|\newfontfamily\bn[Language=Bengali,Script=Bengali,Scale=0.8]{Kalpurush}|
\verb|{\bn| {\bn এই ব্র্যাকেটে বাংলা লিখতে হবে \verb|}|

\section{Formatting\label{Formatting}}
In this section basic \LaTeX{} formatting will be discussed.

To format section style use titlesec
package with titleformat command \\

\subsection{Paragraph}

\verb|\par|
command will create a paragraph.\par
This is a paragraph. Created using
\verb|\par|
command.\\
    This line is not a Paragraph.

This is a Paragraph.
Created using two return or blank line.

\subsubsection{Line Break}
\verb|\newline or \\| Will create new line\\
Hello
This line is not a new line. \\
This new line was created
using \verb|\\| \newline
This new line was created
using \verb|\newline|

\subsubsection{Spacing}
This is \hspace{5cm} 5cm horizontal space.
Created using
\hspace*{5cm} 5cm horizontal space
at the beginning of a line.
\verb|\hspace{5cm}| \\
This is\\
Leftside \hfill Rightside \\
of a page in a line. Created using
\verb|\hfill| \\
Similarly, \\
\verb|\vspace{value}| \\
Will create blank
vertical space of desired length.\\
\verb|\vfill| \\
will create blank
vertical space
in a page
and start new text from
bottom of the page. \\
\verb|\smallskip \medskip \bigskip| \\
are some vertical spacing command.

\subsubsection{Formatting Paragraph And  Lines}
\verb|\renewcommand{\baselinestretch}{1.5}| \\
This will set space between lines. \\
\verb|\setlength{\baselineskip}{value}| \\
Minimum space between the bottom of
two successive lines in a paragraph. \\
\verb|\setlength{\parindent}{4em}| \\
This will set paragraph indent. \\
\verb|\setlength{\parskip}{1em}| \\
This will set space between two paragraph.

\subsubsection{Alignment Of Paragraph And Lines}
Default \LaTeX{} alignment is justified. \\
\verb|\usepackage[document]{ragged2e}| \\
Will align text to the left.
Command of this package are
\verb|\\raggedright \\raggedleft \centering \justify| \\
Use between begin/end environment.

\subsubsection{Page Break}
\verb|\clearpage or \newpage| \\
Will create new page.
newpage command will be safer.

\subsection{Normal Text}
This is a normal text.

\subsection{Bold Text}
\textbf{This is a Bold text.}

\subsection{Italic Text}
\textit{This is an Italicized text.}

\subsection{Emphatic Text}
\emph{This is an Emphatic text.}

\subsection{Underline Text}
\underline{This is an Underlined text.}

\subsection{Teletype Text}
\texttt{This is a Teletype text}

\subsection{Quoted Text}
This is not a perfect double "Quoted" text. \\
This is not a perfect single 'Quoted' text. \\
This is a perfect double ``Quoted'' text. \\
This is a perfect single `Quoted' text.

\section{Lists\label{Lists}}
Change numbering style with\\
\verb|\renewcommand{\labelenumi}{\arabic{enumi}}| \\
label options are
arabic,
roman,
Roman,
alph,
Alph\\ \\
enumi-for 1st level\\
enumii-for 2nd level\\
enumiii-for 3rd level\\
enumiv-for 4th level\\
start 2nd level counter from
number 5 with\\
\verb|\setcounter{enumii}{4}|
\\ \\
Change non numbered list style
with\\
\verb|\renewcommand{\labelitemi}{$\blacksquare$}|\\
label item options are
textbullet,
textendash,
textasteriskcentered,
textperiodcentered,
blacksquare,
square.\\
labelitemi-for 1st level\\
labelitemii-for 2nd level\\
labelitemiii-for 3rd level\\
labelitemiv-for 4th level\\

\subsection{Numbered List}
\begin{enumerate}
\item This is 1st item of enumerate or numbered list.
\item This is 2nd item of enumerate or numbered list.
\item This is 3rd item of enumerate or numbered list.
\end{enumerate}

\subsection{Non Numbered List}
\verb|\renewcommand{\labelitemi}{$\textendash$}| change initial bullet style to dash.

\begin{itemize}
\item This is 1st item of itemize or non numbered list.
\item This is 2nd item of itemize or non numbered list.
\item This is 3rd item of itemize or non numbered list.
\end{itemize}

\subsection{Numbered Sub List}
\begin{enumerate}
\item This is numbered list 1.
\begin{enumerate}
\item This is numbered sub list 1.
\item This is numbered sub list 2.
\end{enumerate}
\item This is numbered list 2.
\end{enumerate}

\subsection{Non Numbered Sub List}
\begin{itemize}
\item This is non numbered list 1.
\begin{itemize}
\item This is non numbered sub list 1.
\item This is non numbered sub list 2.
\end{itemize}
\item This is non numbered list 2.
\end{itemize}

\subsection{List Without Bullet}
\begin{description}
 \item [Nested List 1] This is list 1 without bullet point.
 \item [Nested List 2] This is list 2 without bullet point.
\end{description}

\section{Reference\label{Reference}}
To add reference define label with
\verb=\label{label name}= and call this lable with
\verb=\ref{label name}=

\end{document}
