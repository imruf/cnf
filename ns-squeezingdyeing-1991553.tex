\documentclass[12pt]{article}
\usepackage[legalpaper,
            lmargin=1in,rmargin=0.5in,
            bmargin=0.5in,tmargin=1in]{geometry}
\usepackage{fontspec}
\usepackage{titlesec}
\usepackage{multirow}
\usepackage[colorlinks=true,urlcolor=black]{hyperref}
\usepackage{graphicx}
\usepackage{array}
\usepackage{makecell}
\usepackage{fancyhdr}
\usepackage[none]{hyphenat}
\usepackage{longtable}
\usepackage[dvipsnames]{xcolor}
\usepackage[banglamainfont=Kalpurush,
            banglattfont=SolaimanLipi,
            % feature=1,
            % changecounternumbering=0
           ]{latexbangla}

% oncehsis imp file name
\newcommand{\eif}{নথি নং- এস \hspace{4em}/স্টাফ/কাস/\hspace{5em}/অংশ-}
\newcommand{\jdlf}{নথি নং- এস৫-২৪৫/স্টাফ/কাস/০৬-০৭/অংশ-২}
\newcommand{\jdwlf}{নথি নং- এস৫-১৭০/স্টাফ/কাস/১৪-১৫/অংশ-২}
\newcommand{\jealf}{নথি নং- এস-৫-৫৮৫/স্টাফ/কাস/২০১৩-২০১৪/অংশ-১২}
\newcommand{\jfclf}{নথি নং- এস৫-৫১৯/স্টাফ/কাস/২০-২১}
\newcommand{\jsmlf}{নথি নং- এস৫-৬৯১/স্টাফ/কাস/২০১৭-২০১৮/অংশ-০৩}
\newcommand{\jtrif}{নথি নং- এস-৫-১২১০/স্টাফ/কাস/১৬-১৭/অংশ-১}
\newcommand{\jwelf}{নথি নং- এস-৫-১১৪/স্টাফ/কাস/২০১৮-১৯/}
\newcommand{\scmlf}{নথি নং- এস-৫-১৬৩/স্টাফ/কাস/০৬-০৭/অংশ-৫}
\newcommand{\ssmlf}{নথি নং- এস-৫-১৪৪/স্টাফ/কাস/০৬-০৭/অংশ-১১}
\newcommand{\tdjf}{নথি নং- এস-৫-৪৪৩/স্টাফ/কাস/০৬-০৭/অংশ-}
\newcommand{\htff}{নথি নং- এস-৫-৬৮৪/স্টাফ/কাস/২০১৭-১৮/অংশ-৩}

% fileno
\newcommand{\filenou}{
\begin{minipage}[t]{0.60\linewidth}
\hspace{1em}
\end{minipage}
\begin{minipage}[t]{0.40\linewidth}
\underline{\fileno}
\end{minipage}
}


% new page
\newcommand{\nfpage}{
\newpage
\small
{\filenou}
}

% imp name
\newcommand{\ssml}{SHAMEEM SPINNING MILLS LTD.}
\newcommand{\ssmla}{SHAFIPUR, KALIAKAIR, GAZIPUR-1751, BANGLADESH.}
\newcommand{\scml}{SHAMEEM COMPOSITE MILLS LTD.}
\newcommand{\scmla}{SHAFIPUR, KALIAKAIR, GAZIPUR-1751, BANGLADESH}
\newcommand{\jsml}{JAMUNA SPINNING MILLS LTD.}
\newcommand{\jsmla}{SHAFIPUR, KALIAKAIR, GAZIPUR-1751, BANGLADESH.}
\newcommand{\jsmlaut}{JAMUNA SPINNING MILLS LTD. UNIT-2}
\newcommand{\jsmlauta}{BEJURA, SOUTH BEJURA, MADHABPUR, HOBIGONJ-3331, BANGLADESH.}
\newcommand{\jdl}{JAMUNA DENIMS LTD.}
\newcommand{\jdla}{SHAFIPUR, KALIAKAIR, GAZIPUR-1751, BANGLADESH.}
\newcommand{\jeal}{JAMUNA ELECTRONICS \& AUTOMOBILES LTD.}
\newcommand{\jeala}{SINABHA, KALIAKAIR PS, GAZIPUR-1750, BANGLADESH}
\newcommand{\jdwl}{JAMUNA DENIMS WEAVING LTD.}
\newcommand{\jdwla}{KASHIMPUR ROAD, JARUN, KONABARI PS, GAZIPUR-1751, BD}
\newcommand{\jfcl}{JAMUNA FAN AND CABLES LTD.}
\newcommand{\jfcla}{KASHIMPUR ROAD, JARUN, KONABARI PS, GAZIPUR-1751, BD}
\newcommand{\htf}{HOORAIN HTF LTD.}
\newcommand{\htfa}{BEJURA, SOUTH BEJURA, MADHABPUR, HOBIGONJ-3331, BANGLADESH.}
\newcommand{\tdj}{M/S THE DAILY JUGANTOR}
\newcommand{\tdja}{KA-244, KURIL, PROGATI SARANI, VATARA PS, DHAKA-1229, BANGLADESH}
\newcommand{\jtri}{JAMUNA TYRE AND RUBBER INDUSTRIES}
\newcommand{\jtria}{BEJURA, SOUTH BEJURA, MADHABPUR, HOBIGONJ}
\newcommand{\jwel}{JAMUNA WELDING ELECTRODE LTD.}
\newcommand{\jwela}{CHOYDANA, HAZIRPUKUR; GAZIPUR SADAR, GAZIPUR-1704; BANGLADESH}
\newcommand{\cbl}{CROWN BEVERAGE LIMITED}
\newcommand{\cbla}{SHAFIPUR, KALIAKAIR, GAZIPUR-1751, BANGLADESH}
\newcommand{\jpml}{JAMUNA PAPER MILLS LIMITED}
\newcommand{\jpmla}{BEJURA, SOUTH BEJURA MADHABPUR, HOBIGONJ, BANGLADESH}

% imp bin
\newcommand{\jealbin}{000146478-0103}
\newcommand{\tdjbin}{001960365-0101}
\newcommand{\jsmlbin}{000144425-0103}
\newcommand{\jfclbin}{001753334-0103}
\newcommand{\scmlbin}{000151754-0103}
\newcommand{\htfbin}{000146478-0103}
\newcommand{\cblbin}{000149506-0103}

% cnf name
\newcommand{\cnfn}{SHAMEEM SPINNING MILLS LTD.}
\newcommand{\cnfadd}{92, HIGH LEVEL ROAD, LALKHAN BAZAR, CHITTAGONG}
\newcommand{\cnfain}{301 08 3417}

% sros
\newcommand{\srofs}{এসআরও নং- ৫৭- আইন/২০০০/১৮২১/শুল্ক}
\newcommand{\srofsd}{তারিখ: ২৩/০২/২০০০ খ্রি:}
\newcommand{\srooot}{এসআরও নং- ১১৩- আইন/২০২১/০২/কাস্টমস}
\newcommand{\sroootd}{তারিখ: ২৪/০৫/২০২১ খ্রি:}
\newcommand{\srooof}{এসআরও নং- ১১৪- আইন/২০২১/০৩/কাস্টমস}
\newcommand{\srooofd}{তারিখ: ২৪/০৫/২০২১ খ্রি:}
\newcommand{\srootz}{এসআরও নং- ১২০- আইন/২০২১/০৯/কাস্টমস}
\newcommand{\srootzd}{তারিখ: ২৪/০৫/২০২১ খ্রিঃ}

\newcommand{\sroott}{এসআরও নং- ১৩৩- আইন/২০২১/২২/কাস্টমস}
\newcommand{\sroottd}{তারিখ: ২৪/০৫/২০২১ খ্রিঃ}

\newcommand{\srotsz}{এসআরও নং- ৩৬০- আইন/২০১৩/২৪৬৮/কাস্টমস}
\newcommand{\srotszd}{তারিখ: ২৫/১১/২০১৩ খ্রি:}

% cpcs
\newcommand{\cpcofs}{CPC- 4000/157}
\newcommand{\cpcost}{CPC- 4000/173}
\newcommand{\cpcttz}{CPC- 4000/220}
\newcommand{\cpcfzo}{CPC- 4000/401}
\newcommand{\cpcfzn}{CPC- 4000/409}
\newcommand{\cpcsso}{CPC- 4000/661}

% nbrl
\newcommand{\nbrosn}{জাতীয় রাজস্ব বোর্ডের পত্র নং- ০৮.০১.০০০০.০৬৮.১৮.০০৪.১৭/১৬৯}
\newcommand{\nbrosnd}{তারিখ: ২৭/০৬/২০২১ ইং}
\newcommand{\nbrosnt}{জাতীয় রাজস্ব বোর্ডের পত্র নং- ০৮.০১.০০০০.০৬৮.১৮.০০৪.১৭/১৬৯(৩)}
\newcommand{\nbrosntd}{তারিখ: ২৭/০৬/২০২১ খ্রি:}
\newcommand{\nbrfs}{জাতীয় রাজস্ব বোর্ডের পত্র নং- ০৮.০১.০০০০.০৩৪.০২.৩০১.১৯-৪৭}
\newcommand{\nbrfsd}{তারিখ: ১২/০৭/২০২১ খ্রি:}

% imp reg name
\newcommand{\ssmlreg}{২৩১৩৯৫০৬১৮০-এইচ, তারিখ: ০৯/০১/২০১৩}
\newcommand{\scmlreg}{৯৭০৪০১৭-এইচ তারিখ: ২১/০৪/১৯৯৭}
\newcommand{\jealreg}{এল-২৯৩০১৩০১৩৪৩১-এইচ, তারিখ: ০৯/০১/২০১৩}
\newcommand{\eireg}{\hspace{10em} তারিখ: \hspace{5em}}

% imp irc no
\newcommand{\jealirc}{260326120426720}
\newcommand{\scmlirc}{260326120041719}
\newcommand{\jfclirc}{260326120515020}

% san name
\newcommand{\maersk}{MAERSK BANGLADESH LIMITED}
\newcommand{\maerska}{58, AGRABAD COMMERCIAL AREA (3RD FLOOR), CHITTAGONG, 4100.}
\newcommand{\apl}{APL (BANGLADESH) PVT.LTD}
\newcommand{\apla}{PLOT NO. 30, 3RD FLOOR OF SURAIYA MANSION, AGRABAD, CHITTAGONG.}
\newcommand{\baridhi}{BARIDHI SHIPPING LINES LTD}
\newcommand{\baridhia}{3/F HRC BHABAN, 64-66 AGRABAD C/A, CHITTAGONG.}
\newcommand{\continentalbd}{CONTINENTAL TRADERS (BD) LIMITED}
\newcommand{\continentalbda}{73, AGRABAD C/A, CHITTAGONG.}
\newcommand{\continentaltr}{CONTINENTAL TRADERS (BD) LIMITED}
\newcommand{\continentaltra}{IQBAL BHABAN, AGRABAD C/A, CHITTAGONG.}
\newcommand{\gbx}{GBX LOGISTICS LTD}
\newcommand{\gbxa}{AYUB TRADE CENTER(1ST FLOOR), 1269/B, SK MUJIB ROAD, AGRABAD C/A, CHITTAGONG.}
\newcommand{\transmarine}{TRANSMARINE LOGISTICS LTD}
\newcommand{\transmarinea}{B.M.HEIGHTS(4TH FLOOR), 318, SK, MUJIB ROAD, AGRABAD C/A, CHITTAGONG.}
\newcommand{\trident}{TRIDENT SHIPPING LINE LTD}
\newcommand{\tridenta}{AKHTARUZZAMAN CENTER, 6TH FLOOR 21/22 AGRABAD, CHITTAGONG.}
\newcommand{\msc}{MSC MEDITERRANEAN SHIPP.CO.BD.LTD}
\newcommand{\msca}{IIUC TOWER, 4TH FLOOR, 1700/A SK.MUJIB ROAD, PLOT-09, AGRABAD, CHITTAGONG.}
\newcommand{\alviline}{ALVILINE BANGLADESH LIMITED}
\newcommand{\alvilinea}{78, AGRABAD C/A, MACCA MADINA TRADE CENTER, 9TH FLOOR, CHITTAGONG}
\newcommand{\ocean}{OCEAN NETWORK EXPRESS (BD) LTD}
\newcommand{\oceana}{IIUC TOWER (10TH FLOOR), 1700/A, PLOT-9, SK.MUJIB ROAD, AGRABAD C/A, CHITTAGONG}
\newcommand{\vega}{VEGA MARINE PVT LIMITED}
\newcommand{\vegaa}{DAAR-E SHAHIDI, 4TH FLOOR, 69 AGRABAD C/A, CTG}
\newcommand{\mega}{MEGATREND SHIPPING LINES LTD.}
\newcommand{\megaa}{MAKKAH MADINAH TRADE CENTER (16TH FLOOR), 78, AGRABAD, CHITTAGONG}
\newcommand{\famfa}{FAMFA SOLUTION LIMITED}
\newcommand{\famfaa}{BONANI, AGRABAD, CHITTAGONG}
\newcommand{\reliance}{RELIANCE SHIPPING SERVICES}
\newcommand{\reliancea}{34 AGRABAD C/A, CHITTAGONG 4100, BANGLADESH}
%\newcommand{\alvilinea}{}


% mujib logo
\newcommand{\my}{\includegraphics[height=3.2em]{pic/my.png}}

% slogan
\fancypagestyle{slogan}
{
\fancyhf{}
\renewcommand{\headrulewidth}{0pt}
% header
\lhead{
\framebox[1.1\width]{\footnotesize{``উন্নয়নের অক্সিজেন রাজস্ব''}}
}
\rhead{
\my
\\
\framebox[1.1\width]{\footnotesize{``জনকল্যানে রাজস্ব''}}
}
}

% customs
\newcommand{\tca}{The Customs Act, 1969}





\pagestyle{fancy}
\fancyhf{}
\renewcommand{\headrulewidth}{0pt}
% header
\chead{
\underline{{পৃষ্ঠা - \thepage}}
\\
}
\rhead{{\nfilenou}}
% footer
%\rfoot{চলমান পৃষ্ঠা-\thepage}

\fancypagestyle{laststyle}
{
   \fancyfoot[R]{}
}

\fancypagestyle{slogan}
{
\fancyhf{}
\renewcommand{\headrulewidth}{0pt}
% header
\lhead{
\framebox[1.1\width]{\footnotesize{``উন্নয়নের অক্সিজেন রাজস্ব''}}
}
\rhead{
\my
\\
\framebox[1.1\width]{\footnotesize{``জনকল্যানে রাজস্ব''}}
}
}

\newcommand{\fileno}{নথি নং - ১২৮৭/এপি/সেকশন-৫(বি)/২০২১-২০২২}
\newcommand{\nfilenou}{\underline{\fileno}}
\newcommand{\fdt}{\hspace*{3em} তারিখ: \hspace{2.4em} /১২/২১}

\newcommand{\good}{SQUEEZING DYEING STAINLESS ROLLER, CHAIN LINK ETC.}
\newcommand{\pkg}{2 Pallet=2,329.00 KG}
\newcommand{\hscode}{84519000, 73159000, 84818090, 40169300 ETC.}
\newcommand{\price}{US\$ 31,752.00}

\newcommand{\co}{NEW TAIWAN}
\newcommand{\coship}{NEW TAIWAN}

\newcommand{\vessel}{MV.MAERSK BINTULU}

\newcommand{\blno}{KEL/CGP/2111507}
\newcommand{\bldt}{18.11.21}

\newcommand{\beno}{C-1991553}
\newcommand{\bedt}{09.12.2021}
\newcommand{\menifest}{2021/5591}

\newcommand{\lcno}{0000174121010724}
\newcommand{\lcdt}{11.10.21}
\newcommand{\lcano}{268823}
\newcommand{\lcadt}{\lcdt}
\newcommand{\lienbank}{MERCANTILE BANK LIMITED}

\newcommand{\invno}{PL211004}
\newcommand{\invdt}{09.11.2021}

\newcommand{\impn}{\jdwl}
\newcommand{\impadd}{\jdwla}
\newcommand{\impbin}{}
\newcommand{\impoldbin}{}

\newcommand{\crf}{NON CRF}
\newcommand{\crfdt}{}

\newcommand{\ircno}{}
\newcommand{\ircrenewdt}{}
\newcommand{\musokr}{নভেম্বর/২১}
\newcommand{\btmaltno}{প্রত্যয়ন পত্র নং- ২৮৬৩৫}
\newcommand{\btmaltnodt}{তারিখ:  ০৪/১২/২০২২ খ্রি:}
\newcommand{\rodt}{তারিখ: \hspace{2.0em}/০১/২০২২ ইং}

\newcommand{\taxtab}{
\begin{longtable}{|c|c|c|c|c|c|c|c|}
\hline
\textbf{
\makecell{
ক্রঃ \\ নং
}
}
&
\textbf{
\makecell{
পণ্যের বর্ণনা
}
}
&
\textbf{
\makecell{
পরিমাণ
}
}
& \textbf{
\makecell{
ইনভয়েস
\\
ঘোষিত
\\
এইচএসকোড
\\
ও শুল্কহার
}
}
&
\textbf{
\makecell{
প্রকৃত
\\
এইচএসকোড
\\
ও শুল্কহার
}
}
&
\textbf{
\makecell{
ইনভয়েস
\\
প্রত্যায়িত
\\
একক মূল্য
\\
(US\$)
}
}
&
\textbf{
\makecell{
শুল্কায়ন বিধিমালা ২০০০
\\
-এর বিধি
৪ ও ৫ অনুযায়ী
\\
পণ্যের
একক মূল্য
(US\$)
}
}
&
\textbf{
\makecell{
প্রস্তাবিত
\\
একক মূল্য
\\
(US\$)
}
} \\
\hline
% row
\makecell{
01
}
&
\makecell{
SQUEEZING DYEING
\\
STAINLESS ROLLER
\\
C/O. {\co}
}
&
\makecell{
1,786.00
\\
KGS
}
&
\makecell{
84519000
\\
CD-1\%
\\
SRO 113/21
\\
CPC-4000/220
}
&
\makecell{
84519000
\\
CD-1\%
\\
SRO 113/21
\\
CPC-4000/220
}
&
\makecell{
US\$
\\
9.76/KG
}
&
\makecell{
US\$
\\
1.80/KG
\\
SECTION
VALUE
}
&
\makecell{
US\$
\\
9.76/KG
} \\
\hline
% row
\makecell{
02
}
&
\makecell{
CHAIN LINK
\\
C/O. {\co}
}
&
\makecell{
396.00
\\
KGS
}
&
\makecell{
73158900
\\
CD-25\%
\\
RD-3\%
\\
VAT-15\%
\\
AIT-5\%
\\
AT-5\%
\\
CPC-4000/173
}
&
\makecell{
73159000
\\
CD-1\%
\\
AIT-5\%
\\
AT-5\%
\\
SRO 120/21
\\
CPC-4000/157
}
&
\makecell{
US\$
\\
10.81/KG
}
&
\makecell{
US\$
\\
1.80/KG
\\
DATA BASE
VALUE
\\
C-\hspace{4em}
\\
DT:\hspace{4em}
}
&
\makecell{
US\$
\\
10.81/KG
} \\
\hline
% row
\makecell{
03
}
&
\makecell{
ROTARY JOINT
\\
C/O. {\co}
}
&
\makecell{
40.00
\\
KGS
}
&
\makecell{
84519000
\\
CD-1\%
\\
SRO 113/21
\\
CPC-4000/220
}
&
\makecell{
84519000
\\
CD-1\%
\\
SRO 113/21
\\
CPC-4000/220
}
&
\makecell{
US\$
\\
38.30/KG
}
&
\makecell{
US\$
\\
1.80/KG
\\
SECTION
VALUE
}
&
\makecell{
US\$
\\
38.30/KG
} \\
\hline
% row
\makecell{
04
}
&
\makecell{
STEAM TRAP
\\
C/O. {\co}
}
&
\makecell{
30.00
\\
KGS
}
&
\makecell{
84818090
\\
CD-10\%
\\
VAT-15\%
\\
AIT-5\%
\\
AT-5\%
\\
CPC-4000/173
}
&
\makecell{
84818090
\\
CD-10\%
\\
VAT-15\%
\\
AIT-5\%
\\
AT-5\%
\\
CPC-4000/173
}
&
\makecell{
US\$
\\
27.67/KG
}
&
\makecell{
US\$
\\
4.00/KG
\\
SECTION
VALUE
}
&
\makecell{
US\$
\\
27.67/KG
} \\
\hline
% row
\makecell{
05
}
&
\makecell{
CYLINDER-Y
\\
CONTROL VALVE
\\
C/O. {\co}
}
&
\makecell{
15.60
\\
KGS
}
&
\makecell{
84818090
\\
CD-1\%
\\
AIT-5\%
\\
AT-5\%
\\
SRO 120/21
\\
CPC-4000/157
}
&
\makecell{
84818029
\\
CD-25\%
\\
RD-3\%
\\
VAT-15\%
\\
AIT-5\%
\\
AT-5\%
\\
CPC-4000/173
}
&
\makecell{
US\$
\\
172.12/KG
}
&
\makecell{
US\$
\\
1.80/KG
\\
DATA BASE
VALUE
\\
C-\hspace{4em}
\\
DT:\hspace{4em}
}
&
\makecell{
US\$
\\
172.12/KG
} \\
\hline
% row
\makecell{
06
}
&
\makecell{
OIL SEAL
FOR
\\
HYDRAULIC CYLINDER
\\
C/O. {\co}
}
&
\makecell{
3.80
\\
KGS
}
&
\makecell{
40169300
\\
CD-5\%
\\
VAT-15\%
\\
AIT-5\%
\\
AT-5\%
\\
CPC-4000/173
}
&
\makecell{
40169300
\\
CD-5\%
\\
VAT-15\%
\\
AIT-5\%
\\
AT-5\%
\\
CPC-4000/173
}
&
\makecell{
US\$
\\
506.45/KG
}
&
\makecell{
US\$
\\
2.80/KG
\\
SECTION
VALUE
\\
C-\hspace{4em}
\\
DT:\hspace{4em}
}
&
\makecell{
US\$
\\
506.45/KG
} \\
\hline
% row
\makecell{
07
}
&
\makecell{
COUPLING
\\
C/O. {\co}
}
&
\makecell{
10.70
\\
KGS
}
&
\makecell{
84836000
\\
CD-1\%
\\
SRO 113/21
\\
CPC-4000/220
}
&
\makecell{
73079990
\\
CD-1\%
\\
AIT-5\%
\\
AT-5\%
\\
SRO 120/21
\\
CPC-4000/157
}
&
\makecell{
US\$
\\
98.00/KG
}
&
\makecell{
US\$
\\
2.80/KG
\\
DATA BASE
VALUE
\\
C-\hspace{4em}
\\
DT:\hspace{4em}
}
&
\makecell{
US\$
\\
98.00/KG
} \\
\hline
% row
\makecell{
08
}
&
\makecell{
TORQUE LIMITTER
\\
C/O. {\co}
}
&
\makecell{
1.20
\\
KGS
}
&
\makecell{
85369090
\\
CD-25\%
\\
RD-3\%
\\
SD-10\%
\\
CD-25\%
\\
VAT-15\%
\\
AIT-5\%
\\
AT-5\%
\\
CPC-4000/173
}
&
\makecell{
85369090
\\
CD-25\%
\\
RD-3\%
\\
SD-10\%
\\
CD-25\%
\\
VAT-15\%
\\
AIT-5\%
\\
AT-5\%
\\
CPC-4000/173
}
&
\makecell{
US\$
\\
101.47/KG
}
&
\makecell{
US\$
\\
1.80/KG
\\
SECTION
VALUE
}
&
\makecell{
US\$
\\
101.47/KG
} \\
\hline
% row
\makecell{
09
}
&
\makecell{
UPE FIXED
\\
C/O. {\co}
}
&
\makecell{
0.60
\\
KGS
}
&
\makecell{
84519000
\\
CD-1\%
\\
SRO 113/21
\\
CPC-4000/220
}
&
\makecell{
84519000
\\
CD-1\%
\\
SRO 113/21
\\
CPC-4000/220
}
&
\makecell{
US\$
\\
78.55/KG
}
&
\makecell{
US\$
\\
1.80/KG
\\
DATA BASE
VALUE
\\
C-\hspace{4em}
\\
DT:\hspace{4em}
}
&
\makecell{
US\$
\\
78.55/KG
} \\
\hline
% row
\makecell{
10
}
&
\makecell{
MC NYLON PULLEY
\\
C/O. {\co}
}
&
\makecell{
26.50
\\
KGS
}
&
\makecell{
84519000
\\
CD-1\%
\\
SRO 113/21
\\
CPC-4000/220
}
&
\makecell{
84835000
\\
CD-1\%
\\
SRO 113/21
\\
CPC-4000/220
}
&
\makecell{
US\$
\\
63.85/KG
}
&
\makecell{
US\$
\\
1.80/KG
\\
DATA BASE
VALUE
\\
C-\hspace{4em}
\\
DT:\hspace{4em}
}
&
\makecell{
US\$
\\
63.85/KG
} \\
\hline
% row
\makecell{
11
}
&
\makecell{
BEARING WITH BLOCK
\\
AND LOCK NUT UCFC
\\
C/O. {\co}
}
&
\makecell{
18.60
\\
KGS
}
&
\makecell{
84828000
\\
CD-1\%
\\
AIT-5\%
\\
AT-5\%
\\
CPC-4000/157
}
&
\makecell{
84821000
\\
CD-1\%
\\
AIT-5\%
\\
AT-5\%
\\
CPC-4000/157
}
&
\makecell{
US\$
\\
41.75/KG
}
&
\makecell{
US\$
\\
4.00/KG
\\
MINIMUM
VALUE
}
&
\makecell{
US\$
\\
41.75/KG
} \\
\hline
\end{longtable}
}

\begin{document}
\noindent
\begin{minipage}[t]{0.05\linewidth}
% ek
০১।
\end{minipage}
\begin{minipage}[t]{0.95\linewidth}
বি/ই রেজি: নং- {\beno}, তারিখ: {\bedt}
নথিভূক্ত করে
পরবর্তী কার্যক্রমের জন্য উপস্থাপন করা হলো।
\\
\\
\end{minipage}
\begin{minipage}[t]{0.05\linewidth}
\hspace*{0em}
\end{minipage}
\begin{minipage}[t]{0.05\linewidth}
সহকারী
\end{minipage}
\begin{minipage}[t]{0.37\linewidth}
\hspace{0em}
\end{minipage}
\begin{minipage}[t]{0.53\linewidth}
\textbf{শুল্কায়ন কর্মকর্তা}
\\
\end{minipage}
\begin{minipage}[t]{0.05\linewidth}
% dui
০২।
\end{minipage}
\begin{minipage}[t]{0.95\linewidth}
\underline{\textbf {আমদানিকৃত পণ্য চালানের
মৌলিক তথ্য:}}
\\
\end{minipage}
\footnotesize
\begin{minipage}[t]{0.05\linewidth}
\hspace*{1em}
\end{minipage}
\begin{minipage}[t]{0.40\linewidth}
(ক) বি/ই রেজি: নং ও তারিখ
\end{minipage}
\begin{minipage}[t]{0.02\linewidth}
:
\end{minipage}
\begin{minipage}[t]{0.53\linewidth}
\textbf{{\beno}} \hspace{2em} DT: {\bedt}
\\
\end{minipage}
\begin{minipage}[t]{0.05\linewidth}
\hspace*{1em}
\end{minipage}
\begin{minipage}[t]{0.40\linewidth}
(খ) আমদানিকারকের নাম, ঠিকানা
ও BIN নম্বর
\end{minipage}
\begin{minipage}[t]{0.02\linewidth}
:
\end{minipage}
\begin{minipage}[t]{0.53\linewidth}
\textbf{{\impn}}
\\
{\impadd}
\\
BIN NO. {\impbin}
\\
\end{minipage}
\begin{minipage}[t]{0.05\linewidth}
\hspace*{1em}
\end{minipage}
\begin{minipage}[t]{0.40\linewidth}
(গ) সিএন্ডএফ এজেন্টের নাম, ঠিকানা
ও AIN নম্বর
\end{minipage}
\begin{minipage}[t]{0.02\linewidth}
:
\end{minipage}
\begin{minipage}[t]{0.53\linewidth}
\textbf{{\cnfn}}
\\
{\cnfadd}
\\
AIN NO. {\cnfain}
\\
\end{minipage}
\begin{minipage}[t]{0.05\linewidth}
\hspace*{1em}
\end{minipage}
\begin{minipage}[t]{0.40\linewidth}
(ঘ) এল/সি নং ও তারিখ
\end{minipage}
\begin{minipage}[t]{0.02\linewidth}
:
\end{minipage}
\begin{minipage}[t]{0.53\linewidth}
{\lcno} \hspace{2em} DT: {\lcdt}
\\
\end{minipage}
\begin{minipage}[t]{0.05\linewidth}
\hspace*{1em}
\end{minipage}
\begin{minipage}[t]{0.40\linewidth}
(ঙ) লিয়েন ব্যাংকের নাম
\end{minipage}
\begin{minipage}[t]{0.02\linewidth}
:
\end{minipage}
\begin{minipage}[t]{0.53\linewidth}
{\lienbank}
\\
\end{minipage}
\begin{minipage}[t]{0.05\linewidth}
\hspace*{1em}
\end{minipage}
\begin{minipage}[t]{0.40\linewidth}
(চ) এলসিএ নং ও তারিখ
\end{minipage}
\begin{minipage}[t]{0.02\linewidth}
:
\end{minipage}
\begin{minipage}[t]{0.53\linewidth}
{\lcano} \hspace{2em} DT: {\lcadt}
\\
\end{minipage}
\begin{minipage}[t]{0.05\linewidth}
\hspace*{1em}
\end{minipage}
\begin{minipage}[t]{0.40\linewidth}
(ছ) বি/এল নং ও তারিখ
\end{minipage}
\begin{minipage}[t]{0.02\linewidth}
:
\end{minipage}
\begin{minipage}[t]{0.53\linewidth}
{\blno} \hspace{2em} DT: {\bldt}
\\
\end{minipage}
\begin{minipage}[t]{0.05\linewidth}
\hspace*{1em}
\end{minipage}
\begin{minipage}[t]{0.40\linewidth}
(জ) বাণিজ্যিক ইনভয়েস নং ও তারিখ
\end{minipage}
\begin{minipage}[t]{0.02\linewidth}
:
\end{minipage}
\begin{minipage}[t]{0.53\linewidth}
{\invno} \hspace{2em} DT: {\invdt}
\\
\end{minipage}
\begin{minipage}[t]{0.05\linewidth}
\hspace*{1em}
\end{minipage}
\begin{minipage}[t]{0.40\linewidth}
(ঝ) সিআরএফ নং ও ইস্যুর তারিখ
\end{minipage}
\begin{minipage}[t]{0.02\linewidth}
:
\end{minipage}
\begin{minipage}[t]{0.53\linewidth}
{\crf} \hspace{2em} {\crfdt}
\\
\end{minipage}
\begin{minipage}[t]{0.05\linewidth}
\hspace*{1em}
\end{minipage}
\begin{minipage}[t]{0.40\linewidth}
(ঞ) পণ্যের বিবরণ
\end{minipage}
\begin{minipage}[t]{0.02\linewidth}
:
\end{minipage}
\begin{minipage}[t]{0.53\linewidth}
{\good}
\\
\end{minipage}
\begin{minipage}[t]{0.05\linewidth}
\hspace*{1em}
\end{minipage}
\begin{minipage}[t]{0.40\linewidth}
(ট) পণ্যের পরিমাণ (একক সহ)
\end{minipage}
\begin{minipage}[t]{0.02\linewidth}
:
\end{minipage}
\begin{minipage}[t]{0.53\linewidth}
{\pkg}
\\
\end{minipage}
\begin{minipage}[t]{0.05\linewidth}
\hspace*{1em}
\end{minipage}
\begin{minipage}[t]{0.40\linewidth}
(ঠ) পণ্যের এইচ.এস.কোড
\end{minipage}
\begin{minipage}[t]{0.02\linewidth}
:
\end{minipage}
\begin{minipage}[t]{0.53\linewidth}
{\hscode}
\\
\end{minipage}
\begin{minipage}[t]{0.05\linewidth}
\hspace*{1em}
\end{minipage}
\begin{minipage}[t]{0.40\linewidth}
(ড) পণ্যের মূল্য (ইনভয়েস অনুযায়ী)
\end{minipage}
\begin{minipage}[t]{0.02\linewidth}
:
\end{minipage}
\begin{minipage}[t]{0.53\linewidth}
{\price}
\\
\end{minipage}
\begin{minipage}[t]{0.05\linewidth}
\hspace*{1em}
\end{minipage}
\begin{minipage}[t]{0.40\linewidth}
(ঢ) কান্ট্রি অব অরিজিন
\end{minipage}
\begin{minipage}[t]{0.02\linewidth}
:
\end{minipage}
\begin{minipage}[t]{0.53\linewidth}
{\co}
\\
\end{minipage}
\begin{minipage}[t]{0.05\linewidth}
\hspace*{1em}
\end{minipage}
\begin{minipage}[t]{0.40\linewidth}
(ণ) কান্ট্রি অব শিপমেন্ট
\end{minipage}
\begin{minipage}[t]{0.02\linewidth}
:
\end{minipage}
\begin{minipage}[t]{0.53\linewidth}
{\coship}
\\
\end{minipage}
\begin{minipage}[t]{0.05\linewidth}
\hspace*{1em}
\end{minipage}
\begin{minipage}[t]{0.40\linewidth}
(ত) জাহাজের নাম
\end{minipage}
\begin{minipage}[t]{0.02\linewidth}
:
\end{minipage}
\begin{minipage}[t]{0.53\linewidth}
{\vessel}
\end{minipage}
\begin{minipage}[t]{0.05\linewidth}
\hspace*{1em}
\end{minipage}
\begin{minipage}[t]{0.40\linewidth}
\hspace*{1.8em}পালা নং বি/এল নং
\end{minipage}
\begin{minipage}[t]{0.02\linewidth}
\hspace{1em}
\end{minipage}
\begin{minipage}[t]{0.53\linewidth}
{\menifest}, B/L {\blno}
\\
\end{minipage}
\begin{minipage}[t]{0.05\linewidth}
\hspace*{1em}
\end{minipage}
\begin{minipage}[t]{0.40\linewidth}
(থ) মেনিফিস্ট নং
\end{minipage}
\begin{minipage}[t]{0.02\linewidth}
:
\end{minipage}
\begin{minipage}[t]{0.53\linewidth}
{\menifest}
\\
\end{minipage}
\normalsize
\begin{minipage}[t]{0.05\linewidth}
% tin
০৩।
\end{minipage}
\begin{minipage}[t]{0.95\linewidth}
\underline{\textbf{শুল্কায়ন সেকশনের পর্যালোচনা:}}
\end{minipage}
\begin{minipage}[t]{0.05\linewidth}
\hspace{1em}
\end{minipage}
\begin{minipage}[t]{0.05\linewidth}
% tina
(ক)
\end{minipage}
\begin{minipage}[t]{0.90\linewidth}
\underline{\textbf{আমদানি দলিল পত্র যাচাই:}}
পণ্যচালান খালাসের জন্য নিম্নবর্ণিত দলিলাদিসহ বি/ই দাখিল করা
হয়েছে যা সংশ্লিষ্ট ব্যাংকের অথরাইজড কর্মকর্তা কর্তৃক সত্যায়িত।
\\
(১) এল.সি এবং এল.সি.এ ফরম।
\\
(২) ইনভয়েস।
\\
(৩) প্যাকিং লিস্ট।
\\
(৪) বি/এল।
\\
(৫) কান্ট্রি অব অরিজিন সনদ।
\\
দলিলাদি পর্যালোচনায় এগুলো
সঠিক পাওয়া যায়।
\\
\end{minipage}
\begin{minipage}[t]{0.05\linewidth}
\hspace{1em}
\end{minipage}
\begin{minipage}[t]{0.05\linewidth}
% tinb
(খ)
\end{minipage}
\begin{minipage}[t]{0.90\linewidth}
\underline{\textbf{আমদানি যোগ্যতা যাচাই:}}
প্রচলিত আমদানিনীতি আদেশ পর্যালোচনা করে দেখা যায় যে, পণ্যগুলি অবাধে আমদানিযোগ্য।
আলোচ্য চালানের ক্ষেত্রে আমদানিনীতি আদেশের প্রযোজ্য অন্যান্য শর্ত (কান্ট্রি অব অরিজিন, রেজিঃ
সার্টিফিকেট ইত্যাদি) প্রতিপালিত হয়েছে।
\\
\end{minipage}
\begin{minipage}[t]{0.05\linewidth}
\hspace{1em}
\end{minipage}
\begin{minipage}[t]{0.05\linewidth}
% tinc
(গ)
\end{minipage}
\begin{minipage}[t]{0.90\linewidth}
\underline{\textbf{কায়িক পরীক্ষার প্রতিবেদন পর্যালোচনা:}}
আলোচ্য পণ্যচালানটির ক্ষেত্রে প্রযোজ্য নয়।
\\
\end{minipage}
\begin{minipage}[t]{0.05\linewidth}
\hspace{1em}
\end{minipage}
\begin{minipage}[t]{0.05\linewidth}
% tind
(ঘ)
\end{minipage}
\begin{minipage}[t]{0.90\linewidth}
\underline{\textbf{রাসায়নিক পরীক্ষা সংক্রান্ত মন্তব্য:}}
আলোচ্য পণ্যচালানটির ক্ষেত্রে প্রযোজ্য নয়।
\\
\end{minipage}
\begin{minipage}[t]{0.05\linewidth}
\hspace{1em}
\end{minipage}
\begin{minipage}[t]{0.05\linewidth}
% tine
(ঙ)
\end{minipage}
\begin{minipage}[t]{0.90\linewidth}
\underline{\textbf{এইচ.এস.কোড সঠিকতা যাচাই:}}
The Customs Act, 1969 এর FIRST SCHEDULE ও
EXPLANATORY NOTES প্রচলিত এসআরও/স্থায়ী আদেশ ইত্যাদির আলোকে পরীক্ষা করে পণ্যের সঠিক HS Code নোট অনুচ্ছেদ-৪
এর ছকে উপস্থাপন করা হলো।
\\
\end{minipage}
\begin{minipage}[t]{0.05\linewidth}
\hspace{1em}
\end{minipage}
\begin{minipage}[t]{0.05\linewidth}
% ting
(চ)
\end{minipage}
\begin{minipage}[t]{0.90\linewidth}
\underline{\textbf{অভিযোগ সংক্রান্ত:}} আলোচ্য পণ্যচালানে
গোপন সংবাদ দাতা, শুল্ক গোয়েন্দা বা
জাতীয় রাজস্ব বোর্ডের কিংবা অন্য দপ্তর থেকে
কোন অভিযোগ পাওয়া যায় নাই।
\\
\end{minipage}
\begin{minipage}[t]{0.05\linewidth}
\hspace{1em}
\end{minipage}
\begin{minipage}[t]{0.05\linewidth}
% tinh
(ছ)
\end{minipage}
\begin{minipage}[t]{0.90\linewidth}
\underline{\textbf{ন্যায় নির্ণয় সংক্রান্ত:}} প্রযোজ্য নয়।
\\
\end{minipage}
\begin{minipage}[t]{0.05\linewidth}
\hspace{1em}
\end{minipage}
\begin{minipage}[t]{0.05\linewidth}
% tinh
(জ)
\end{minipage}
\begin{minipage}[t]{0.05\linewidth}
\end{minipage}
\begin{minipage}[t]{0.90\linewidth}
\underline{\textbf{উল্লেখ করার মত প্রাসঙ্গিক অন্যান্য বিষয়:}}
\end{minipage}
\footnotesize
\begin{minipage}[t]{0.05\linewidth}
\hspace{1em}
\end{minipage}
\begin{minipage}[t]{0.05\linewidth}
\hspace{1em}
\end{minipage}
\begin{minipage}[t]{0.05\linewidth}
(১)
\end{minipage}
\begin{minipage}[t]{0.85\linewidth}
আলোচ্য পণ্যচালানটির ঘোষিত HS Code {\hscode}
যা {\srooot}, {\sroootd} এর শর্ত মোতাবেক
আমদানি করা হয়েছে।
\end{minipage}
\begin{minipage}[t]{0.05\linewidth}
\hspace{1em}
\end{minipage}
\begin{minipage}[t]{0.05\linewidth}
\hspace{1em}
\end{minipage}
\begin{minipage}[t]{0.05\linewidth}
(২)
\end{minipage}
\begin{minipage}[t]{0.85\linewidth}
আলোচ্য পণ্যচালানের বিপরীতে বাংলাদেশ
টেক্সটাইল মিলস এসোসিয়েশন এর সভাপতি
কর্তৃক স্বাক্ষরিত প্রত্যয়নপত্র
দাখিল করেছেন। প্রত্যয়নপত্রে আমদানিকৃত
যন্ত্রপাতি, যন্ত্রাংশ, ও উপকরণ
উক্ত প্রতিষ্ঠানের উৎপাদন প্রক্রিয়ায়
ব্যবহৃত হইবে এবং আমদানিকৃত
যন্ত্রপাতি, যন্ত্রাংশ, ও উপকরণ
{\srooot}, {\sroootd} অনুযায়ী
রেয়াতীহারে শুল্কায়নের সুপারিশ
করেন। বাংলাদেশ টেক্সটাইল মিলস
এসোসিয়েশনের দাখিলকৃত {\btmaltno}
{\btmaltnodt} ইং নথির যোগাযোগ অংশে রক্ষিত
আছে, দয়া করে দেখা যেতে পারে।
\end{minipage}
\begin{minipage}[t]{0.05\linewidth}
\hspace{1em}
\end{minipage}
\begin{minipage}[t]{0.05\linewidth}
\hspace{1em}
\end{minipage}
\begin{minipage}[t]{0.05\linewidth}
(৩)
\end{minipage}
\begin{minipage}[t]{0.85\linewidth}
এসআরও শর্ত মোতাবেক আমদানিকারক ৩০০.০০ (তিনশত)
টাকার নন-জুডিশিয়াল স্ট্যাম্পে একখানা অঙ্গীকারনামা
দাখিল করেছেন। দাখিলকৃত অঙ্গীকারনামায় আমদানিকৃত
যন্ত্রপাতি, যন্ত্রাংশ, ও উপকরণ
উক্ত প্রতিষ্ঠানের উৎপাদন প্রক্রিয়ায়
ব্যবহার করবেন। অন্যথায় আমদানিকৃত উক্ত পণ্যের উপর
প্রযোজ্য স্বাভাবিক হারের শুল্ককর পরিশোধ ছাড়াও
শুল্ক আইন মোতাবেক শুল্ক কর্তৃপক্ষ
গৃহীত যে কোন আইনানুগ সিদ্ধান্ত মানিয়া নিতে বাধ্য থাকিবেন
মর্মে উল্লেখ করেছেন। আমদানিকারক কর্তৃক দাখিলকৃত
অঙ্গীকারনামা নথির যোগাযোগ অংশে রক্ষিত আছে,
দয়া করে দেখা যেতে পারে।
\\
\end{minipage}
\normalsize
\begin{minipage}[t]{0.05\linewidth}
% char
০৪।
\end{minipage}
\begin{minipage}[t]{0.95\linewidth}
\underline{\textbf{শুল্কায়ন সম্পর্কিত প্রস্তাব:}}
পণ্য চালান সংক্রান্ত আমদানি দলিলপত্র
ইনভয়েস পর্যালোচনা পূর্বক পণ্যের বর্ণনা, পরিমাণ, এইচ.এস.কোড, ঘোষিত মূল্য,
সমসাময়িককালের অভিন্ন/অনুরূপ পণ্যের একক মূল্যসহ নিম্নের ছকে শুল্কায়নের প্রস্তাব উপস্থাপন
করা হলো:
\end{minipage}
\scriptsize
{\taxtab}
\vspace*{1mm}
\normalsize
\noindent
\begin{minipage}[t]{0.05\linewidth}
% sat
০৫।
\end{minipage}
\begin{minipage}[t]{0.95\linewidth}
আলোচ্য চালানের আইটেম নং- ১, ৩, ৯, ১০
এর ক্ষেত্রে
{\srooot}, {\sroootd}
এর টেবিলভূক্ত হওয়ায়
{\cpcttz}
ঘোষণা প্রদান করা হয়েছে।
SRO শর্ত মোতাবেক নিম্নোক্ত দলিলাদি
দাখিল করেছেন।
\\
(ক) নবায়নকৃত IRC।
\\
(খ) ১৩ ডিজিট সম্বলিত মূসক-২.৩ সনদপত্র।
\\
(গ) {\musokr} মাস পর্যন্ত অনলাইন রিটার্ণ।
\\
\end{minipage}
\begin{minipage}[t]{0.05\linewidth}
% sat
০৬।
\end{minipage}
\begin{minipage}[t]{0.95\linewidth}
\underline{\textbf{পর্যালোচনা:}}
আইটেম নং-  ২, ৫, ৭, ১০ ও ১১ এর ক্ষেত্রে
অসত্য H.S Code ঘোষণা প্রদান করা হয়েছে যা
{\tca}
-এর Section-32 এর লঙ্ঘন, যা একই আইনের
Section-156 -এর টেবিল ১ এর ১৪ নং মোতাবেক ন্যায় নির্ণয়ন যোগ্য।
\\
\end{minipage}
\begin{minipage}[t]{0.05\linewidth}
% sat
০৭।
\end{minipage}
\begin{minipage}[t]{0.95\linewidth}
আমদানিকারকের পক্ষে সংশ্লিষ্ট সি.এন্ড.এফ এজেন্ট
জয়েন্ট কমিশনার বরাবর একটি আবেদন দাখিল করেছেন।
আবেদনে জানিয়েছেন যে,
আলোচ্য পণ্যচালানের
আমদানিকারক একটি ১০০\% রপ্তানিমুখী শিল্প প্রতিষ্ঠান এবং
BTMA - এর সদস্য। আলোচ্য চালানের বি/ই আইটেম নং-২
CHAIN LINK, HS Code 73158900 এবং {\cpcost}
দিয়ে বি/ই দাখিল করেন। আমদানিকারক {\srootz}, {\srootzd}
এর শর্ত সমূহ পালন করায় SRO সুবিধাতে পণ্যচালানটি
সাময়িক শুল্কায়নে ইচ্ছুক।
\\
\end{minipage}
\begin{minipage}[t]{0.05\linewidth}
% sat
০৮।
\end{minipage}
\begin{minipage}[t]{0.95\linewidth}
\underline{\textbf{শুল্কায়ন শাখার মতামত:}}
আলোচ্য চালানের আইটেম নং-
২, ৫, ৭, ১০ ও ১১
এর ক্ষেত্রে অসত্য H.S Code ঘোষণা প্রদান করা হয়েছে।
আমদানিকারকের পক্ষে সি.এন্ড.এফ এজেন্ট আইটেম নং
২ ও ৭ এর প্রস্তাবিত H.S Code
{\srootz}, {\srootzd}
এর টেবিলভূক্ত হওয়ায় SRO -এর শর্ত মোতাবেক দলিলাদি
দাখিল করেছেন। একই সাথে উক্ত দুটি আইটেম
{\cpcofs} -তে শুল্কায়নের আবেদন করেছেন।
SRO শর্ত মোতাবেক প্রয়োজনীয় BTMA এর
{\btmaltno}, {\btmaltnodt}
ও ৩০০(তিনশত) টাকার নন জুডিশিয়াল স্ট্যাম্পে
অঙ্গীকারনামা দাখিল করেছেন। SRO শর্ত পূরণ হওয়ায়
আইটেম নং ২ ও ৭
{\cpcofs}
শুল্কায়ন করা যেতে পারে।
উক্ত আইটেমের ক্ষেত্রে
{\cpcofs} -তে শুল্কায়ন করা হলে
--- টাকা রাজস্ব হ্রাস পাবে।
তবে আমদানিকারক কর্তৃক অসত্য
H.S Code ঘোষণা প্রদান করায় {\tca} -এর Section-32
লঙ্ঘন হয়েছে যা একই আইনের Section-156(1) এর টেবিল-১
মোতাবেক ন্যায় নির্ণয়ন যোগ্য।
আমদানিকারককে Section-180 মোতাবেক কারন দর্শাও
নোটিশ জারী করা যেতে পারে।
তবে আমদানিকারক কারন দর্শাও
ও নোটিশ জারী ব্যতিরেকে সংক্ষিপ্ত বিচারাদেশের
আবেদন দাখিল করেছেন। আবেদন বিবেচনা করে ন্যায়
নির্ণয়ন সাপেক্ষে নোট অনুচ্ছেদ ৪ এর প্রস্তাবিত
H.S Code, CPC ও Value -তে শুল্কায়ন করা যেতে পারে।
\\
\end{minipage}
\begin{minipage}[t]{0.05\linewidth}
% at
০৯।
\end{minipage}
\begin{minipage}[t]{0.95\linewidth}
\underline{\textbf{প্রস্তাব:}}
\end{minipage}
\begin{minipage}[t]{0.05\linewidth}
\hspace{1em}
\end{minipage}
\begin{minipage}[t]{0.05\linewidth}
(ক)
\end{minipage}
\begin{minipage}[t]{0.90\linewidth}
আলোচ্য চালানে ঘোষণা মোতাবেক শুল্ক করাদির পরিমান
--- টাকা
\end{minipage}
\begin{minipage}[t]{0.05\linewidth}
\hspace{1em}
\end{minipage}
\begin{minipage}[t]{0.05\linewidth}
(খ)
\end{minipage}
\begin{minipage}[t]{0.90\linewidth}
সঠিক H.S Code ও আমদানিকারকের আবেদন
মোতাবেক আইটেম নং- ২ ও ৭ {\cpcofs} -তে
শুল্কায়ন করলে নির্ণীত শুল্ক করাদির পরিমান
--- টাকা (ডামি সদয় দ্রষ্টব্য।)।
\end{minipage}
\begin{minipage}[t]{0.05\linewidth}
\hspace{1em}
\end{minipage}
\begin{minipage}[t]{0.05\linewidth}
(গ)
\end{minipage}
\begin{minipage}[t]{0.90\linewidth}
আইটেম নং-  ২, ৫, ৭, ১০, ও ১১ এর
ক্ষেত্রে অসত্য H.S Code ঘোষণা প্রদান করায়
আমদানিকারকের সংক্ষিপ্ত বিচারাদেশের আবেদন
বিবেচনায় নিয়ে {\tca} -এর Section-156(1) এর
টেবিল-১ মোতাবেক ন্যায় নির্ণয়ন সাপেক্ষে শুল্কায়ন করা
যেতে পারে।
\\
\\
\\
\normalsize
সদয় অবগতি ও আদেশার্থে উপস্থাপন করা
হলো।
\end{minipage}

\newpage
\begin{center}
\vspace*{6MM}
গণপ্রজাতন্ত্রী বাংলাদেশ সরকার
\\
\footnotesize{কাস্টম হাউস, চট্টগ্রাম।}
\\
\href{}{ওয়েব: www.chc.gov.bd}\hspace{1em}
\href{}{ইমেইল: customhousectg@gmail.com}
\end{center}
\normalsize
\begin{minipage}[t]{.74\linewidth}
{\fileno}
\end{minipage}
\begin{minipage}[t]{.26\linewidth}
{\fdt}
\\
\end{minipage}
\begin{minipage}[t]{.07\linewidth}
প্রাপক:
\end{minipage}
\begin{minipage}[t]{.93\linewidth}
কমিশনার
\\
শুল্ক মূল্যায়ন ও অভ্যন্তরীণ নিরীক্ষা কমিশনারেট
\\
গুলফেঁশা প্লাজা (৬ষ্ঠ তলা)
\\
৬৯, আউটার সার্কুলার রোড
\\
মগবাজার, ঢাকা-১২১৭।
\\
\end{minipage}
\begin{minipage}[t]{.07\linewidth}
বিষয়:
\end{minipage}
\begin{minipage}[t]{.93\linewidth}
\textbf{আমদানিকৃত পণ্য {\good} এর যথাযথ ব্যবহার নিশ্চিতকরণ।}
\end{minipage}
\begin{minipage}[t]{.07\linewidth}
সূত্র:
\end{minipage}
\begin{minipage}[t]{.93\linewidth}
{\srooot}, {\sroootd}।
\\
উপর্যুক্ত বিষয় ও সূত্রের প্রতি আপনার সদয় দৃষ্টি
আকর্ষন করা হলো।
\\
\end{minipage}
\begin{minipage}[t]{.07\linewidth}
২।
\end{minipage}
\begin{minipage}[t]{.93\linewidth}
আমদানিকারক প্রতিষ্ঠান- {\impn}, {\impadd}
কর্তৃক এলসি নং- {\lcno}, Date: {\lcdt}
এর মাধ্যমে {\co} হতে
{\pkg} সূত্রোক্ত জাহাজ যোগে {\good}
পণ্যচালানটি আমদানি করে পণ্য
খালাসের নিমিত্তে বি/ই নং- {\beno}, Date: {\bedt}
দাখিল করেন। আমদানিকারক একটি
শিল্প প্রতিষ্ঠান বিধায় পণ্যচালানটি
{\srooot}, {\sroootd}
এর শর্ত মোতাবেক আমদানি করেছেন এবং
তার প্রতিষ্ঠানটি বাংলাদেশ টেক্সটাইল মিলস
এসোসিয়েশন এর একটি সদস্য মর্মে
{\btmaltno}, {\btmaltnodt}
দাখিল করেছেন।
\\
\end{minipage}
\begin{minipage}[t]{.07\linewidth}
৩।
\end{minipage}
\begin{minipage}[t]{.93\linewidth}
উক্ত এসআরও এর শর্ত মোতাবেক
আলোচ্য আমদানিকারকের আমদানিকৃত পণ্য
শিল্প কারখানায় যথাযথভাবে ব্যবহার করা হয়েছে
কিনা তা আপনার দপ্তর কর্তৃক
যথাযথ স্বল্প সময়ে
নিরীক্ষা সমাপ্তির ৬(ছয়) মাসের মধ্যে এসআরও
এর পরিশিষ্ট-৩ মোতাবেক একটি বস্তুনিষ্ট
প্রতিবেদন জাতীয় রাজস্ব বোর্ডের সদস্য
(কাস্টমস নীতি) এর নিকট এবং অনুলিপি সংশ্লিষ্ট
কাস্টমস স্টেশন কর্তৃপক্ষ বরাবরে প্রেরণ করার
নির্দেশনা রয়েছে। উক্ত নির্দেশনা
অনুযায়ী প্রয়োজনীয় কর্যক্রম গ্রহণের জন্য
আপনাকে সবিনয় অনুরোধ করা হলো।
\\
\\
\\
\end{minipage}
\begin{minipage}[t]{0.60\linewidth}
\hspace{1em}
\end{minipage}
\begin{minipage}[t]{0.40\linewidth}
\begin{center}
জাকির হোসেন
\\
ডেপুটি কমিশনার অব কাস্টমস
\\
কমিশনার অব কাস্টমস এর পক্ষে
\\
কাস্টম হাউস, চট্টগ্রাম।
\\
\footnotesize{{\rodt}}
\vspace*{10MM}
\end{center}
\end{minipage}
\footnotesize{{\fileno}}
\\
\underline{\footnotesize{অনুলিপি অবগতি ও প্রয়োজনীয় ব্যবস্থা গ্রহনের জন্য প্রেরণ করা হলো}}
\\
\begin{minipage}[t]{0.06\linewidth}
\footnotesize{০১।}
\end{minipage}
\begin{minipage}[t]{0.94\linewidth}
বিভাগীয় কর্মকর্তা, কাস্টমস এক্সাইজ ও ভ্যাট বিভাগ
\end{minipage}
\begin{minipage}[t]{0.06\linewidth}
\footnotesize{০২।}
\end{minipage}
\begin{minipage}[t]{0.94\linewidth}
\footnotesize{
আমদানিকারক {\impn}, {\impadd}
}
\end{minipage}
\begin{minipage}[t]{0.06\linewidth}
\footnotesize{০৩।}
\end{minipage}
\begin{minipage}[t]{0.94\linewidth}
সভাপতি/সাধারণ সম্পাদক, বাংলাদেশ টেক্সটাইল
মিলস এসোসিয়েশন, ইউনিক ট্রেড সেন্টার (লেভেল-৮),
৮, পান্থপথ, কারওয়ান বাজার, ঢাকা-১২১৫।
\end{minipage}
\begin{minipage}[t]{0.06\linewidth}
\footnotesize{০৪।}
\end{minipage}
\begin{minipage}[t]{0.94\linewidth}
 অফিস কপি।
 \\
 \\
 \\
 \\
 \\
\end{minipage}
\begin{minipage}[t]{0.60\linewidth}
\hspace{1em}
\end{minipage}
\normalsize
\begin{minipage}[t]{0.40\linewidth}
\begin{center}
জাকির হোসেন
\\
ডেপুটি কমিশনার অব কাস্টমস
\\
কাস্টম হাউস, চট্টগ্রাম।
\end{center}
\end{minipage}
\thispagestyle{slogan}

\end{document}
