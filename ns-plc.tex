\documentclass[12pt]{article}
\usepackage[legalpaper,
            lmargin=1in,rmargin=0.5in,
            bmargin=0.5in,tmargin=1in]{geometry}
\usepackage{fontspec}
\usepackage{titlesec}
\usepackage{multirow}
\usepackage[colorlinks=true,urlcolor=black]{hyperref}
\usepackage{graphicx}
\usepackage{array}
\usepackage{makecell}
\usepackage{fancyhdr}
\usepackage[none]{hyphenat}
\usepackage{longtable}
\usepackage[dvipsnames]{xcolor}
\usepackage[banglamainfont=Kalpurush,
            banglattfont=SolaimanLipi,
            % feature=1,
            % changecounternumbering=0
           ]{latexbangla}

% oncehsis imp file name
\newcommand{\eif}{নথি নং- এস \hspace{4em}/স্টাফ/কাস/\hspace{5em}/অংশ-}
\newcommand{\jdlf}{নথি নং- এস৫-২৪৫/স্টাফ/কাস/০৬-০৭/অংশ-২}
\newcommand{\jdwlf}{নথি নং- এস৫-১৭০/স্টাফ/কাস/১৪-১৫/অংশ-২}
\newcommand{\jealf}{নথি নং- এস-৫-৫৮৫/স্টাফ/কাস/২০১৩-২০১৪/অংশ-১২}
\newcommand{\jfclf}{নথি নং- এস৫-৫১৯/স্টাফ/কাস/২০-২১}
\newcommand{\jsmlf}{নথি নং- এস৫-৬৯১/স্টাফ/কাস/২০১৭-২০১৮/অংশ-০৩}
\newcommand{\jtrif}{নথি নং- এস-৫-১২১০/স্টাফ/কাস/১৬-১৭/অংশ-১}
\newcommand{\jwelf}{নথি নং- এস-৫-১১৪/স্টাফ/কাস/২০১৮-১৯/}
\newcommand{\scmlf}{নথি নং- এস-৫-১৬৩/স্টাফ/কাস/০৬-০৭/অংশ-৫}
\newcommand{\ssmlf}{নথি নং- এস-৫-১৪৪/স্টাফ/কাস/০৬-০৭/অংশ-১১}
\newcommand{\tdjf}{নথি নং- এস-৫-৪৪৩/স্টাফ/কাস/০৬-০৭/অংশ-}
\newcommand{\htff}{নথি নং- এস-৫-৬৮৪/স্টাফ/কাস/২০১৭-১৮/অংশ-৩}

% fileno
\newcommand{\filenou}{
\begin{minipage}[t]{0.60\linewidth}
\hspace{1em}
\end{minipage}
\begin{minipage}[t]{0.40\linewidth}
\underline{\fileno}
\end{minipage}
}


% new page
\newcommand{\nfpage}{
\newpage
\small
{\filenou}
}

% imp name
\newcommand{\ssml}{SHAMEEM SPINNING MILLS LTD.}
\newcommand{\ssmla}{SHAFIPUR, KALIAKAIR, GAZIPUR-1751, BANGLADESH.}
\newcommand{\scml}{SHAMEEM COMPOSITE MILLS LTD.}
\newcommand{\scmla}{SHAFIPUR, KALIAKAIR, GAZIPUR-1751, BANGLADESH}
\newcommand{\jsml}{JAMUNA SPINNING MILLS LTD.}
\newcommand{\jsmla}{SHAFIPUR, KALIAKAIR, GAZIPUR-1751, BANGLADESH.}
\newcommand{\jsmlaut}{JAMUNA SPINNING MILLS LTD. UNIT-2}
\newcommand{\jsmlauta}{BEJURA, SOUTH BEJURA, MADHABPUR, HOBIGONJ-3331, BANGLADESH.}
\newcommand{\jdl}{JAMUNA DENIMS LTD.}
\newcommand{\jdla}{SHAFIPUR, KALIAKAIR, GAZIPUR-1751, BANGLADESH.}
\newcommand{\jeal}{JAMUNA ELECTRONICS \& AUTOMOBILES LTD.}
\newcommand{\jeala}{SINABHA, KALIAKAIR PS, GAZIPUR-1750, BANGLADESH}
\newcommand{\jdwl}{JAMUNA DENIMS WEAVING LTD.}
\newcommand{\jdwla}{KASHIMPUR ROAD, JARUN, KONABARI PS, GAZIPUR-1751, BD}
\newcommand{\jfcl}{JAMUNA FAN AND CABLES LTD.}
\newcommand{\jfcla}{KASHIMPUR ROAD, JARUN, KONABARI PS, GAZIPUR-1751, BD}
\newcommand{\htf}{HOORAIN HTF LTD.}
\newcommand{\htfa}{BEJURA, SOUTH BEJURA, MADHABPUR, HOBIGONJ-3331, BANGLADESH.}
\newcommand{\tdj}{M/S THE DAILY JUGANTOR}
\newcommand{\tdja}{KA-244, KURIL, PROGATI SARANI, VATARA PS, DHAKA-1229, BANGLADESH}
\newcommand{\jtri}{JAMUNA TYRE AND RUBBER INDUSTRIES}
\newcommand{\jtria}{BEJURA, SOUTH BEJURA, MADHABPUR, HOBIGONJ}
\newcommand{\jwel}{JAMUNA WELDING ELECTRODE LTD.}
\newcommand{\jwela}{CHOYDANA, HAZIRPUKUR; GAZIPUR SADAR, GAZIPUR-1704; BANGLADESH}
\newcommand{\cbl}{CROWN BEVERAGE LIMITED}
\newcommand{\cbla}{SHAFIPUR, KALIAKAIR, GAZIPUR-1751, BANGLADESH}
\newcommand{\jpml}{JAMUNA PAPER MILLS LIMITED}
\newcommand{\jpmla}{BEJURA, SOUTH BEJURA MADHABPUR, HOBIGONJ, BANGLADESH}

% imp bin
\newcommand{\jealbin}{000146478-0103}
\newcommand{\tdjbin}{001960365-0101}
\newcommand{\jsmlbin}{000144425-0103}
\newcommand{\jfclbin}{001753334-0103}
\newcommand{\scmlbin}{000151754-0103}
\newcommand{\htfbin}{000146478-0103}
\newcommand{\cblbin}{000149506-0103}

% cnf name
\newcommand{\cnfn}{SHAMEEM SPINNING MILLS LTD.}
\newcommand{\cnfadd}{92, HIGH LEVEL ROAD, LALKHAN BAZAR, CHITTAGONG}
\newcommand{\cnfain}{301 08 3417}

% sros
\newcommand{\srofs}{এসআরও নং- ৫৭- আইন/২০০০/১৮২১/শুল্ক}
\newcommand{\srofsd}{তারিখ: ২৩/০২/২০০০ খ্রি:}
\newcommand{\srooot}{এসআরও নং- ১১৩- আইন/২০২১/০২/কাস্টমস}
\newcommand{\sroootd}{তারিখ: ২৪/০৫/২০২১ খ্রি:}
\newcommand{\srooof}{এসআরও নং- ১১৪- আইন/২০২১/০৩/কাস্টমস}
\newcommand{\srooofd}{তারিখ: ২৪/০৫/২০২১ খ্রি:}
\newcommand{\srootz}{এসআরও নং- ১২০- আইন/২০২১/০৯/কাস্টমস}
\newcommand{\srootzd}{তারিখ: ২৪/০৫/২০২১ খ্রিঃ}

\newcommand{\sroott}{এসআরও নং- ১৩৩- আইন/২০২১/২২/কাস্টমস}
\newcommand{\sroottd}{তারিখ: ২৪/০৫/২০২১ খ্রিঃ}

\newcommand{\srotsz}{এসআরও নং- ৩৬০- আইন/২০১৩/২৪৬৮/কাস্টমস}
\newcommand{\srotszd}{তারিখ: ২৫/১১/২০১৩ খ্রি:}

% cpcs
\newcommand{\cpcofs}{CPC- 4000/157}
\newcommand{\cpcost}{CPC- 4000/173}
\newcommand{\cpcttz}{CPC- 4000/220}
\newcommand{\cpcfzo}{CPC- 4000/401}
\newcommand{\cpcfzn}{CPC- 4000/409}
\newcommand{\cpcsso}{CPC- 4000/661}

% nbrl
\newcommand{\nbrosn}{জাতীয় রাজস্ব বোর্ডের পত্র নং- ০৮.০১.০০০০.০৬৮.১৮.০০৪.১৭/১৬৯}
\newcommand{\nbrosnd}{তারিখ: ২৭/০৬/২০২১ ইং}
\newcommand{\nbrosnt}{জাতীয় রাজস্ব বোর্ডের পত্র নং- ০৮.০১.০০০০.০৬৮.১৮.০০৪.১৭/১৬৯(৩)}
\newcommand{\nbrosntd}{তারিখ: ২৭/০৬/২০২১ খ্রি:}
\newcommand{\nbrfs}{জাতীয় রাজস্ব বোর্ডের পত্র নং- ০৮.০১.০০০০.০৩৪.০২.৩০১.১৯-৪৭}
\newcommand{\nbrfsd}{তারিখ: ১২/০৭/২০২১ খ্রি:}

% imp reg name
\newcommand{\ssmlreg}{২৩১৩৯৫০৬১৮০-এইচ, তারিখ: ০৯/০১/২০১৩}
\newcommand{\scmlreg}{৯৭০৪০১৭-এইচ তারিখ: ২১/০৪/১৯৯৭}
\newcommand{\jealreg}{এল-২৯৩০১৩০১৩৪৩১-এইচ, তারিখ: ০৯/০১/২০১৩}
\newcommand{\eireg}{\hspace{10em} তারিখ: \hspace{5em}}

% imp irc no
\newcommand{\jealirc}{260326120426720}
\newcommand{\scmlirc}{260326120041719}
\newcommand{\jfclirc}{260326120515020}

% san name
\newcommand{\maersk}{MAERSK BANGLADESH LIMITED}
\newcommand{\maerska}{58, AGRABAD COMMERCIAL AREA (3RD FLOOR), CHITTAGONG, 4100.}
\newcommand{\apl}{APL (BANGLADESH) PVT.LTD}
\newcommand{\apla}{PLOT NO. 30, 3RD FLOOR OF SURAIYA MANSION, AGRABAD, CHITTAGONG.}
\newcommand{\baridhi}{BARIDHI SHIPPING LINES LTD}
\newcommand{\baridhia}{3/F HRC BHABAN, 64-66 AGRABAD C/A, CHITTAGONG.}
\newcommand{\continentalbd}{CONTINENTAL TRADERS (BD) LIMITED}
\newcommand{\continentalbda}{73, AGRABAD C/A, CHITTAGONG.}
\newcommand{\continentaltr}{CONTINENTAL TRADERS (BD) LIMITED}
\newcommand{\continentaltra}{IQBAL BHABAN, AGRABAD C/A, CHITTAGONG.}
\newcommand{\gbx}{GBX LOGISTICS LTD}
\newcommand{\gbxa}{AYUB TRADE CENTER(1ST FLOOR), 1269/B, SK MUJIB ROAD, AGRABAD C/A, CHITTAGONG.}
\newcommand{\transmarine}{TRANSMARINE LOGISTICS LTD}
\newcommand{\transmarinea}{B.M.HEIGHTS(4TH FLOOR), 318, SK, MUJIB ROAD, AGRABAD C/A, CHITTAGONG.}
\newcommand{\trident}{TRIDENT SHIPPING LINE LTD}
\newcommand{\tridenta}{AKHTARUZZAMAN CENTER, 6TH FLOOR 21/22 AGRABAD, CHITTAGONG.}
\newcommand{\msc}{MSC MEDITERRANEAN SHIPP.CO.BD.LTD}
\newcommand{\msca}{IIUC TOWER, 4TH FLOOR, 1700/A SK.MUJIB ROAD, PLOT-09, AGRABAD, CHITTAGONG.}
\newcommand{\alviline}{ALVILINE BANGLADESH LIMITED}
\newcommand{\alvilinea}{78, AGRABAD C/A, MACCA MADINA TRADE CENTER, 9TH FLOOR, CHITTAGONG}
\newcommand{\ocean}{OCEAN NETWORK EXPRESS (BD) LTD}
\newcommand{\oceana}{IIUC TOWER (10TH FLOOR), 1700/A, PLOT-9, SK.MUJIB ROAD, AGRABAD C/A, CHITTAGONG}
\newcommand{\vega}{VEGA MARINE PVT LIMITED}
\newcommand{\vegaa}{DAAR-E SHAHIDI, 4TH FLOOR, 69 AGRABAD C/A, CTG}
\newcommand{\mega}{MEGATREND SHIPPING LINES LTD.}
\newcommand{\megaa}{MAKKAH MADINAH TRADE CENTER (16TH FLOOR), 78, AGRABAD, CHITTAGONG}
\newcommand{\famfa}{FAMFA SOLUTION LIMITED}
\newcommand{\famfaa}{BONANI, AGRABAD, CHITTAGONG}
\newcommand{\reliance}{RELIANCE SHIPPING SERVICES}
\newcommand{\reliancea}{34 AGRABAD C/A, CHITTAGONG 4100, BANGLADESH}
%\newcommand{\alvilinea}{}


% mujib logo
\newcommand{\my}{\includegraphics[height=3.2em]{pic/my.png}}

% slogan
\fancypagestyle{slogan}
{
\fancyhf{}
\renewcommand{\headrulewidth}{0pt}
% header
\lhead{
\framebox[1.1\width]{\footnotesize{``উন্নয়নের অক্সিজেন রাজস্ব''}}
}
\rhead{
\my
\\
\framebox[1.1\width]{\footnotesize{``জনকল্যানে রাজস্ব''}}
}
}

% customs
\newcommand{\tca}{The Customs Act, 1969}





%\pagestyle{fancy}
%\fancyhf{}
%\renewcommand{\headrulewidth}{0pt}
%% header
%\chead{
%\underline{{পৃষ্ঠা - \thepage}}
%\\
%}
%\rhead{{\filenou}}
% footer
%\rfoot{চলমান পৃষ্ঠা-\thepage}

%\fancypagestyle{laststyle}
%{
%   \fancyfoot[R]{}
%}

%\fancypagestyle{slogan}
%{
%\fancyhf{}
%\renewcommand{\headrulewidth}{0pt}
%% header
%\lhead{
%\framebox[1.1\width]{\footnotesize{``উন্নয়নের অক্সিজেন রাজস্ব''}}
%}
%\rhead{
%\my
%\\
%\framebox[1.1\width]{\footnotesize{``জনকল্যানে রাজস্ব''}}
%}
%}

\newcommand{\fileno}{নথি নং - ১৪৫৬/এপি/সেকশন-৫(বি)/২০২১-২০২২}
\newcommand{\btmaltno}{প্রত্যয়ন পত্র নং- ২৮৭৫৩}
\newcommand{\btmaltnodt}{তারিখ:  ২২/১২/২০২১ খ্রি:}
\newcommand{\ircno}{}
\newcommand{\ircrenewdt}{}
\newcommand{\musokr}{ডিসেম্বর-২১}
\newcommand{\rodt}{তারিখ: \hspace{2.0em}/০১/২০২২ ইং}
\newcommand{\taxtab}{
\begin{longtable}{|c|c|c|c|c|c|c|c|}
\hline
\textbf{
\makecell{
ক্রঃ \\ নং
}
}
&
\textbf{
\makecell{
পণ্যের বর্ণনা
}
}
&
\textbf{
\makecell{
পরিমাণ
}
}
& \textbf{
\makecell{
ইনভয়েস
\\
ঘোষিত
\\
এইচএসকোড
\\
ও শুল্কহার
}
}
&
\textbf{
\makecell{
প্রকৃত
\\
এইচএসকোড
\\
ও শুল্কহার
}
}
&
\textbf{
\makecell{
ইনভয়েস
\\
প্রত্যায়িত
\\
একক মূল্য
\\
(US\$)
}
}
&
\textbf{
\makecell{
শুল্কায়ন বিধিমালা ২০০০
\\
-এর বিধি
৪ ও ৫ অনুযায়ী
\\
পণ্যের
একক মূল্য
(US\$)
}
}
&
\textbf{
\makecell{
প্রস্তাবিত
\\
একক মূল্য
\\
(US\$)
}
} \\
\hline
% row
\makecell{
01
}
&
\makecell{
SQUEEZING DYEING
\\
STAINLESS ROLLER
\\
C/O. CHINA
}
&
\makecell{
1,786.00
\\
KGS
}
&
\makecell{
84519000
\\
CD-1\%
\\
SRO 113/21
\\
CPC-4000/220
}
&
\makecell{
84519000
\\
CD-1\%
\\
SRO 113/21
\\
CPC-4000/220
}
&
\makecell{
US\$
\\
9.76/KG
}
&
\makecell{
US\$
\\
1.80/KG
\\
SECTION
VALUE
}
&
\makecell{
US\$
\\
9.76/KG
} \\
\hline
\end{longtable}
}

\begin{document}
\noindent
%\scriptsize
%{\taxtab}
%\vspace*{1mm}
%\normalsize
%\noindent
\begin{minipage}[t]{0.05\linewidth}
% sat
০৮।
\end{minipage}
\begin{minipage}[t]{0.95\linewidth}
আলোচ্য চালানের বি/ই আইটেম নং-
১১, ১২, ২২, ২৭, ৩১, ৪৮, ৪৯, ৫০,
৫১, ৫২, ৫৩, ৫৫, ৬১
-এর ক্ষেত্রে
{\srooot}, {\sroootd}
এর টেবিলভূক্ত হওয়ায়
{\cpcttz}
ঘোষনা প্রদান করা হয়েছে।
SRO শর্ত মোতাবেক নিম্নোক্ত
দলিলাদি দাখিল করেছেন।
\\
\\
(ক) নবায়নকৃত IRC
\\
(খ) ১৩ ডিজিট সম্বলিত মূসক-২.৩ সনদপত্র।
\\
(গ) {\musokr} মাস পর্যন্ত অনলাইন রিটার্ণ।
\\
\end{minipage}
\begin{minipage}[t]{0.05\linewidth}
% sat
০৯।
\end{minipage}
\begin{minipage}[t]{0.95\linewidth}
আলোচ্য চালানের
বি/ই আইটেম নং-
২, ৩, ৬, ৭, ৮, ৯,
১৪, ১৫, ১৬, ১৭, ১৯, ২০, ২৩,
২৪, ৩০, ৩২, ৩৪, ৩৯, ৪০, ৪১, ৪৩, ৪৪,
৫৪, ৫৯
-এর ক্ষেত্রে
{\srootz}, {\srootzd}
এর টেবিলভূক্ত হওয়ায়
{\cpcofs}
ঘোষনা প্রদান করা হয়েছে।
SRO শর্ত মোতাবেক নিম্নোক্ত
দলিলাদি দাখিল করেছেন।
\\
\end{minipage}
\begin{minipage}[t]{0.05\linewidth}
\hspace{0em}
\end{minipage}
\begin{minipage}[t]{0.05\linewidth}
(ক)
\end{minipage}
\begin{minipage}[t]{0.90\linewidth}
আলোচ্য পণ্যচালানের বিপরীতে বাংলাদেশ
টেক্সটাইল মিলস এসোসিয়েশন এর সভাপতি
কর্তৃক স্বাক্ষরিত প্রত্যয়নপত্র।
প্রত্যয়নপত্রে আমদানিকৃত
যন্ত্রপাতি, যন্ত্রাংশ, ও উপকরণ
উক্ত প্রতিষ্ঠানের উৎপাদন প্রক্রিয়ায়
ব্যবহৃত হইবে এবং আমদানিকৃত
যন্ত্রপাতি, যন্ত্রাংশ, ও উপকরণ
{\srootz}, {\srootzd} অনুযায়ী
রেয়াতীহারে শুল্কায়নের সুপারিশ
করেন। বাংলাদেশ টেক্সটাইল মিলস
এসোসিয়েশনের দাখিলকৃত
{\btmaltno},
{\btmaltnodt}
নথির যোগাযোগ অংশে রক্ষিত
আছে, দয়া করে দেখা যেতে পারে।
\\
\end{minipage}
\begin{minipage}[t]{0.05\linewidth}
\hspace{0em}
\end{minipage}
\begin{minipage}[t]{0.05\linewidth}
(খ)
\end{minipage}
\begin{minipage}[t]{0.90\linewidth}
এসআরও শর্ত মোতাবেক আমদানিকারক ৩০০.০০ (তিনশত)
টাকার নন-জুডিশিয়াল স্ট্যাম্পে একখানা অঙ্গীকারনামা
দাখিল করেছেন। দাখিলকৃত অঙ্গীকারনামায় আমদানিকৃত
যন্ত্রপাতি, যন্ত্রাংশ, ও উপকরণ
উক্ত প্রতিষ্ঠানের উৎপাদন প্রক্রিয়ায়
ব্যবহার করবেন। অন্যথায় আমদানিকৃত উক্ত পণ্যের উপর
প্রযোজ্য স্বাভাবিক হারের শুল্ককর পরিশোধ ছাড়াও
শুল্ক আইন মোতাবেক শুল্ক কর্তৃপক্ষ
গৃহীত যে কোন আইনানুগ সিদ্ধান্ত মানিয়া নিতে বাধ্য থাকিবেন
মর্মে উল্লেখ করেছেন। আমদানিকারক কর্তৃক দাখিলকৃত
অঙ্গীকারনামা নথির যোগাযোগ অংশে রক্ষিত আছে,
দয়া করে দেখা যেতে পারে।
\\
\end{minipage}
\begin{minipage}[t]{0.05\linewidth}
% at
১০।
\end{minipage}
\begin{minipage}[t]{0.95\linewidth}
\underline{\textbf{সার্বিক পর্যালোচনা পূর্বক নিম্নোক্ত প্রস্তাব উপস্থাপন করা হলো:}}
\end{minipage}
\begin{minipage}[t]{0.05\linewidth}
\hspace{0em}
\end{minipage}
\begin{minipage}[t]{0.95\linewidth}
প্রস্তাব:
\end{minipage}
\begin{minipage}[t]{0.05\linewidth}
(ক)
\end{minipage}
\begin{minipage}[t]{0.90\linewidth}
বি/ই আইটেম নং-
১১, ১২, ২২, ২৭, ৩১, ৪৮, ৪৯, ৫০,
৫১, ৫২, ৫৩, ৫৫, ৬১
-এর ক্ষেত্রে
{\srooot}, {\sroootd}
এর টেবিলভূক্ত হওয়ায়
এবং SRO শর্ত মোতাবেক
দলিলাদি দাখিল করায়
{\cpcttz}
-তে
সাময়িক
শুল্কায়নের অনুমোদন দেয়া
যেতে পারে।
\\
\end{minipage}
\begin{minipage}[t]{0.05\linewidth}
\hspace{1em}
\end{minipage}
\begin{minipage}[t]{0.05\linewidth}
(খ)
\end{minipage}
\begin{minipage}[t]{0.90\linewidth}
বি/ই আইটেম নং-
২, ৩, ৬, ৭, ৮, ৯,
১৪, ১৫, ১৬, ১৭, ১৯, ২০, ২৩,
২৪, ৩০, ৩২, ৩৪, ৩৯, ৪০, ৪১, ৪৩, ৪৪,
৫৪, ৫৯
-এর ক্ষেত্রে
{\srootz}, {\srootzd}
এর টেবিলভূক্ত হওয়ায়
এবং SRO শর্ত মোতাবেক
দলিলাদি দাখিল করায়
{\cpcofs}
-তে
সাময়িক
শুল্কায়নের অনুমোদন দেয়া
যেতে পারে।
\\
\end{minipage}
\begin{minipage}[t]{0.05\linewidth}
\hspace{1em}
\end{minipage}
\begin{minipage}[t]{0.05\linewidth}
(গ)
\end{minipage}
\begin{minipage}[t]{0.90\linewidth}
এছাড়া বি/ই আইটেম নং-
১, ৪, ৫, ‌১০, ১৩, ১৮,
২১, ২৫, ২৬, ৩৩, ৩৫, ৩৬, ৩৭, ৩৮,
৪২, ৪৫, ৪৬, ৪৭, ৫৬, ৫৭, ৫৮, ৬০
স্বাভাবিক হারে
সাময়িক
শুল্কায়ন করা
যেতে পারে।
\\
\end{minipage}
\begin{minipage}[t]{0.05\linewidth}
\hspace{1em}
\end{minipage}
\begin{minipage}[t]{0.05\linewidth}
(ঘ)
\end{minipage}
\begin{minipage}[t]{0.90\linewidth}
কমিশনার, কাস্টমস মূল্যায়ন ও অভ্যন্তরীণ
নিরীক্ষা কমিশনারেট যথাসম্ভব স্বল্পতম সময়ে
নিরীক্ষা সমাপ্ত করিয়া নিরীক্ষা সমাপ্তির
১(এক) মাসের মধ্যে পরিশিষ্ট-৩ মোতাবেক একটি
বস্তুনিষ্ট প্রতিবেদন জাতীয় রাজস্ব বোর্ডের
সদস্য (কাস্টমস নীতি) এর নিকট এর অনুলিপি সংশ্লিষ্ট
কাস্টম হাউস বা কাস্টমস স্টেশন কর্তৃপক্ষ
বরাবরে প্রেরণ করিবেন।
নিরীক্ষায় রেয়াতী হারে আমদানিকৃত উপকরণের
ধর্তব্যযোগ্য অপব্যবহার পাওয়া গেলে
Customs Act এর Section-156
এর Sub Section (1) এর Table এর Item 10A
এর বিধান অনুযায়ী ব্যবস্থা গ্রহণ করিতে কাস্টমস
মূল্যায়ন ও অভ্যন্তরীন নিরীক্ষা কমিশনারেট আমদানি
সংশ্লিষ্ট কাস্টম কমিশনারকে অনুরোধ করিবেন।
প্রতিবেদন প্রাপ্তির পর যদি দেখা যায় যে, আমদানিকৃত
উপকরণ আমদানিকারকের প্রতিষ্ঠানে ব্যবহৃত হইয়াছে তাহা
হইলে অঙ্গীকারনামা ফেরতযোগ্য হইবে।
\\
\end{minipage}
\begin{minipage}[t]{0.05\linewidth}
\hspace{1em}
\end{minipage}
\begin{minipage}[t]{0.05\linewidth}
(ঙ)
\end{minipage}
\begin{minipage}[t]{0.90\linewidth}
{\srootz}, {\srootzd} এর শর্ত মোতাবেক
শুল্ক মূল্যায়ন ও অভ্যন্তরীন নিরীক্ষা কমিশনারেট,
গুলফেশাঁ প্লাজা (৬ষ্ট তলা), ৬৯, আউটার
সার্কুলার রোড, মগবাজার, ঢাকা-১২১৭ এবং
আমদানিকারক বরাবর পত্র প্রেরণ করা যেতে পারে।
\\
\\
সদয় অবগতি ও আদেশার্থে উপস্থাপন করা
হলো।
\end{minipage}

\end{document}
