\documentclass[12pt]{article}
\usepackage[legalpaper,
            lmargin=1in,rmargin=0.5in,
            bmargin=0.5in,tmargin=1in,includefoot]{geometry}
\usepackage{fontspec}
\usepackage{titlesec}
\usepackage{multirow}
\usepackage[colorlinks=true,urlcolor=Blue]{hyperref}
\usepackage{graphicx}
\usepackage{array}
\usepackage{makecell}
\usepackage{fancyhdr}
\usepackage[none]{hyphenat}
\usepackage{longtable}
\usepackage[dvipsnames]{xcolor}
\usepackage[banglamainfont=Kalpurush,
            banglattfont=SolaimanLipi,
            % feature=1,
            % changecounternumbering=0
           ]{latexbangla}

\pagestyle{fancy}
\fancyhf{}
\renewcommand{\headrulewidth}{0pt}
% header
\chead{
\underline{{পৃষ্ঠা - \thepage}}
\\
}
\rhead{{\filenou}}
% footer
\rfoot{চলমান পৃষ্ঠা-\thepage}

\fancypagestyle{laststyle}
{
   \fancyfoot[R]{}
}

\newcommand{\fileno}{নথি নং - ৭০০/এপি/সেকশন-৮(এ)/২০২১-২০২২}
\newcommand{\filenou}{\underline{\fileno}}
\newcommand{\product}{Refrigerator}
\newcommand{\good}{METAL SHEET}
\newcommand{\pkg}{18 PKG=69030.00KG}
\newcommand{\co}{CHINA}
\newcommand{\coship}{CHINA}
\newcommand{\vessel}{MV. MAERSK XIAMEN}
\newcommand{\rotno}{2021/4987}
\newcommand{\blno}{HACK210990857}
\newcommand{\bldt}{12.10.2021}
\newcommand{\beno}{C-1728821}
\newcommand{\bedt}{28.10.2021}
\newcommand{\lcno}{0000088821020096}
\newcommand{\lcdt}{12.09.2021}
\newcommand{\lcano}{326023}
\newcommand{\lcadt}{12.09.2021}
\newcommand{\lienbank}{ISLAMI BANK BANGLADESH LIMITED}
\newcommand{\invno}{BST2021100601}
\newcommand{\invdt}{06.10.2021}
\newcommand{\ssml}{SHAMEEM SPINNING MILLS LTD.}
\newcommand{\ssmla}{SHAFIPUR, KALIAKAIR
\\
GAZIPUR-1751, BANGLADESH.}
\newcommand{\jsml}{JAMUNA SPINNING MILLS LTD.}
\newcommand{\jsmla}{SHAFIPUR, KALIAKAIR
\\
GAZIPUR-1751, BANGLADESH.}
\newcommand{\jdl}{JAMUNA DENIMS LTD.}
\newcommand{\jdla}{SHAFIPUR, KALIAKAIR
\\
GAZIPUR-1751, BANGLADESH.}
\newcommand{\scml}{SHAMEEM COMPOSITE MILLS LTD.}
\newcommand{\jeal}{JAMUNA ELECTRONICS \& AUTOMOBILES LTD.}
\newcommand{\jdwl}{JAMUNA DENIMS \& WEAVING LTD.}
\newcommand{\htf}{HOORAIN HTF LTD.}
\newcommand{\tdj}{M/S THE DAILY JUGANTOR}
\newcommand{\tdja}{Ka-244, Kuril, Progati Sarani, Vatara PS
\\
Dhaka-1229,Bangladesh}

\newcommand{\impn}{JAMUNA ELECTRONICS \& AUTOMOBILES LTD.}
\newcommand{\impadd}{SINABHA, KALIAKAIR
\newline
PS:GAZIPUR-1750, BANGLADESH}
\newcommand{\impbin}{000146478-0103}
\newcommand{\impoldbin}{000146478}
\newcommand{\cnfn}{SHAMEEM SPINNING MILLS LTD.}
\newcommand{\cnfadd}{92,HIGH LEVEL ROAD
\newline
LALKHAN BAZAR, CHITTAGONG}
\newcommand{\cnfain}{301 08 3417}
\newcommand{\crf}{NON CRF}
\newcommand{\crfdt}{}
\newcommand{\ircno}{নং - ২৬০৩২৬১২০৪২৬৭২০}
\newcommand{\ircrenewdt}{৩০/০৬/২০২২ ইং}
\newcommand{\musokr}{সেপ্টেম্বর-২১}
\newcommand{\hscode}{7210.69.10}
\newcommand{\price}{US\$ 76,735.25}
\newcommand{\menifest}{2021/4987}
\newcommand{\srooof}{এসআরও নং -১১৪-আইন/২০২১/০৩/কাস্টমস}
\newcommand{\srooofd}{তারিখ: ২৪/০৫/২০২১ ইং}
\newcommand{\srotsz}{এসআরও নং - ৩৬০-আইন/২০১৩/২৪৬৮/কাস্টমস}
\newcommand{\srotszd}{তারিখ: ২৫/১১/২০১৩ ইং}
\newcommand{\nbrl}{জাতীয় রাজস্ব বোর্ডের পত্র নং - ০৮.০১.০০০০.০৬৮.১৮.০০৪.১৭/১৬৯}
\newcommand{\nbrld}{তারিখ: ২৭/০৬/২০২১ ইং}
\newcommand{\cpc}{(সিপিসি ৪০০০/৪০৯)}
\newcommand{\taxtab}{
\begin{longtable}{|c|c|c|c|c|c|c|c|}
\hline
\textbf{
\makecell{
ক্রঃ \\ নং
}
}
&
\textbf{
\makecell{
পণ্যের বর্ণনা
}
}
&
\textbf{
\makecell{
পরিমাণ
}
}
& \textbf{
\makecell{
ইনভয়েস
\\
ঘোষিত
\\
এইচএসকোড
\\
(শুল্কহারসহ)
}
}
&
\textbf{
\makecell{
প্রকৃত
\\
এইচএসকোড
\\
(শুল্কহারসহ)
}
}
&
\textbf{
\makecell{
ইনভয়েস
\\
প্রত্যায়িত
\\
একক মূল্য
\\
(US\$)
}
}
&
\textbf{
\makecell{
সাময়িক
\\
কালের
\\
অভিন্ন/অনুরুপ
\\
পণ্যের
\\
একক মূল্য
(US\$)
}
}
&
\textbf{
\makecell{
প্রস্তাবিত
\\
একক মূল্য
\\
(US\$)
}
} \\
\hline
% row
\makecell{
01
}
&
\makecell{
METAL SHEET
\\
($1\textrm{MM}^*1080\textrm{MM}^*\textrm{C}$)
\\
C/O. CHINA
}
&
\makecell{
45710.00
\\
KGS
}
&
\makecell{
7210.69.10
\\
CD-10\%
\\
AIT-5\%
\\
AT-3\%
\\
CPC 4000/409
}
&
\makecell{
7210.69.10
\\
CD-10\%
\\
AIT-5\%
\\
AT-3\%
\\
CPC 4000/409
\\
NBR LETTER SRO
\\
114/21
}
&
\makecell{
US\$
\\
1.12/KG
}
&
\makecell{
US\$
\\
1.02/KG
\\
DATA BASE
VALUE
\\
C-999863
\\
DT:20/06/21
}
&
\makecell{
US\$
\\
1.12\$/KG
} \\
\hline
% row
\makecell{
02
}
&
\makecell{
METAL SHEET
\\
$(1.8\times 1165 \times \textrm{C})\textrm{MM}$
\\
C/O. CHINA
}
&
\makecell{
23320.00
\\
KGS
}
&
\makecell{
7210.69.10
\\
CD-10\%
\\
AIT-5\%
\\
AT-3\%
\\
CPC 4000/409
}
&
\makecell{
7210.69.10
\\
CD-10\%
\\
AIT-5\%
\\
AT-3\%
\\
CPC 4000/409
\\
NBR LETTER SRO
\\
114/21
}
&
\makecell{
US\$
\\
1.10/KG
}
&
\makecell{
US\$
\\
1.02/KG
\\
DATA BASE
VALUE
\\
C-999863
\\
DT: 6/20/21
}
&
\makecell{
US\$
\\
1.10\$/KG
} \\
\hline
\end{longtable}
}

\begin{document}
\noindent
\begin{minipage}[t]{0.05\linewidth}
% ek
০১।
\end{minipage}
\begin{minipage}[t]{0.95\linewidth}
নির্দেশনা মোতাবেক বি/ই রেজি: নং- {\beno}, তারিখ: {\bedt}
তৎসংশ্লিষ্ট দলিলাদি নথিভূক্ত করে
পরবর্তী কার্যক্রমের জন্য উপস্থাপন করা হলো।
\\
\\
\\
\end{minipage}
\begin{minipage}[t]{0.05\linewidth}
\hspace*{0em}
\end{minipage}
\begin{minipage}[t]{0.05\linewidth}
সহকারী
\end{minipage}
\begin{minipage}[t]{0.37\linewidth}
\hspace{0em}
\end{minipage}
\begin{minipage}[t]{0.53\linewidth}
\textbf{শুল্কায়ন কর্মকর্তা}
\\
\end{minipage}
\begin{minipage}[t]{0.05\linewidth}
% dui
০২।
\end{minipage}
\begin{minipage}[t]{0.95\linewidth}
\underline{\textbf {আমদানিকৃত পণ্য চালানের
মৌলিক তথ্য:}}
\\
\end{minipage}
\footnotesize
\begin{minipage}[t]{0.05\linewidth}
\hspace*{1em}
\end{minipage}
\begin{minipage}[t]{0.40\linewidth}
(ক) বি/ই রেজি: নং ও তারিখ
\end{minipage}
\begin{minipage}[t]{0.02\linewidth}
:
\end{minipage}
\begin{minipage}[t]{0.53\linewidth}
\textbf{{\beno}} \hspace{2em} DT: {\bedt}
\\
\end{minipage}
\begin{minipage}[t]{0.05\linewidth}
\hspace*{1em}
\end{minipage}
\begin{minipage}[t]{0.40\linewidth}
(খ) আমদানিকারকের নাম, ঠিকানা
ও BIN নম্বর
\end{minipage}
\begin{minipage}[t]{0.02\linewidth}
:
\end{minipage}
\begin{minipage}[t]{0.53\linewidth}
\textbf{{\impn}}
\\
{\impadd}
\\
BIN NO. {\impbin}
\\
\end{minipage}
\begin{minipage}[t]{0.05\linewidth}
\hspace*{1em}
\end{minipage}
\begin{minipage}[t]{0.40\linewidth}
(গ) সিএন্ডএফ এজেন্টের নাম, ঠিকানা
ও AIN নম্বর
\end{minipage}
\begin{minipage}[t]{0.02\linewidth}
:
\end{minipage}
\begin{minipage}[t]{0.53\linewidth}
\textbf{{\cnfn}}
\\
{\cnfadd}
\\
AIN NO. {\cnfain}
\\
\end{minipage}
\begin{minipage}[t]{0.05\linewidth}
\hspace*{1em}
\end{minipage}
\begin{minipage}[t]{0.40\linewidth}
(ঘ) এল/সি নং ও তারিখ
\end{minipage}
\begin{minipage}[t]{0.02\linewidth}
:
\end{minipage}
\begin{minipage}[t]{0.53\linewidth}
{\lcno} \hspace{2em} DT: {\lcdt}
\\
\end{minipage}
\begin{minipage}[t]{0.05\linewidth}
\hspace*{1em}
\end{minipage}
\begin{minipage}[t]{0.40\linewidth}
(ঙ) লিয়েন ব্যাংকের নাম
\end{minipage}
\begin{minipage}[t]{0.02\linewidth}
:
\end{minipage}
\begin{minipage}[t]{0.53\linewidth}
{\lienbank}
\\
\end{minipage}
\begin{minipage}[t]{0.05\linewidth}
\hspace*{1em}
\end{minipage}
\begin{minipage}[t]{0.40\linewidth}
(চ) এলসিএ নং ও তারিখ
\end{minipage}
\begin{minipage}[t]{0.02\linewidth}
:
\end{minipage}
\begin{minipage}[t]{0.53\linewidth}
{\lcano} \hspace{2em} DT: {\lcadt}
\\
\end{minipage}
\begin{minipage}[t]{0.05\linewidth}
\hspace*{1em}
\end{minipage}
\begin{minipage}[t]{0.40\linewidth}
(ছ) বি/এল নং ও তারিখ
\end{minipage}
\begin{minipage}[t]{0.02\linewidth}
:
\end{minipage}
\begin{minipage}[t]{0.53\linewidth}
{\blno} \hspace{2em} DT: {\bldt}
\\
\end{minipage}
\begin{minipage}[t]{0.05\linewidth}
\hspace*{1em}
\end{minipage}
\begin{minipage}[t]{0.40\linewidth}
(জ) বাণিজ্যিক ইনভয়েস নং ও তারিখ
\end{minipage}
\begin{minipage}[t]{0.02\linewidth}
:
\end{minipage}
\begin{minipage}[t]{0.53\linewidth}
{\invno} \hspace{2em} DT: {\invdt}
\\
\end{minipage}
\begin{minipage}[t]{0.05\linewidth}
\hspace*{1em}
\end{minipage}
\begin{minipage}[t]{0.40\linewidth}
(ঝ) সিআরএফ নং ও ইস্যুর তারিখ
\end{minipage}
\begin{minipage}[t]{0.02\linewidth}
:
\end{minipage}
\begin{minipage}[t]{0.53\linewidth}
{\crf} \hspace{2em} {\crfdt}
\\
\end{minipage}
\begin{minipage}[t]{0.05\linewidth}
\hspace*{1em}
\end{minipage}
\begin{minipage}[t]{0.40\linewidth}
(ঞ) পণ্যের বিবরণ
\end{minipage}
\begin{minipage}[t]{0.02\linewidth}
:
\end{minipage}
\begin{minipage}[t]{0.53\linewidth}
{\good}
\\
\end{minipage}
\begin{minipage}[t]{0.05\linewidth}
\hspace*{1em}
\end{minipage}
\begin{minipage}[t]{0.40\linewidth}
(ট) পণ্যের পরিমাণ (একক সহ)
\end{minipage}
\begin{minipage}[t]{0.02\linewidth}
:
\end{minipage}
\begin{minipage}[t]{0.53\linewidth}
{\pkg}
\\
\end{minipage}
\begin{minipage}[t]{0.05\linewidth}
\hspace*{1em}
\end{minipage}
\begin{minipage}[t]{0.40\linewidth}
(ঠ) পণ্যের এইচ.এস.কোড
\end{minipage}
\begin{minipage}[t]{0.02\linewidth}
:
\end{minipage}
\begin{minipage}[t]{0.53\linewidth}
{\hscode}
\\
\end{minipage}
\begin{minipage}[t]{0.05\linewidth}
\hspace*{1em}
\end{minipage}
\begin{minipage}[t]{0.40\linewidth}
(ড) পণ্যের মূল্য (ইনভয়েস অনুযায়ী)
\end{minipage}
\begin{minipage}[t]{0.02\linewidth}
:
\end{minipage}
\begin{minipage}[t]{0.53\linewidth}
{\price}
\\
\end{minipage}
\begin{minipage}[t]{0.05\linewidth}
\hspace*{1em}
\end{minipage}
\begin{minipage}[t]{0.40\linewidth}
(ঢ) কান্ট্রি অব অরিজিন
\end{minipage}
\begin{minipage}[t]{0.02\linewidth}
:
\end{minipage}
\begin{minipage}[t]{0.53\linewidth}
{\co}
\\
\end{minipage}
\begin{minipage}[t]{0.05\linewidth}
\hspace*{1em}
\end{minipage}
\begin{minipage}[t]{0.40\linewidth}
(ণ) কান্ট্রি অব শিপমেন্ট
\end{minipage}
\begin{minipage}[t]{0.02\linewidth}
:
\end{minipage}
\begin{minipage}[t]{0.53\linewidth}
{\coship}
\\
\end{minipage}
\begin{minipage}[t]{0.05\linewidth}
\hspace*{1em}
\end{minipage}
\begin{minipage}[t]{0.40\linewidth}
(ত) জাহাজের নাম
\end{minipage}
\begin{minipage}[t]{0.02\linewidth}
:
\end{minipage}
\begin{minipage}[t]{0.53\linewidth}
{\vessel}
\end{minipage}
\begin{minipage}[t]{0.05\linewidth}
\hspace*{1em}
\end{minipage}
\begin{minipage}[t]{0.40\linewidth}
\hspace*{1.8em}পালা নং বি/এল নং
\end{minipage}
\begin{minipage}[t]{0.02\linewidth}
\hspace{1em}
\end{minipage}
\begin{minipage}[t]{0.53\linewidth}
{\menifest}, B/L {\blno}
\\
\end{minipage}
\begin{minipage}[t]{0.05\linewidth}
\hspace*{1em}
\end{minipage}
\begin{minipage}[t]{0.40\linewidth}
(থ) মেনিফিস্ট নং
\end{minipage}
\begin{minipage}[t]{0.02\linewidth}
:
\end{minipage}
\begin{minipage}[t]{0.53\linewidth}
{\menifest}
\\
\end{minipage}
\normalsize
\begin{minipage}[t]{0.05\linewidth}
% tin
০৩।
\end{minipage}
\begin{minipage}[t]{0.95\linewidth}
\underline{\textbf{শুল্কায়ন সেকশনের পর্যালোচনা:}}
\end{minipage}
\begin{minipage}[t]{0.05\linewidth}
\hspace{1em}
\end{minipage}
\begin{minipage}[t]{0.05\linewidth}
% tina
(ক)
\end{minipage}
\begin{minipage}[t]{0.90\linewidth}
\underline{\textbf{আমদানি দলিল পত্র যাচাই:}}
পণ্য চালান খালাসের জন্য নিম্নবর্ণিত দলিলাদিসহ বি/ই দাখিল করা
হয়েছে।
\\
(১) এল.সি এবং এল.সি.এ ফরম।
\\
(২) ইনভয়েস।
\\
(৩) প্যাকিং লিস্ট।
\\
(৪) বি/এল।
\\
(৫) কান্ট্রি অব অরিজিন সনদ।
\\
দলিলাদি পর্যালোচনায় এগুলো
সঠিক পাওয়া যায়।
\\
\end{minipage}
\begin{minipage}[t]{0.05\linewidth}
\hspace{1em}
\end{minipage}
\begin{minipage}[t]{0.05\linewidth}
% tinb
(খ)
\end{minipage}
\begin{minipage}[t]{0.90\linewidth}
\underline{\textbf{আমদানি যোগ্যতা যাচাই:}}
প্রচলিত আমদানিনীতি আদেশ ২০১৫-২০১৮  পর্যালোচনা করে দেখা যায় যে, পণ্যগুলি অবাধে আমদানিযোগ্য।
আলোচ্য চালানের ক্ষেত্রে আমদানিনীতি আদেশের প্রযোজ্য অন্যান্য শর্ত (কান্ট্রি অব অরিজিন, রেজিঃ
সার্টিফিকেট ইত্যাদি) প্রতিপালিত হয়েছে।
\\
\end{minipage}
\begin{minipage}[t]{0.05\linewidth}
\hspace{1em}
\end{minipage}
\begin{minipage}[t]{0.05\linewidth}
% tinc
(গ)
\end{minipage}
\begin{minipage}[t]{0.90\linewidth}
\underline{\textbf{কায়িক পরীক্ষার প্রতিবেদন পর্যালোচনা:}}
আলোচ্য পণ্যচালানটির ক্ষেত্রে প্রযোজ্য নয়।
\end{minipage}
\begin{minipage}[t]{0.05\linewidth}
\hspace{1em}
\end{minipage}
\begin{minipage}[t]{0.05\linewidth}
% tind
(ঘ)
\end{minipage}
\begin{minipage}[t]{0.90\linewidth}
\underline{\textbf{রাসায়নিক পরীক্ষা সংক্রান্ত মন্তব্য:}}
আলোচ্য পণ্যচালানটির ক্ষেত্রে প্রযোজ্য নয়।
\\
\end{minipage}
\begin{minipage}[t]{0.05\linewidth}
\hspace{1em}
\end{minipage}
\begin{minipage}[t]{0.05\linewidth}
% tine
(ঙ)
\end{minipage}
\begin{minipage}[t]{0.90\linewidth}
\underline{\textbf{এইচ.এস.কোড সঠিকতা যাচাই:}}
আমদানিকারক কর্তৃক ঘোষিত এইচ.এস.কোড দি কাস্টমস্ এ্যাক্ট ১৯৬৯ এর FIRST SCHEDULE ও
EXPLANATORY NOTES প্রচলিত এসআরও/স্থায়ী আদেশ ইত্যাদির আলোকে পরীক্ষা করা হলো।
প্রত্যায়িত এইচ.এস.কোড যথাযথ আছে।
\\
\end{minipage}
\begin{minipage}[t]{0.05\linewidth}
\hspace{1em}
\end{minipage}
\begin{minipage}[t]{0.05\linewidth}
% tinf
(চ)
\end{minipage}
\begin{minipage}[t]{0.90\linewidth}
\underline{\textbf{রেয়াতী হার বা বিশেষ মওকুফ সংক্রান্ত মন্তব্য:}}
\end{minipage}
\begin{minipage}[t]{0.1\linewidth}
\hspace{1em}
\end{minipage}
\begin{minipage}[t]{0.05\linewidth}
(১)
\end{minipage}
\begin{minipage}[t]{0.85\linewidth}
{\srooof}, {\srooofd}।
\end{minipage}
\begin{minipage}[t]{0.1\linewidth}
\hspace{1em}
\end{minipage}
\begin{minipage}[t]{0.05\linewidth}
(২)
\end{minipage}
\begin{minipage}[t]{0.85\linewidth}
{\nbrl}, {\nbrld}।
\\
\end{minipage}
\begin{minipage}[t]{0.05\linewidth}
\hspace{1em}
\end{minipage}
\begin{minipage}[t]{0.05\linewidth}
% ting
(ছ)
\end{minipage}
\begin{minipage}[t]{0.90\linewidth}
\underline{\textbf{অভিযোগ সংক্রান্ত:}} আলোচ্য পণ্যচালানে
গোপন সংবাদ দাতা, শুল্ক গোযেন্দা বা
জাতীয় রাজস্ব বোর্ডের কিংবা অন্য দপ্তর থেকে
কোন অভিযোগ পাওয়া যায় নাই।
\\
\end{minipage}
\begin{minipage}[t]{0.05\linewidth}
\hspace{1em}
\end{minipage}
\begin{minipage}[t]{0.05\linewidth}
% tinh
(জ)
\end{minipage}
\begin{minipage}[t]{0.90\linewidth}
\underline{\textbf{ন্যায় নির্ণয় সংক্রান্ত:}} প্রযোজ্য নয়।
\\
\end{minipage}
\begin{minipage}[t]{0.05\linewidth}
% pach
০৪।
\end{minipage}
\begin{minipage}[t]{0.95\linewidth}
\underline{\textbf{এসআরও শর্ত পূরণের লক্ষ্যে
দাখিলকৃত দলিলাদি:}}
\end{minipage}
\begin{minipage}[t]{0.05\linewidth}
\hspace{0em}
\end{minipage}
\begin{minipage}[t]{0.05\linewidth}
% chard
(১)
\end{minipage}
\begin{minipage}[t]{0.90\linewidth}
আলোচ্য আমদানিকারক হালনাগাদ Industrial IRC
দাখিল করেছেন যার {\ircno} এবং
{\ircrenewdt} তারিখ পর্যন্ত নবায়নকৃত আছে এবং
ব্যবসার প্রকৃতি
Type Of Industry (Sector): Manufacture of domestic appliance
উল্লেখ আছে, যা Q/R Scan এর মাধ্যমে যাচাই করে
সঠিক পাওয়া যায়।
\end{minipage}
\begin{minipage}[t]{0.05\linewidth}
\hspace{0em}
\end{minipage}
\begin{minipage}[t]{0.05\linewidth}
% chard
(২)
\end{minipage}
\begin{minipage}[t]{0.90\linewidth}
আমদানিকারক প্রতিষ্ঠানটি শিল্প প্রতিষ্ঠান হিসেবে
মূসক-২.৩ দাখিল করেছেন। মূসক-২.৩ এ ব্যবসার
প্রকৃতি Major Area of Economic
Activity: Manufacturing,
Imports
উল্লেখ আছে। NBR এর ওয়েবসাইটে প্রবেশ
করে যাচাইকরে সঠিক সঠিক পাওয়া যায়।
\end{minipage}
\begin{minipage}[t]{0.05\linewidth}
\hspace{0em}
\end{minipage}
\begin{minipage}[t]{0.05\linewidth}
% chard
(৩)
\end{minipage}
\begin{minipage}[t]{0.90\linewidth}
আমদানিকারক {\srooof}, {\srooofd} এর
শর্ত-৩ অনুযায়ী বি/ই দাখিলের অব্যবহিত
পূর্ববর্তী মাসের {\musokr} পর্যন্ত দাখিলপত্র
(রিটার্ন) দাখিল করেছেন, যা NBR এর ওয়েবসাইট
যাচাইয়ে সঠিক পাওয়া যায়। রিটার্ন সাবমিট
সংক্রান্ত হার্ডকপি নথির পত্রাংশে রক্ষিত আছে।
\end{minipage}
\begin{minipage}[t]{0.05\linewidth}
\hspace{0em}
\end{minipage}
\begin{minipage}[t]{0.05\linewidth}
% chard
(৪)
\end{minipage}
\begin{minipage}[t]{0.90\linewidth}
আমদানিকারক কর্তৃক দাখিলকৃত মূসক-৪.৩
পর্যালোচনায় দেখা যায় যে, প্রতিষ্ঠানটির উৎপাদিত
পণ্য {\product} এবং উপকরণের তালিকায়
B/E এর ঘোষণা মোতাবেক {\good} পণ্যের নাম
উল্লেখ রয়েছে।
\end{minipage}
\begin{minipage}[t]{0.05\linewidth}
\hspace{0em}
\end{minipage}
\begin{minipage}[t]{0.05\linewidth}
% chard
(৫)
\end{minipage}
\begin{minipage}[t]{0.90\linewidth}
জাতীয় রাজস্ব বোর্ডের সাধারণ আদেশ নং -
১০/মূসক/২০২০, তারিখ: ১১/০৬/২০২০ ইং
এর সংশোধিত আদেশ নং -
০৪/মূসক/২০২১, তারিখ: ০৩/০৬/২০২১ ইং
অনুযায়ী হালনাগাদ আইআরসি, মূসক-২.৩,
মূসক-৪.৩ দাখিল করেছেন বিধায় আমদানিকৃত
উপকরণের ক্ষেত্রে ৩ (তিন) শতাংশ রেয়াতি
সুবিধা প্রাপ্ত হবে।
উল্লেখ্য দাখিলকৃত মূসক-৪.৩ তে উপকরণের
তালিকায় আমদানিকৃত পণ্যের নাম উল্লেখ রয়েছে।
\\
\end{minipage}
\begin{minipage}[t]{0.05\linewidth}
% choi
০৬।
\end{minipage}
\begin{minipage}[t]{0.95\linewidth}
\underline{\textbf{শুল্কায়ন সম্পর্কিত প্রস্তাব:}}
পণ্য চালান সংক্রান্ত আমদানি দলিলপত্র
ইনভয়েস পর্যালোচনা পূর্বক পণ্যের বর্ণনা, পরিমাণ, এইচ.এস.কোড, ঘোষিত মূল্য,
সমসাময়িককালের অভিন্ন/অনুরূপ পণ্যের একক মূল্যসহ নিম্নের ছকে শুল্কায়নের প্রস্তাব উপস্থাপন
করা হলো:
\end{minipage}
\scriptsize
\begin{minipage}{1\textwidth}
{\taxtab}
\vspace{2mm}
\end{minipage}
\normalsize
\begin{minipage}[t]{0.05\linewidth}
% sat
০৭।
\end{minipage}
\begin{minipage}[t]{0.95\linewidth}
\underline{\textbf{শুল্কায়নযোগ্য মূল্য নিরুপন:}} ASYCUDA WORLD SYSTEM
-এ আলোচ্য পণ্য চালানের বি/ই দাখিলের পূর্ববর্তী ৯০ (নব্বই) দিনের
মূল্য তথ্য পর্যালোচনা করে দেখা যায়, আমদানিকৃত
পণ্যের শুল্কায়িত সর্বনিম্ন যে বিনিময় মূল্য পাওয়া যায়, উক্ত মূল্য অপেক্ষা ঘোষিত মূল্য বেশি হওয়ায়
শুল্ক মূল্যায়ন (আমদানি পণ্যের মূল্য নির্ধারণ) বিধিমালা,
২০০০ (এসআরও নং- ৫৭/আইন/২০০০/১৮২১/শুল্ক, তারিখ: ২৩/০২/২০০০)
এর বিধি-৪ অনুযায়ী আলোচ্য পণ্যচালানটি ঘোষিত
মূল্যে শুল্কায়নযোগ্য।
\\
\end{minipage}
\begin{minipage}[t]{0.05\linewidth}
% at
০৮।
\end{minipage}
\begin{minipage}[t]{0.95\linewidth}
\underline{\textbf{সার্বিক পর্যালোচনা পূর্বক নিম্নোক্ত প্রস্তাব উপস্থাপন করা হলো:}}
\\
\end{minipage}
\begin{minipage}[t]{0.05\linewidth}
\hspace{0em}
\end{minipage}
\begin{minipage}[t]{0.05\linewidth}
(ক)
\end{minipage}
\begin{minipage}[t]{0.90\linewidth}
{\srooof}, {\srooofd}
এর অনুচ্ছেদ-৭ মোতাবেক রেয়াতী হারে
আমদানিকৃত কাঁচামাল ব্যবহারপূর্বক
উৎপাদিত পণ্য মূসক আইন ও বিধিমালা
অনুযায়ী সরবরাহপূর্বক যথাযথ পরিমান মূসক
ও সম্পূরক শুল্ক প্রদান করা হয়েছে কিনা তা এ
দপ্তরকে অবহিত করার জন্য সংশ্লিষ্ট ভ্যাট বিভাগ
বরাবর পত্র প্রদান করা যেতে পারে।
পত্রের খসড়া প্রস্তুতপূর্বক অনুমোদন/স্বাক্ষরের জন্য
নথির পত্রাংশে সংযুক্ত করা হলো।
\\
\end{minipage}
\begin{minipage}[t]{0.05\linewidth}
\hspace{0em}
\end{minipage}
\begin{minipage}[t]{0.05\linewidth}
(খ)
\end{minipage}
\begin{minipage}[t]{0.90\linewidth}
আমদানিকারক প্রতিষ্ঠান কর্তৃক আলোচ্য চালানের
ক্ষেত্রে
{\srooof}, {\srooofd}
এর শর্তাবলী (প্রযোজ্য ক্ষেত্রে) পরিপালিত হয়েছে
বিধায় পণ্যচালানটি {\srooof}, {\srooofd}
{\cpc} নোট অনুচ্ছেদ-৬ এর প্রকৃত H.S Code
ও প্রস্তাবিত মূল্যে শুল্কায়ন অনুমোদন দেয়া যেতে পারে।
\end{minipage}
\newpage
\noindent
\begin{minipage}[t]{0.05\linewidth}
% at
১১।
\end{minipage}
\begin{minipage}[t]{0.95\linewidth}
নোট অনুচ্ছেদ ১ হতে ১০ সদয় দ্রষ্টব্য। আলোচ্য চালানের আমদানিকৃত পণ্য হলো Metal Sheet, নোট অনুচ্ছেদ ১০ এ ADC-1 মহোদয় খালাসকালে শুল্কায়ন সেকশনের সহকারী রাজস্ব কর্মকর্তার উপস্থিতিতে নমুনা উত্তোলন করে BUET এর Materials and Metallurgical Engineering and Technology ডিপার্টমেন্টে প্রেরণ করতে বলেন এবং অঙ্গীকার নামার ভিত্তিতে সাময়িক শুল্কায়নের অনুমোদন প্রদান করেন।
\\
\end{minipage}
\begin{minipage}[t]{0.05\linewidth}
% at
১২।
\end{minipage}
\begin{minipage}[t]{0.95\linewidth}
আমদানিকারক ৩০০/- টাকার নন জুডিশিয়াল স্ট্যাম্পে অঙ্গীকারনামা প্রদান করেছেন। অঙ্গীকারনামায় পণ্য BUET এ পরীক্ষায় যাবতীয় ব্যয় আমদানিকারক বহন করবেন বলে উল্লেখ করেন। BUET পরীক্ষায় প্রাপ্ত প্রতিবেদনে কোনো গড়মিল পরিলক্ষিত হলে The Customs Act, 1969 মোতাবেক যে কোনো সিদ্ধান্ত মেনে নিতে বাধ্য থাকিবেন।
\\
\end{minipage}
\begin{minipage}[t]{0.05\linewidth}
% at
১৩।
\end{minipage}
\begin{minipage}[t]{0.95\linewidth}
এমতাবস্থায় নোট অনুচ্ছেদ-১০ এর নির্দেশনা মোতাবেক সাময়িক শুল্কায়ন প্রস্তাব ছক আকারে উপস্থাপন করা হলো:
\\
পণ্য চালান সংক্রান্ত আমদানি দলিলপত্র
ইনভয়েস পর্যালোচনা পূর্বক পণ্যের বর্ণনা, পরিমাণ, এইচ.এস.কোড, ঘোষিত মূল্য,
সমসাময়িককালের অভিন্ন/অনুরূপ পণ্যের একক মূল্যসহ নিম্নের ছকে শুল্কায়নের প্রস্তাব উপস্থাপন
করা হলো:
\end{minipage}
\scriptsize
\begin{minipage}{1\textwidth}
{\taxtab}
\vspace{3mm}
\end{minipage}
\normalsize
\begin{minipage}[t]{0.05\linewidth}
% at
১৪।
\end{minipage}
\begin{minipage}[t]{0.95\linewidth}
প্রস্তাব:
\\
(ক) খালাসকালে শুল্কায়ন শাখা ও AIR শাখার সহকারী রাজস্ব কর্মকর্তার উপস্থিতিতে আলোচ্য পণ্যচালান হতে নমুনা উত্তোলনের প্রস্তাব অনুমোদন করা যেতে পারে।
\\
(খ) অঙ্গীকারনামা গ্রহণ করে পণ্যচালানটি সাময়িক শুল্কায়ন করা যেতে পারে।
\\
(গ) আমদানিকারক প্রতিষ্ঠান কর্তৃক আলোচ্য চালানের ক্ষেত্রে এসআরও -১১৪/আইন/২০২১-২২/কাস্টমস, তারিখ: ২৪/০৫/২০২১ খ্রি: এর শর্তাবলী (প্রযোজ্য ক্ষেত্রে) প্রতিপালিত হয়েছে বিধায় এসআরও -১১৪/আইন/২০২১-২২/কাস্টমস, তারিখ: ২৪/০৫/২০২১ খ্রি: (সিপিসি ৪০০০/৪০৯) নোট অনুচ্ছেদ -১৩ এর প্রকৃত HS Code ও প্রস্তাবিত মূল্যে সাময়িক শুল্কায়নের অনুমোদন দেয়া যেতে পারে।
\end{minipage}
\thispagestyle{laststyle}
\end{document}
