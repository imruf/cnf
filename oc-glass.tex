\documentclass[12pt]{article}
\usepackage[legalpaper,
            lmargin=1in,rmargin=0.5in,
            bmargin=1in,tmargin=1in]{geometry}
\usepackage{fontspec}
\usepackage[utf8]{inputenc}
\usepackage{titlesec}
\usepackage{multirow}
\usepackage{graphicx}
\usepackage{titling}
\usepackage{array}
\usepackage{makecell}
\usepackage{fancyhdr}
\usepackage[none]{hyphenat}
\usepackage{longtable}
\usepackage{adjustbox}
\usepackage[dvipsnames]{xcolor}
\usepackage[]{amsmath}
\usepackage[banglamainfont=Kalpurush,
            banglattfont=SolaimanLipi,
            % feature=1,
            % changecounternumbering=0
           ]{latexbangla}
% oncehsis imp file name
\newcommand{\eif}{নথি নং- এস \hspace{4em}/স্টাফ/কাস/\hspace{5em}/অংশ-}
\newcommand{\jdlf}{নথি নং- এস৫-২৪৫/স্টাফ/কাস/০৬-০৭/অংশ-২}
\newcommand{\jdwlf}{নথি নং- এস৫-১৭০/স্টাফ/কাস/১৪-১৫/অংশ-২}
\newcommand{\jealf}{নথি নং- এস-৫-৫৮৫/স্টাফ/কাস/২০১৩-২০১৪/অংশ-১২}
\newcommand{\jfclf}{নথি নং- এস৫-৫১৯/স্টাফ/কাস/২০-২১}
\newcommand{\jsmlf}{নথি নং- এস৫-৬৯১/স্টাফ/কাস/২০১৭-২০১৮/অংশ-০৩}
\newcommand{\jtrif}{নথি নং- এস-৫-১২১০/স্টাফ/কাস/১৬-১৭/অংশ-১}
\newcommand{\jwelf}{নথি নং- এস-৫-১১৪/স্টাফ/কাস/২০১৮-১৯/}
\newcommand{\scmlf}{নথি নং- এস-৫-১৬৩/স্টাফ/কাস/০৬-০৭/অংশ-৫}
\newcommand{\ssmlf}{নথি নং- এস-৫-১৪৪/স্টাফ/কাস/০৬-০৭/অংশ-১১}
\newcommand{\tdjf}{নথি নং- এস-৫-৪৪৩/স্টাফ/কাস/০৬-০৭/অংশ-}
\newcommand{\htff}{নথি নং- এস-৫-৬৮৪/স্টাফ/কাস/২০১৭-১৮/অংশ-৩}

% fileno
\newcommand{\filenou}{
\begin{minipage}[t]{0.60\linewidth}
\hspace{1em}
\end{minipage}
\begin{minipage}[t]{0.40\linewidth}
\underline{\fileno}
\end{minipage}
}


% new page
\newcommand{\nfpage}{
\newpage
\small
{\filenou}
}

% imp name
\newcommand{\ssml}{SHAMEEM SPINNING MILLS LTD.}
\newcommand{\ssmla}{SHAFIPUR, KALIAKAIR, GAZIPUR-1751, BANGLADESH.}
\newcommand{\scml}{SHAMEEM COMPOSITE MILLS LTD.}
\newcommand{\scmla}{SHAFIPUR, KALIAKAIR, GAZIPUR-1751, BANGLADESH}
\newcommand{\jsml}{JAMUNA SPINNING MILLS LTD.}
\newcommand{\jsmla}{SHAFIPUR, KALIAKAIR, GAZIPUR-1751, BANGLADESH.}
\newcommand{\jsmlaut}{JAMUNA SPINNING MILLS LTD. UNIT-2}
\newcommand{\jsmlauta}{BEJURA, SOUTH BEJURA, MADHABPUR, HOBIGONJ-3331, BANGLADESH.}
\newcommand{\jdl}{JAMUNA DENIMS LTD.}
\newcommand{\jdla}{SHAFIPUR, KALIAKAIR, GAZIPUR-1751, BANGLADESH.}
\newcommand{\jeal}{JAMUNA ELECTRONICS \& AUTOMOBILES LTD.}
\newcommand{\jeala}{SINABHA, KALIAKAIR PS, GAZIPUR-1750, BANGLADESH}
\newcommand{\jdwl}{JAMUNA DENIMS WEAVING LTD.}
\newcommand{\jdwla}{KASHIMPUR ROAD, JARUN, KONABARI PS, GAZIPUR-1751, BD}
\newcommand{\jfcl}{JAMUNA FAN AND CABLES LTD.}
\newcommand{\jfcla}{KASHIMPUR ROAD, JARUN, KONABARI PS, GAZIPUR-1751, BD}
\newcommand{\htf}{HOORAIN HTF LTD.}
\newcommand{\htfa}{BEJURA, SOUTH BEJURA, MADHABPUR, HOBIGONJ-3331, BANGLADESH.}
\newcommand{\tdj}{M/S THE DAILY JUGANTOR}
\newcommand{\tdja}{KA-244, KURIL, PROGATI SARANI, VATARA PS, DHAKA-1229, BANGLADESH}
\newcommand{\jtri}{JAMUNA TYRE AND RUBBER INDUSTRIES}
\newcommand{\jtria}{BEJURA, SOUTH BEJURA, MADHABPUR, HOBIGONJ}
\newcommand{\jwel}{JAMUNA WELDING ELECTRODE LTD.}
\newcommand{\jwela}{CHOYDANA, HAZIRPUKUR; GAZIPUR SADAR, GAZIPUR-1704; BANGLADESH}
\newcommand{\cbl}{CROWN BEVERAGE LIMITED}
\newcommand{\cbla}{SHAFIPUR, KALIAKAIR, GAZIPUR-1751, BANGLADESH}
\newcommand{\jpml}{JAMUNA PAPER MILLS LIMITED}
\newcommand{\jpmla}{BEJURA, SOUTH BEJURA MADHABPUR, HOBIGONJ, BANGLADESH}

% imp bin
\newcommand{\jealbin}{000146478-0103}
\newcommand{\tdjbin}{001960365-0101}
\newcommand{\jsmlbin}{000144425-0103}
\newcommand{\jfclbin}{001753334-0103}
\newcommand{\scmlbin}{000151754-0103}
\newcommand{\htfbin}{000146478-0103}
\newcommand{\cblbin}{000149506-0103}

% cnf name
\newcommand{\cnfn}{SHAMEEM SPINNING MILLS LTD.}
\newcommand{\cnfadd}{92, HIGH LEVEL ROAD, LALKHAN BAZAR, CHITTAGONG}
\newcommand{\cnfain}{301 08 3417}

% sros
\newcommand{\srofs}{এসআরও নং- ৫৭- আইন/২০০০/১৮২১/শুল্ক}
\newcommand{\srofsd}{তারিখ: ২৩/০২/২০০০ খ্রি:}
\newcommand{\srooot}{এসআরও নং- ১১৩- আইন/২০২১/০২/কাস্টমস}
\newcommand{\sroootd}{তারিখ: ২৪/০৫/২০২১ খ্রি:}
\newcommand{\srooof}{এসআরও নং- ১১৪- আইন/২০২১/০৩/কাস্টমস}
\newcommand{\srooofd}{তারিখ: ২৪/০৫/২০২১ খ্রি:}
\newcommand{\srootz}{এসআরও নং- ১২০- আইন/২০২১/০৯/কাস্টমস}
\newcommand{\srootzd}{তারিখ: ২৪/০৫/২০২১ খ্রিঃ}

\newcommand{\sroott}{এসআরও নং- ১৩৩- আইন/২০২১/২২/কাস্টমস}
\newcommand{\sroottd}{তারিখ: ২৪/০৫/২০২১ খ্রিঃ}

\newcommand{\srotsz}{এসআরও নং- ৩৬০- আইন/২০১৩/২৪৬৮/কাস্টমস}
\newcommand{\srotszd}{তারিখ: ২৫/১১/২০১৩ খ্রি:}

% cpcs
\newcommand{\cpcofs}{CPC- 4000/157}
\newcommand{\cpcost}{CPC- 4000/173}
\newcommand{\cpcttz}{CPC- 4000/220}
\newcommand{\cpcfzo}{CPC- 4000/401}
\newcommand{\cpcfzn}{CPC- 4000/409}
\newcommand{\cpcsso}{CPC- 4000/661}

% nbrl
\newcommand{\nbrosn}{জাতীয় রাজস্ব বোর্ডের পত্র নং- ০৮.০১.০০০০.০৬৮.১৮.০০৪.১৭/১৬৯}
\newcommand{\nbrosnd}{তারিখ: ২৭/০৬/২০২১ ইং}
\newcommand{\nbrosnt}{জাতীয় রাজস্ব বোর্ডের পত্র নং- ০৮.০১.০০০০.০৬৮.১৮.০০৪.১৭/১৬৯(৩)}
\newcommand{\nbrosntd}{তারিখ: ২৭/০৬/২০২১ খ্রি:}
\newcommand{\nbrfs}{জাতীয় রাজস্ব বোর্ডের পত্র নং- ০৮.০১.০০০০.০৩৪.০২.৩০১.১৯-৪৭}
\newcommand{\nbrfsd}{তারিখ: ১২/০৭/২০২১ খ্রি:}

% imp reg name
\newcommand{\ssmlreg}{২৩১৩৯৫০৬১৮০-এইচ, তারিখ: ০৯/০১/২০১৩}
\newcommand{\scmlreg}{৯৭০৪০১৭-এইচ তারিখ: ২১/০৪/১৯৯৭}
\newcommand{\jealreg}{এল-২৯৩০১৩০১৩৪৩১-এইচ, তারিখ: ০৯/০১/২০১৩}
\newcommand{\eireg}{\hspace{10em} তারিখ: \hspace{5em}}

% imp irc no
\newcommand{\jealirc}{260326120426720}
\newcommand{\scmlirc}{260326120041719}
\newcommand{\jfclirc}{260326120515020}

% san name
\newcommand{\maersk}{MAERSK BANGLADESH LIMITED}
\newcommand{\maerska}{58, AGRABAD COMMERCIAL AREA (3RD FLOOR), CHITTAGONG, 4100.}
\newcommand{\apl}{APL (BANGLADESH) PVT.LTD}
\newcommand{\apla}{PLOT NO. 30, 3RD FLOOR OF SURAIYA MANSION, AGRABAD, CHITTAGONG.}
\newcommand{\baridhi}{BARIDHI SHIPPING LINES LTD}
\newcommand{\baridhia}{3/F HRC BHABAN, 64-66 AGRABAD C/A, CHITTAGONG.}
\newcommand{\continentalbd}{CONTINENTAL TRADERS (BD) LIMITED}
\newcommand{\continentalbda}{73, AGRABAD C/A, CHITTAGONG.}
\newcommand{\continentaltr}{CONTINENTAL TRADERS (BD) LIMITED}
\newcommand{\continentaltra}{IQBAL BHABAN, AGRABAD C/A, CHITTAGONG.}
\newcommand{\gbx}{GBX LOGISTICS LTD}
\newcommand{\gbxa}{AYUB TRADE CENTER(1ST FLOOR), 1269/B, SK MUJIB ROAD, AGRABAD C/A, CHITTAGONG.}
\newcommand{\transmarine}{TRANSMARINE LOGISTICS LTD}
\newcommand{\transmarinea}{B.M.HEIGHTS(4TH FLOOR), 318, SK, MUJIB ROAD, AGRABAD C/A, CHITTAGONG.}
\newcommand{\trident}{TRIDENT SHIPPING LINE LTD}
\newcommand{\tridenta}{AKHTARUZZAMAN CENTER, 6TH FLOOR 21/22 AGRABAD, CHITTAGONG.}
\newcommand{\msc}{MSC MEDITERRANEAN SHIPP.CO.BD.LTD}
\newcommand{\msca}{IIUC TOWER, 4TH FLOOR, 1700/A SK.MUJIB ROAD, PLOT-09, AGRABAD, CHITTAGONG.}
\newcommand{\alviline}{ALVILINE BANGLADESH LIMITED}
\newcommand{\alvilinea}{78, AGRABAD C/A, MACCA MADINA TRADE CENTER, 9TH FLOOR, CHITTAGONG}
\newcommand{\ocean}{OCEAN NETWORK EXPRESS (BD) LTD}
\newcommand{\oceana}{IIUC TOWER (10TH FLOOR), 1700/A, PLOT-9, SK.MUJIB ROAD, AGRABAD C/A, CHITTAGONG}
\newcommand{\vega}{VEGA MARINE PVT LIMITED}
\newcommand{\vegaa}{DAAR-E SHAHIDI, 4TH FLOOR, 69 AGRABAD C/A, CTG}
\newcommand{\mega}{MEGATREND SHIPPING LINES LTD.}
\newcommand{\megaa}{MAKKAH MADINAH TRADE CENTER (16TH FLOOR), 78, AGRABAD, CHITTAGONG}
\newcommand{\famfa}{FAMFA SOLUTION LIMITED}
\newcommand{\famfaa}{BONANI, AGRABAD, CHITTAGONG}
\newcommand{\reliance}{RELIANCE SHIPPING SERVICES}
\newcommand{\reliancea}{34 AGRABAD C/A, CHITTAGONG 4100, BANGLADESH}
%\newcommand{\alvilinea}{}


% mujib logo
\newcommand{\my}{\includegraphics[height=3.2em]{pic/my.png}}

% slogan
\fancypagestyle{slogan}
{
\fancyhf{}
\renewcommand{\headrulewidth}{0pt}
% header
\lhead{
\framebox[1.1\width]{\footnotesize{``উন্নয়নের অক্সিজেন রাজস্ব''}}
}
\rhead{
\my
\\
\framebox[1.1\width]{\footnotesize{``জনকল্যানে রাজস্ব''}}
}
}

% customs
\newcommand{\tca}{The Customs Act, 1969}





\pagestyle{fancy}
\fancyhf{}
\renewcommand{\headrulewidth}{0pt}
% header
\lhead{
\framebox[1.1\width]{\footnotesize{``উন্নয়নের অক্সিজেন রাজস্ব''}}
}
\rhead{
\my
\\
\framebox[1.1\width]{\footnotesize{``জনকল্যানে রাজস্ব''}}
}
% footer

\fancypagestyle{laststyle}
{
   \fancyhead[C]{\underline{পাতা নং-}}
   \fancyhead[L]{}
   \fancyhead[R]{}
}

\newcommand{\fileno}{\eif}

\newcommand{\beno}{C-258601}
\newcommand{\bedt}{07.02.2022}
\newcommand{\rotno}{2022/697}
\newcommand{\stc}{STC:211 PKGS}
\newcommand{\co}{CHINA}
\newcommand{\vessel}{MV CAPE FORTIUS}
\newcommand{\good}{PRINTED PAPER ETC.}

\newcommand{\depo}{...................................., DEPOT, CHITTAGONG}
\newcommand{\rdepo}{...................................., DEPOT, CHITTAGONG}

\newcommand{\blno}{OOLU8890502280}
\newcommand{\bldt}{}

\newcommand{\rno}{R-\hspace{10em}}
\newcommand{\rdt}{\hspace{2em}-02-22}
\newcommand{\rodt}{তারিখ: \hspace{2.0em}/০২/২০২২ ইং}
\newcommand{\robdt}{\hspace{2.0em}/০২/২০২২ ইং}
\newcommand{\letterdt}{তারিখ: \hspace{3.0em}/০২/২০২২ ইং}
\newcommand{\frbdt}{তারিখ: \hspace{2.4em}/০২/২২}
\newcommand{\fdt}{\hspace*{3em} তারিখ: \hspace{2.4em}/০২/২২}
\newcommand{\riskbonddate}{\hspace{2em}/০২/২২ খ্রি:}

\newcommand{\riskbondno}{......................}
\newcommand{\riskbondt}{৮০০০০০/-}
\newcommand{\riskbondg}{৫০০/-}

\newcommand{\impn}{\cbl}
\newcommand{\impadd}{\cbla}
\newcommand{\impreg}{\eireg}

%\newcommand{\san}{}
%\newcommand{\sad}{}
\newcommand{\san}{... ... ... ... ... ... ... ... ... ... ... ... ... ... ... ... ... ... ... ...}
\newcommand{\sad}{... ... ... ... ... ... ... ... ... ... ... ... ... ... ... ... ... ... ... ...}


%\newcommand{\conno}{$\boldsymbol{1\times 20^\prime\hspace{0.2em}\&\hspace{0.2em}06\times 40^\prime}$}
\newcommand{\conno}{$\boldsymbol{05\times 40^\prime}$}
%\newcommand{\conno}{$\boldsymbol{05\times 20^\prime}$}

\newcommand{\conn}{] FCL Container}
\newcommand{\condt}{$\boldsymbol{\times 20^\prime}$}
\newcommand{\condf}{$\boldsymbol{\times 40^\prime}$}

\newcommand{\consh}{
\noindent{
\begin{tabular}{|l|l|l|l|l|}
\hline
\textbf{CBHU9159696}{\condf}
&
\textbf{CCLU7126944}{\condf}
&
\textbf{CCLU7197399}{\condf}
&
\textbf{OOLU8784863}{\condf}
&
\textbf{TGHU9708512}{\condf}
\\
\hline
\end{tabular}
}
}

\begin{document}
\begin{center}
\vspace*{0.2\baselineskip}
\footnotesize{গণপ্রজাতন্ত্রী বাংলাদেশ সরকার}
\\
\footnotesize{কাস্টম হাউস, চট্টগ্রাম।}
\end{center}
\begin{minipage}[t]{.76\linewidth}
\hspace{0.5em}
\end{minipage}
\begin{minipage}[t]{0.24\linewidth}
\footnotesize{রিস্ক বন্ড নং - {\riskbondno}}
\end{minipage}
\begin{minipage}[t]{.74\linewidth}
\footnotesize{{\fileno}}
\end{minipage}
\begin{minipage}[t]{.26\linewidth}
\footnotesize{\hspace{3.5em}তারিখ:{\riskbonddate}}
\\
\end{minipage}
\footnotesize{বরাবরে
\\
SHAMEEM SPINNING MILLS LTD.
\\
92, HIGH LEVEL ROAD,
\\
LALKHAN BAZAR,
\\
CHITTAGONG.}
\\
\\
\begin{minipage}[t]{0.04\linewidth}
\footnotesize{বিষয়:}
\end{minipage}
\begin{minipage}[t]{0.96\linewidth}
\textbf{\footnotesize{আমদানিকৃত ও শুল্কায়িত
[{\stc}, {\good}]
C/O. {\co}
হতে মালামালসহ
% container
[{\conno}{\conn}
চট্টগ্রাম বন্দর হতে
আমদানিকারকের কারখানা চত্বরে
নিয়ে যাওয়ার অনুমতি প্রসঙ্গে।}}
\\
\footnotesize{Vessel: {\vessel},
ROT NO. {\rotno},
B/L NO. \textbf{{\blno}}{\bldt}
\\
B/E NO. \textbf{\beno},
DT: {\bedt},
BANK/TREASURY NO: {\rno},
DT: {\rdt}}
\\
\footnotesize{
[{\conno}{\conn}}
\\
\scriptsize{
{\consh}
}
\\
\footnotesize{
\framebox[1.02\width]{IMP:{\impn}}
\\
\framebox[1.02\width]{FACTORY:{\impadd}}}
\\
\end{minipage}
\footnotesize{উপর্যুক্ত বিষয়ে আপনার পত্র নং-নাই,
{\letterdt}
এর প্রেক্ষিতে জানানো যাচ্ছে যে, সংশ্লিষ্ট
\textbf{Shipping Agent: {\san}},
{\sad}
কর্তৃক অনাপত্তি প্রদান করায় এবং সংশ্লিষ্ট
শুল্কায়ন সেকশন কর্তৃক পণ্যচালানটি শুল্কায়নের
বিপরীতে ইস্যুকৃত রিলিজ অর্ডারের ভিত্তিতে উপরোক্ত
% container
[{\conno}{\conn}
এ আমদানিকৃত মালামালসহ আমদানিকারকের নিজস্ব
কারখানা চত্বর {\impadd} -এ নিম্নোক্ত
শর্তে খালাস করার অনুমতি
প্রদান করা হলো।}
\\
\begin{minipage}[t]{0.04\linewidth}
\hspace{0.5em}
\end{minipage}
\begin{minipage}[t]{0.04\linewidth}
\scriptsize{(ক)}
\end{minipage}
\begin{minipage}[t]{.92\linewidth}
\scriptsize{আমদানিকৃত পণ্য ভর্তি
কন্টেইনার বন্দর জেটি
হতে স্থানান্তর কালে স্থায়ী
আদেশ নং-০২, তারিখ: ২৯/০৮/২০১৯ ইং
মোতাবেক (জেটি স্ক্যানিং)
কর্মকর্তা কর্তৃক
নিয়োজিত কর্মকর্তা
কন্টেইনার
নম্বর পরীক্ষা পূর্বক
সঠিক প্রাপ্তি সাপেক্ষে
শুল্কায়িত (ঘোষিত) মালামাল সঠিক পেলে
ডেলিভাররীর অনুমতি
দিবেন। সন্দেহজনক কন্টেইনার হলে
উক্ত পণ্যচালান এ দপ্তরের AIR কর্মকর্তা
পরীক্ষা করবেন এবং পরীক্ষায় পণ্য সঠিক পাওয়া গেলে
তৎক্ষনাৎ খালাসের অনুমতি প্রদান করবেন।
কোন প্রকার গড়মিল পরিলক্ষিত হলে
কন্টেইনারসহ মালামাল ডেলিভারী কার্যক্রম স্থগিত রেখে
তাৎক্ষনিকভাবে তা উর্দ্ধতন কর্তৃপক্ষের গোচরীভূত করবেন।}
\\
\end{minipage}
\begin{minipage}[t]{0.04\linewidth}
\hspace{0.5em}
\end{minipage}
\begin{minipage}[t]{0.04\linewidth}
\scriptsize{(খ)}
\end{minipage}
\begin{minipage}[t]{0.92\linewidth}
\scriptsize{আমদানিকারকের কারখানা চত্বরে
মালামাল খালাসের পর খালি কন্টেইনার পণ্য খালাসের
৩০ (ত্রিশ) দিনের মধ্যে সংশ্লিষ্ট শিপিং এজেন্ট কর্তৃক
প্রদত্ত অনাপত্তির শর্ত মোতাবেক স্থায়ী
আদেশ নং-১৩, তারিখ: ০৪/১১/০৩ ইং
অনুযায়ী শিপিং এজেন্টের মনোনীত কন্টেইনার
ইয়ার্ড/ডিপো {\rdepo} এ ফেরৎ এনে সংশ্লিষ্ট
শিপিং এজেন্ট, কাস্টম কর্তৃপক্ষের মাধ্যমে হস্তান্তর করে
স্থানান্তরের প্রমাণস্বরূপ ডুপ্লিকেট রিস্ক বন্ডের পিছনে
সংশ্লিষ্ট সকল পক্ষের গৃহীত স্বাক্ষরসহ
(রিস্ক বন্ড অবমুক্ত করার জন্য)
স্টাফ শাখায় দাখিল করবেন।
তাছাড়া সহকারী/রাজস্ব কর্মকর্তা সংশ্লিষ্ট (ডিপো) কর্তৃক
স্বাক্ষরিত প্রতিবেদনের কপি সংযুক্ত করতে হবে।
ইহার ভিত্তিতে আমদানিকারক কর্তৃক দেয়
রিস্ক বন্ড বাতিল করা হবে।}
\\
\\
\\
\\
\\
\\
\end{minipage}
\begin{minipage}[t]{0.60\linewidth}
\hspace{1em}
\end{minipage}
\begin{minipage}[t]{0.40\linewidth}
\begin{center}
\footnotesize{(মোঃ জাহিদুল ইসলাম জামাল)}
\\
\footnotesize{রাজস্ব কর্মকর্তা (স্টাফ শাখা)}
\\
\scriptsize{সহকারী/ডেপুটি কমিশনার অব কাস্টমস এর পক্ষে}
\\
\scriptsize{কাস্টম হাউস, চট্টগ্রাম।}
\\
\scriptsize{{\rodt}}
\end{center}
\end{minipage}
\\
\scriptsize{{\fileno}}
\\
\underline{\scriptsize{অনুলিপি অবগতি ও প্রয়োজনীয় ব্যবস্থা গ্রহনের জন্য প্রেরণ করা হলো}}
\\
\begin{minipage}[t]{0.06\linewidth}
\scriptsize{০১।}
\end{minipage}
\begin{minipage}[t]{0.94\linewidth}
\scriptsize{বিভাগীয় কর্মকর্তা, কাস্টমস এক্সাইজ ও ভ্যাট, (গাজীপুর বিভাগ),
বাড়ি নং- ৩১৯,
খন্দকার কমপ্লেক্স, নলজানী,
চান্দানা চৌরাস্তা, জয়দেপপুর রোড,
গাজীপুর,
পন্যচালানটি উপরোক্ত প্রতিষ্ঠানে যথাযথভাবে
ব্যবহার করা হচ্ছে/হয়েছে কিনা সে
সম্পর্কিত একটি প্রতিবেদন নিম্ন স্বাক্ষরকারীর
দপ্তরে প্রেরণ করার জন্য সংশ্লিষ্ট সার্কেলকে
প্রয়োজনীয় নির্দেশ দানের জন্য আপনাকে
বিশেষভাবে অনুরোধ করা হলো।
}
\end{minipage}
\begin{minipage}[t]{0.06\linewidth}
\scriptsize{০২।}
\end{minipage}
\begin{minipage}[t]{0.94\linewidth}
\scriptsize{পরিচালক (নিরাপত্তা), চট্টগ্রাম বন্দর কর্তৃপক্ষ, চট্টগ্রাম (নির্ধারিত গেইট দিয়ে কন্টেইনার নির্গমন নিশ্চিত করার জন্য অনুরোধ করা হলো)।}
\end{minipage}
\begin{minipage}[t]{0.06\linewidth}
\scriptsize{০৩।}
\end{minipage}
\begin{minipage}[t]{0.94\linewidth}
\scriptsize{টার্মিনাল ম্যানেজার, চট্টগ্রাম বন্দর কর্তৃপক্ষ, চট্টগ্রাম।}
\end{minipage}
\begin{minipage}[t]{0.06\linewidth}
\scriptsize{০৪।}
\end{minipage}
\begin{minipage}[t]{0.94\linewidth}
\scriptsize{রাজস্ব কর্মকর্তা (জেটি কাস্টমস) স্ক্যানিং, কাস্টম হাউস, চট্টগ্রাম।}
\end{minipage}
\begin{minipage}[t]{0.06\linewidth}
\scriptsize{০৫।}
\end{minipage}
\begin{minipage}[t]{0.94\linewidth}
\scriptsize{সহকারী রাজস্ব কর্মকর্তা, সংশ্লিষ্ট গেইট ডিভিশন/এমপিবি-২/৩, কাস্টম হাউস, চট্টগ্রাম।}
\end{minipage}
\begin{minipage}[t]{0.06\linewidth}
\scriptsize{০৬।}
\end{minipage}
\begin{minipage}[t]{0.94\linewidth}
\scriptsize{সহকারী রাজস্ব কর্মকর্তা, {\rdepo}}
\end{minipage}
\begin{minipage}[t]{0.06\linewidth}
\scriptsize{০৭।}
\end{minipage}
\begin{minipage}[t]{0.94\linewidth}
\scriptsize{সহকারী রাজস্ব কর্মকর্তা, (স্ক্যানার), সিপিআর/এমপিবি-২/এসজিএস অফিস, বন্দর, চট্টগ্রাম।}
\end{minipage}
\begin{minipage}[t]{0.06\linewidth}
\scriptsize{০৮।}
\end{minipage}
\begin{minipage}[t]{0.94\linewidth}
\scriptsize{শিপিং এজেন্ট:
{\san}, {\sad}}
\end{minipage}
\begin{minipage}[t]{0.06\linewidth}
\scriptsize{০৯।}
\end{minipage}
\begin{minipage}[t]{0.94\linewidth}
\scriptsize{IMP: {\impn}
FACTORY:{\impadd}}
\end{minipage}
\begin{minipage}[t]{0.06\linewidth}
\scriptsize{১০।}
\end{minipage}
\begin{minipage}[t]{0.94\linewidth}
\scriptsize{অফিস কপি।}
\\
\\
\\
\\
\\
\\
\end{minipage}
\begin{minipage}[t]{0.60\linewidth}
\hspace{1em}
\end{minipage}
\begin{minipage}[t]{0.40\linewidth}
\begin{center}
\scriptsize{{\robdt}}
\\
\footnotesize{(মোঃ জাহিদুল ইসলাম জামাল)}
\\
\footnotesize{রাজস্ব কর্মকর্তা (স্টাফ শাখা)}
\\
\scriptsize{সহকারী/ডেপুটি কমিশনার অব কাস্টমস এর পক্ষে}
\\
\scriptsize{কাস্টম হাউস, চট্টগ্রাম।}
\end{center}
\end{minipage}
{\nfpage}
\normalsize
\begin{minipage}[t]{0.04\linewidth}
।
\end{minipage}
\begin{minipage}[t]{0.96\linewidth}
বিষয়:\hspace{1em}\textbf{আমদানিকৃত ও শুল্কায়িত
[{\stc}, {\good}]
C/O. {\co}
হতে মালামালসহ
% container
[{\conno}{\conn}
চট্টগ্রাম বন্দর হতে
আমদানিকারকের কারখানা চত্বরে
নিয়ে যাওয়ার অনুমতি প্রসঙ্গে।}
\\
\footnotesize{Vessel: {\vessel},
ROT NO. {\rotno},
B/L NO. \textbf{{\blno}}{\bldt}
\\
B/E NO. \textbf{\beno},
DT: {\bedt},
BANK/TREASURY NO: {\rno},
DT: {\rdt}}
\\
\footnotesize{
[{\conno}{\conn}}
\\
\scriptsize{
{\consh}
}
\\
\footnotesize{
\framebox[1.02\width]{IMP:{\impn}}
\\
\framebox[1.02\width]{FACTORY:{\impadd}}
\\
\framebox[1.02\width]{ছাড়কারী:{\cnfn},{\cnfadd}}}
\\
\end{minipage}
\begin{minipage}[t]{0.04\linewidth}
\hspace{0.0em}
\end{minipage}
\begin{minipage}[t]{0.96\linewidth}
\normalsize
উপরোক্ত পত্র সংযোজনীসহ নথির পত্রাংশে উপস্থাপিত।
অনুগ্রহ পূর্বক দেখা যেতে পারে।
উপরোক্ত আমদানিকারক কর্তৃক আমদানিকৃত
[{\conno}{\conn}
এর মাধ্যমে আনীত মালামাল সহ
আমদানিকারকের কারখানা চত্বর
{\impadd}-এ
নিয়ে খালাস করার অনুমতি চেয়ে
সিএন্ডএফ এজেন্ট: {\cnfn}, {\cnfadd}
আবেদন করেছেন।
\\
\end{minipage}
\begin{minipage}[t]{0.04\linewidth}
।
\end{minipage}
\footnotesize
\begin{minipage}[t]{0.96\linewidth}
আমদানিকৃত মালামাল সহ কন্টেইনার ডেলিভারী
সংক্রান্ত বিদ্যমান স্থায়ী আদেশ নং-০৮,
তারিখ: ২৯/০৭/০৩ খ্রি: এর আলোকে
আমদানিকারকের অত্র প্রতিষ্ঠানটি বিনিয়োগ
বোর্ডের আওতায় নিবন্ধিত একটি শিল্প
প্রতিষ্ঠান। যার নিবন্ধন সংখ্যা {\impreg}।
উক্ত স্থায়ী আদেশের আলোকে
নিম্নোক্ত শর্তে কন্টেইনার ডেলিভারীর
অনুমতি প্রদান করা যেতে পারে।
\\
\begin{minipage}[t]{0.04\linewidth}
(ক)
\end{minipage}
\begin{minipage}[t]{0.92\linewidth}
আমদানিকৃত পণ্য ভর্তি
কন্টেইনার বন্দর জেটি
হতে স্থানান্তর কালে স্থায়ী
আদেশ নং-০২, তারিখ: ২৯/০৮/২০১৯ ইং
মোতাবেক (জেটি স্ক্যানিং)
কর্মকর্তা কর্তৃক
নিয়োজিত কর্মকর্তা
কন্টেইনার
নম্বর পরীক্ষা পূর্বক
সঠিক প্রাপ্তি সাপেক্ষে
শুল্কায়িত (ঘোষিত) মালামাল সঠিক পেলে
ডেলিভাররীর অনুমতি
দিবেন। সন্দেহজনক কন্টেইনার হলে
উক্ত পণ্যচালান এ দপ্তরের AIR কর্মকর্তা
পরীক্ষা করবেন এবং পরীক্ষায় পণ্য সঠিক পাওয়া গেলে
তৎক্ষনাৎ খালাসের অনুমতি প্রদান করবেন।
কোন প্রকার গড়মিল পরিলক্ষিত হলে
কন্টেইনারসহ মালামাল ডেলিভারী কার্যক্রম স্থগিত রেখে
তাৎক্ষনিকভাবে তা উর্দ্ধতন কর্তৃপক্ষের গোচরীভূত করবেন।
\\
\end{minipage}
\end{minipage}
\begin{minipage}[t]{0.04\linewidth}
।
\end{minipage}
\begin{minipage}[t]{0.96\linewidth}
আমদানিকারকের কারখানা চত্বরে মালামাল
খালাসের পর খালি কন্টেইনার পণ্য খালাস
পরবর্তী ৩০ (ত্রিশ) দিনের মধ্যে
সংশ্লিষ্ট শিপিং এজেন্টের নিজস্ব কন্টেইনার
ইয়ার্ড/ডিপোট-তে ফেরত প্রদান করবেন
মর্মে আমাদানিকারক কর্তৃক রিস্ক বন্ড দাখিল
করা হয়েছে।
সংশ্লিষ্ট মালামালের শুল্কায়িত বি/ই,
রিলিজ অর্ডার, এ্যাসেসমেন্ট নোটিশ,
ইনভয়েস, বি/এল, প্যাকিং লিস্ট,
ভ্যাট রেজিস্ট্রেশন, বিনিয়োগ বোর্ডের
পত্রের ফটোকপি ইত্যাদি দাখিল করা হয়েছে।
\\
\end{minipage}
\begin{minipage}[t]{0.04\linewidth}
।
\end{minipage}
\begin{minipage}[t]{0.96\linewidth}
সিএন্ডএফ এজেন্টের আবেদনের প্রক্ষিতে
সংশ্লিষ্ট শিপি এজেন্ট:
{\san}, {\sad}
কর্তৃক কন্টেইনার স্থানন্তরের কোন আপত্তি নেই
মর্মে অনাপত্তি প্রদান করা হয়েছে।
\\
\end{minipage}
\begin{minipage}[t]{0.04\linewidth}
।
\end{minipage}
\begin{minipage}[t]{0.96\linewidth}
আমদানিকারকের কারখানা চত্বরে মালামাল
খালাসের পর খালি কন্টেইনার পণ্য খালাস পরবর্তী
৩০ (ত্রিশ) দিনের মধ্যে
সংশ্লিষ্ট শিপিং এজেন্ট কর্তৃক প্রদত্ত
অনাপত্তির শর্ত মোতাবেক
স্থায়ী আদেশ নং-১৩,
তারিখ: ০৪/১১/০৩ ইং
অনুযায়ী শিপিং এজেন্টের মনোনীত
কন্টেইনার ডিপো/ইয়ার্ড {\rdepo}-এ
ফেরৎ এনে সংশ্লিষ্ট শিপিং এজেন্ট,
কাস্টম কর্তৃপক্ষের মাধ্যমে হস্তান্তর করে
স্থানান্তরের প্রমাণস্বরূপ ডুপ্লিকেট রিস্ক বন্ডের পিছনে
সংশ্লিষ্ট সকল পক্ষের গৃহীত স্বাক্ষরসহ
(রিস্ক বন্ড অবমুক্ত করার জন্য)
স্টাফ শাখায় দাখিল করবেন।
তাছাড়া আদেশ মোতাবেক
সহকারী/রাজস্ব কর্মকর্তা সংশ্লিষ্ট (ডিপো) কর্তৃক
স্বাক্ষরিত প্রতিবেদনের কপি সংযুক্ত করতে হবে।
ইহার ভিত্তিতে আমদানিকারক কর্তৃক দেয়
রিস্ক বন্ড বাতিল করা হবে।
\\
\end{minipage}
\begin{minipage}[t]{0.04\linewidth}
।
\end{minipage}
\begin{minipage}[t]{0.96\linewidth}
উল্লেখ্য অত্র আমদানিকারকের অনুকূলে ইতিপূর্বে
অনুমোদিত রিস্ক বন্ডের মেয়াদ উত্তীর্ণ কোন
রিস্ক বন্ড বাতিলের অপেক্ষায় অবশিষ্ট নাই।
\\
\end{minipage}
\begin{minipage}[t]{0.04\linewidth}
।
\end{minipage}
\begin{minipage}[t]{0.96\linewidth}
বিনিয়োগ বোর্ডের প্রত্যয়ন পত্র,
ভ্যাট রেজিষ্ট্রেশন কপি,
আউটপাশ বি/ই এবং প্রেরিতব্য পত্রে
আমদানিকারকের প্রতিষ্ঠানের কারখানা চত্বরের
ঠিকানা সঠিক আছে।
\\
\end{minipage}
\begin{minipage}[t]{0.04\linewidth}
।
\end{minipage}
\begin{minipage}[t]{0.96\linewidth}
এমতাবস্থায় আমদানীকারকের পক্ষে
পেশকৃত আবেদনপত্রের অনুকূলে
উপস্থাপিত রিস্ক বন্ডটি গ্রহণ করা যেতে পারে।
উল্লেখ্য, রিস্ক বন্ডে-
{\riskbondt}
টাকার বিপরীতে
{\riskbondg}
টাকার নন-জুডিশিয়াল
স্ট্যাম্পে রিস্ক বন্ড নং {\riskbondno}
{তারিখ:\riskbonddate} দাখিল করেছেন।
\\
\end{minipage}
\begin{minipage}[t]{0.04\linewidth}
।
\end{minipage}
\begin{minipage}[t]{0.96\linewidth}
বিনিয়োগ বোর্ডের নিবন্ধন পত্র এবং
মূসক ২.৩ নথিতে সংযুক্ত আছে।
অবমুক্ত অবশিষ্ট নেই।
\\
\\
\\
\\
\end{minipage}
\normalsize
\begin{minipage}[t]{0.6\linewidth}
\hspace{1em}
\end{minipage}
\begin{minipage}[t]{0.2\linewidth}
এআরও
\end{minipage}
\begin{minipage}[t]{0.2\linewidth}
আরও
\end{minipage}
\thispagestyle{laststyle}

\end{document}
