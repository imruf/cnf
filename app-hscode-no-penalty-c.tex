\documentclass[12pt]{article}
\usepackage[a4paper,
            lmargin=1in,rmargin=1in,
            bmargin=2in,tmargin=2in]{geometry}
\usepackage{fontspec}
\usepackage{titlesec}
\usepackage{multirow}
\usepackage[colorlinks=true,urlcolor=Blue]{hyperref}
\usepackage{graphicx}
\usepackage{array}
\usepackage{makecell}
\usepackage{fancyhdr}
\usepackage[none]{hyphenat}
\usepackage{longtable}
\usepackage[dvipsnames]{xcolor}
\usepackage[banglamainfont=Kalpurush,
            banglattfont=SolaimanLipi,
            % feature=1,
            % changecounternumbering=0
           ]{latexbangla}

\pagestyle{fancy}
\fancyhf{}
\renewcommand{\headrulewidth}{0pt}
% header
\chead{
\underline{{পৃষ্ঠা - \thepage}}
}
% footer
\rfoot{চলমান পৃষ্ঠা-\thepage}

\fancypagestyle{laststyle}
{
   \fancyfoot[R]{}
   \fancyfoot[L]{}
   \fancyfoot[C]{}
   \fancyhead[R]{}
   \fancyhead[L]{}
   \fancyhead[C]{}
}

\newcommand{\beno}{C-1829413}
\newcommand{\bedt}{15.11.21}
\newcommand{\co}{CHINA}
\newcommand{\pkg}{167PKG}
\newcommand{\good}{RAW MATERIALS}
\newcommand{\impn}{JAMUNA ELECTRONICS \& AUTOMOBILES LTD.}
\newcommand{\cnfn}{SHAMEEM SPINNING MILLS LTD.}
\newcommand{\cnfadd}{92,HIGH LEVEL ROAD,\\ LALKHAN BAZAR,\\CHITTAGONG}
\newcommand{\cnfain}{301083417}
\newcommand{\impadd}{SINABHA, KALIAKAIR, PS:GAZIPUR-1750, BANGLADESH}
\newcommand{\impbin}{000146478-0103}
\newcommand{\sro}{এস.আর.ও নং ১১৪ - আইন/২০২১/০৩/কাস্টমস}
\newcommand{\srodt}{তারিখ: ২৪ মে, ২০২১ খ্রি:}
\newcommand{\cpc}{4000/409}

\begin{document}
\noindent
বরাবর
\\
\begin{minipage}[t]{0.06\linewidth}
\hspace{1em}
\end{minipage}
\begin{minipage}[t]{0.94\linewidth}
কমিশনার অব কাস্টমস
\\
কাস্টম হাউজ
\\
চট্টগ্রাম।
\\
\\
\end{minipage}
\begin{minipage}[t]{0.06\linewidth}
বিষয়:
\end{minipage}
\begin{minipage}[t]{0.94\linewidth}
ন্যায়-নির্ণয় ব্যতিরেকে পণ্য চালান ছাড় দেওয়া প্রসঙ্গে।
\\
বি/ই নং {\beno}, DT: {\bedt}
\\
\end{minipage}
জনাব,
\\
\hspace*{2.7em}বিনীত নিবেদন এই যে, আমাদের আমদানিকারক {\impn},
{\co} হতে {\pkg} রেফ্রিজরেটরের {\good} আমদানি করেন। পণ্য চালানটি জেটি কাস্টম ও AIR কর্তৃক ১০০\% কায়িক পরীক্ষা করা হয়। কায়িক পরীক্ষার প্রতিবেদনে বি/ই -এর ২ নং আইটেমের HS Code পরিবর্তন হয়। ঘোষিত HS Code এবং পরিবর্তিত  HS Code এর শুল্কহার সমান, অর্থাৎ কোনো রাজস্ব হানি ঘটেনি।
\\
\\
অতএব, মহোদয় শিল্প প্রতিষ্ঠান বিবেচনায় পণ্য চালানটি ন্যায়-নির্ণয়
ব্যতিরেকে ছাড় দেওয়ার জন্য বিনীত ভাবে অনুরোধ জানাচ্ছি।
\\
\\
\\
\\
\begin{minipage}[t]{0.50\linewidth}
\hspace{1em}
\end{minipage}
\begin{minipage}[t]{0.60\linewidth}
বিনীত নিবেদক
\\
\\
\\
\\
\\
{\cnfn}
\end{minipage}
\thispagestyle{laststyle}

\end{document}

