\documentclass[12pt]{article}
\usepackage[legalpaper,
            lmargin=1in,rmargin=0.5in,
            bmargin=0.5in,tmargin=1in,includefoot]{geometry}
\usepackage{fontspec}
\usepackage{titlesec}
\usepackage{multirow}
\usepackage[colorlinks=true,urlcolor=Blue]{hyperref}
\usepackage{graphicx}
\usepackage{array}
\usepackage{makecell}
\usepackage{fancyhdr}
\usepackage[none]{hyphenat}
\usepackage{longtable}
\usepackage[dvipsnames]{xcolor}
\usepackage[banglamainfont=Kalpurush,
            banglattfont=SolaimanLipi,
            % feature=1,
            % changecounternumbering=0
           ]{latexbangla}

\pagestyle{fancy}
\fancyhf{}
\renewcommand{\headrulewidth}{0pt}
% header
\chead{
\underline{{পৃষ্ঠা - \thepage}}
\\
}
\rhead{{\filenou}}
% footer
\rfoot{চলমান পৃষ্ঠা-\thepage}

\fancypagestyle{laststyle}
{
   \fancyfoot[R]{}
}

\newcommand{\fileno}{নথি নং - ৭৭১/এপি/সেকশন-৮(এ)/২০২১-২০২২}
\newcommand{\filenou}{\underline{\fileno}}
\newcommand{\product}{Fan and Cables}
\newcommand{\good}{ALUMINUM SHEET}
\newcommand{\pkg}{30 PKG=59343.00KG}
\newcommand{\co}{CHINA}
\newcommand{\coship}{CHINA}
\newcommand{\vessel}{MV. MAERSK SONGKHLA}
\newcommand{\rotno}{2021/5100}
\newcommand{\blno}{268162247}
\newcommand{\bldt}{20.10.2021}
\newcommand{\beno}{C-1779804}
\newcommand{\bedt}{06.11.2021}
\newcommand{\lcno}{0000088821020098}
\newcommand{\lcdt}{18.09.2021}
\newcommand{\lcano}{332650}
\newcommand{\lcadt}{18.09.2021}
\newcommand{\lienbank}{ISLAMI BANK BANGLADESH LIMITED}
\newcommand{\invno}{QHZ210914U}
\newcommand{\invdt}{09.10.2021}
\newcommand{\ssml}{SHAMEEM SPINNING MILLS LTD.}
\newcommand{\ssmla}{SHAFIPUR, KALIAKAIR
\\
GAZIPUR-1751, BANGLADESH.}
\newcommand{\jsml}{JAMUNA SPINNING MILLS LTD.}
\newcommand{\jsmla}{SHAFIPUR, KALIAKAIR
\\
GAZIPUR-1751, BANGLADESH.}
\newcommand{\jdl}{JAMUNA DENIMS LTD.}
\newcommand{\jdla}{SHAFIPUR, KALIAKAIR
\\
GAZIPUR-1751, BANGLADESH.}
\newcommand{\scml}{SHAMEEM COMPOSITE MILLS LTD.}
\newcommand{\jeal}{JAMUNA ELECTRONICS \& AUTOMOBILES LTD.}
\newcommand{\jdwl}{JAMUNA DENIMS \& WEAVING LTD.}
\newcommand{\jfcl}{JAMUNA FAN AND CABLES LTD.}
\newcommand{\jfcla}{KASHIMPUR ROAD, JARUN
\\
KONABARI PS, GAZIPUR-1751,BD}
\newcommand{\htf}{HOORAIN HTF LTD.}
\newcommand{\tdj}{M/S THE DAILY JUGANTOR}
\newcommand{\tdja}{Ka-244, Kuril, Progati Sarani, Vatara PS
\\
Dhaka-1229,Bangladesh}

\newcommand{\impn}{\jfcl}
\newcommand{\impadd}{\jfcla}
\newcommand{\impbin}{001753334-0103}
\newcommand{\cnfn}{SHAMEEM SPINNING MILLS LTD.}
\newcommand{\cnfadd}{92,HIGH LEVEL ROAD
\newline
LALKHAN BAZAR, CHITTAGONG}
\newcommand{\cnfain}{301 08 3417}
\newcommand{\crf}{NON CRF}
\newcommand{\crfdt}{}
\newcommand{\ircno}{আইআরসি নং - ২৬০৩২৬১২০৪২৬৭২০}
\newcommand{\ircrenewdt}{৩০/০৬/২০২২ ইং}
\newcommand{\musokr}{সেপ্টেম্বর-২১}
\newcommand{\hscode}{7606.12.10}
\newcommand{\price}{US\$ 216,601.95}
\newcommand{\srooof}{এসআরও নং -১১৪-আইন/২০২১/০৩/কাস্টমস}
\newcommand{\srooofd}{তারিখ: ২৪/০৫/২০২১ ইং}
\newcommand{\srotsz}{এসআরও নং - ৩৬০-আইন/২০১৩/২৪৬৮/কাস্টমস}
\newcommand{\srotszd}{তারিখ: ২৫/১১/২০১৩ ইং}
\newcommand{\nbrl}{জাতীয় রাজস্ব বোর্ডের পত্র নং - ০৮.০১.০০০০.০৬৮.১৮.০০৪.১৭/১৬৯}
\newcommand{\nbrld}{তারিখ: ২৭/০৬/২০২১ ইং}
\newcommand{\cpc}{(সিপিসি ৪০০০/৪০১)}
\newcommand{\taxtab}{
\noindent
\begin{longtable}{|l|l|l|l|l|l|l|l|}
\hline
\textbf{
\makecell{
ক্রঃ \\ নং
}
}
&
\textbf{
\makecell{
পণ্যের বর্ণনা
}
}
&
\textbf{
\makecell{
পরিমাণ
}
}
& \textbf{
\makecell{
ইনভয়েজ
\\
ঘোষিত
\\
এইচএসকোড
\\
(শুল্কহারসহ)
}
}
&
\textbf{
\makecell{
প্রকৃত
\\
এইচএসকোড
\\
(শুল্কহারসহ)
}
}
&
\textbf{
\makecell{
ইনভয়েজ
\\
প্রত্যায়িত
\\
একক মূল্য
\\
(US\$)
}
}
&
\textbf{
\makecell{
সাময়িক
\\
কালের
\\
অভিন্ন/অনুরুপ
\\
পণ্যের
\\
একক মূল্য
(US\$)
}
}
&
\textbf{
\makecell{
প্রস্তাবিত
\\
একক মূল্য
\\
(US\$)
}
} \\
\hline
% row
\makecell{
01
}
&
\makecell{
PCM COIL GRAY
\\
($0.5^*601^*\textrm{MM}$)
}
&
\makecell{
125,682.00
\\
KGS
}
&
\makecell{
7210.70.10
\\
CD-10\%
\\
AIT-5\%
\\
AT-3\%
\\
CPC 4000/409
}
&
\makecell{
7210.70.10
\\
CD-10\%
\\
AIT-5\%
\\
AT-3\%
\\
CPC 4000/409
\\
NBR LETTER SRO
\\
114/21
}
&
\makecell{
US\$
\\
1.38/KG
}
&
\makecell{
US\$
\\
1.08/KG
\\
DATA BASE
\\
VALUE
\\
C-1538597
\\
DT:21/09/21
}
&
\makecell{
US\$
\\
1.38\$/KG
} \\
\hline
% row
\makecell{
02
}
&
\makecell{
PCM COIL RED
\\
($0.5^*601^*\textrm{MM}$)
}
&
\makecell{
6,770.00
\\
KGS
}
&
\makecell{
7210.70.10
\\
CD-10\%
\\
AIT-5\%
\\
AT-3\%
\\
CPC 4000/409
}
&
\makecell{
7210.70.10
\\
CD-10\%
\\
AIT-5\%
\\
AT-3\%
\\
CPC 4000/409
\\
NBR LETTER SRO
\\
114/21
}
&
\makecell{
US\$
\\
1.38/KG
}
&
\makecell{
US\$
\\
1.08/KG
\\
DATA BASE
\\
VALUE
\\
C-1538597
\\
DT:21/09/21
}
&
\makecell{
US\$
\\
1.38\$/KG
} \\
\hline
% row
\makecell{
03
}
&
\makecell{
PCM COIL BLACK
\\
($0.5^*601^*\textrm{MM}$)
}
&
\makecell{
14,809.00
\\
KGS
}
&
\makecell{
7210.70.10
\\
CD-10\%
\\
AIT-5\%
\\
AT-3\%
\\
CPC 4000/409
}
&
\makecell{
7210.70.10
\\
CD-10\%
\\
AIT-5\%
\\
AT-3\%
\\
CPC 4000/409
\\
NBR LETTER SRO
\\
114/21
}
&
\makecell{
US\$
\\
1.38/KG
}
&
\makecell{
US\$
\\
1.08/KG
\\
DATA BASE
\\
VALUE
\\
C-1538597
\\
DT:21/09/21
}
&
\makecell{
US\$
\\
1.38\$/KG
} \\
\hline
% row
\makecell{
04
}
&
\makecell{
PCM COIL BLUE
\\
($0.5^*601^*\textrm{MM}$)
}
&
\makecell{
66,866.00
\\
KGS
}
&
\makecell{
7210.70.10
\\
CD-10\%
\\
AIT-5\%
\\
AT-3\%
\\
CPC 4000/409
}
&
\makecell{
7210.70.10
\\
CD-10\%
\\
AIT-5\%
\\
AT-3\%
\\
CPC 4000/409
\\
NBR LETTER SRO
\\
114/21
}
&
\makecell{
US\$
\\
1.38/KG
}
&
\makecell{
US\$
\\
1.08/KG
\\
DATA BASE
\\
VALUE
\\
C-1538597
\\
DT:21/09/21
}
&
\makecell{
US\$
\\
1.38\$/KG
} \\
\hline
% row
\makecell{
05
}
&
\makecell{
PCM COIL WHITE
\\
($0.5^*601^*\textrm{MM}$)
}
&
\makecell{
45,360.00
\\
KGS
}
&
\makecell{
7210.70.10
\\
CD-10\%
\\
AIT-5\%
\\
AT-3\%
\\
CPC 4000/409
}
&
\makecell{
7210.70.10
\\
CD-10\%
\\
AIT-5\%
\\
AT-3\%
\\
CPC 4000/409
\\
NBR LETTER SRO
\\
114/21
}
&
\makecell{
US\$
\\
1.38/KG
}
&
\makecell{
US\$
\\
1.08/KG
\\
DATA BASE
\\
VALUE
\\
C-1538597
\\
DT:21/09/21
}
&
\makecell{
US\$
\\
1.38\$/KG
} \\
\hline
\end{longtable}
}

\begin{document}
\noindent
\begin{minipage}[t]{0.05\linewidth}
% ek
০১।
\end{minipage}
\begin{minipage}[t]{0.95\linewidth}
বি/ই রেজি: নং- {\beno}, তারিখ: {\bedt}
নথিভূক্ত করে পরবর্তী কার্যক্রমের জন্য উপস্থাপন করা হলো।
\
\\
\\
\end{minipage}
\begin{minipage}[t]{0.05\linewidth}
\hspace*{1em}
\end{minipage}
\begin{minipage}[t]{0.45\linewidth}
\hspace*{1em}
\end{minipage}
\begin{minipage}[t]{0.50\linewidth}
\textbf{এ.আর.ও}
\end{minipage}
\begin{minipage}[t]{0.05\linewidth}
% dui
০২।
\end{minipage}
\begin{minipage}[t]{0.95\linewidth}
\underline{\textbf {আমদানিকৃত পণ্য চালানের
মৌলিক তথ্য:}}
\\
\end{minipage}
\footnotesize
\begin{minipage}[t]{0.05\linewidth}
\hspace*{1em}
\end{minipage}
\begin{minipage}[t]{0.45\linewidth}
(ক) বি/ই রেজি: নং ও তারিখ
\end{minipage}
\begin{minipage}[t]{0.02\linewidth}
:
\end{minipage}
\begin{minipage}[t]{0.50\linewidth}
\textbf{{\beno}} \hspace{2em} DT: {\bedt}
\\
\end{minipage}
\begin{minipage}[t]{0.05\linewidth}
\hspace*{1em}
\end{minipage}
\begin{minipage}[t]{0.45\linewidth}
(খ) আমদানিকারকের নাম, ঠিকানা
ও BIN নম্বর
\end{minipage}
\begin{minipage}[t]{0.02\linewidth}
:
\end{minipage}
\begin{minipage}[t]{0.50\linewidth}
\textbf{{\impn}}
\\
{\impadd}
\\
BIN NO. {\impbin}
\\
\end{minipage}
\begin{minipage}[t]{0.05\linewidth}
\hspace*{1em}
\end{minipage}
\begin{minipage}[t]{0.45\linewidth}
(গ) সিএন্ডএফ এজেন্টের নাম, ঠিকানা
ও AIN নম্বর
\end{minipage}
\begin{minipage}[t]{0.02\linewidth}
:
\end{minipage}
\begin{minipage}[t]{0.50\linewidth}
\textbf{{\cnfn}}
\\
{\cnfadd}
\\
AIN NO. {\cnfain}
\\
\end{minipage}
\begin{minipage}[t]{0.05\linewidth}
\hspace*{1em}
\end{minipage}
\begin{minipage}[t]{0.45\linewidth}
(ঘ) এল/সি নং ও তারিখ
\end{minipage}
\begin{minipage}[t]{0.02\linewidth}
:
\end{minipage}
\begin{minipage}[t]{0.50\linewidth}
{\lcno} \hspace{2em} DT: {\lcdt}
\\
\end{minipage}
\begin{minipage}[t]{0.05\linewidth}
\hspace*{1em}
\end{minipage}
\begin{minipage}[t]{0.45\linewidth}
(ঙ) লিয়েন ব্যাংকের নাম
\end{minipage}
\begin{minipage}[t]{0.02\linewidth}
:
\end{minipage}
\begin{minipage}[t]{0.50\linewidth}
{\lienbank}
\\
\end{minipage}
\begin{minipage}[t]{0.05\linewidth}
\hspace*{1em}
\end{minipage}
\begin{minipage}[t]{0.45\linewidth}
(চ) এলসিএ নং ও তারিখ
\end{minipage}
\begin{minipage}[t]{0.02\linewidth}
:
\end{minipage}
\begin{minipage}[t]{0.50\linewidth}
{\lcano} \hspace{2em} DT: {\lcadt}
\\
\end{minipage}
\begin{minipage}[t]{0.05\linewidth}
\hspace*{1em}
\end{minipage}
\begin{minipage}[t]{0.45\linewidth}
(ছ) বি/এল নং ও তারিখ
\end{minipage}
\begin{minipage}[t]{0.02\linewidth}
:
\end{minipage}
\begin{minipage}[t]{0.50\linewidth}
{\blno} \hspace{2em} DT: {\bldt}
\\
\end{minipage}
\begin{minipage}[t]{0.05\linewidth}
\hspace*{1em}
\end{minipage}
\begin{minipage}[t]{0.45\linewidth}
(জ) বাণিজ্যিক ইনভয়েজ নং ও তারিখ
\end{minipage}
\begin{minipage}[t]{0.02\linewidth}
:
\end{minipage}
\begin{minipage}[t]{0.50\linewidth}
{\invno} \hspace{2em} DT: {\invdt}
\\
\end{minipage}
\begin{minipage}[t]{0.05\linewidth}
\hspace*{1em}
\end{minipage}
\begin{minipage}[t]{0.45\linewidth}
(ঝ) সিআরএফ নং ও ইস্যুর তারিখ
\end{minipage}
\begin{minipage}[t]{0.02\linewidth}
:
\end{minipage}
\begin{minipage}[t]{0.50\linewidth}
{\crf} \hspace{2em} {\crfdt}
\\
\end{minipage}
\begin{minipage}[t]{0.05\linewidth}
\hspace*{1em}
\end{minipage}
\begin{minipage}[t]{0.45\linewidth}
(ঞ) পণ্যের বিবরণ
\end{minipage}
\begin{minipage}[t]{0.02\linewidth}
:
\end{minipage}
\begin{minipage}[t]{0.50\linewidth}
{\good}
\\
\end{minipage}
\begin{minipage}[t]{0.05\linewidth}
\hspace*{1em}
\end{minipage}
\begin{minipage}[t]{0.45\linewidth}
(ট) পণ্যের পরিমাণ (একক সহ)
\end{minipage}
\begin{minipage}[t]{0.02\linewidth}
:
\end{minipage}
\begin{minipage}[t]{0.50\linewidth}
{\pkg}
\\
\end{minipage}
\begin{minipage}[t]{0.05\linewidth}
\hspace*{1em}
\end{minipage}
\begin{minipage}[t]{0.45\linewidth}
(ঠ) পণ্যের এইচ.এস.কোড
\end{minipage}
\begin{minipage}[t]{0.02\linewidth}
:
\end{minipage}
\begin{minipage}[t]{0.50\linewidth}
{\hscode}
\\
\end{minipage}
\begin{minipage}[t]{0.05\linewidth}
\hspace*{1em}
\end{minipage}
\begin{minipage}[t]{0.45\linewidth}
(ড) পণ্যের মূল্য
\end{minipage}
\begin{minipage}[t]{0.02\linewidth}
:
\end{minipage}
\begin{minipage}[t]{0.50\linewidth}
{\price}
\\
\end{minipage}
\begin{minipage}[t]{0.05\linewidth}
\hspace*{1em}
\end{minipage}
\begin{minipage}[t]{0.45\linewidth}
(ঢ) কান্ট্রি অব অরিজিন
\end{minipage}
\begin{minipage}[t]{0.02\linewidth}
:
\end{minipage}
\begin{minipage}[t]{0.50\linewidth}
{\co}
\\
\end{minipage}
\begin{minipage}[t]{0.05\linewidth}
\hspace*{1em}
\end{minipage}
\begin{minipage}[t]{0.45\linewidth}
(ণ) কান্ট্রি অব শিপমেন্ট
\end{minipage}
\begin{minipage}[t]{0.02\linewidth}
:
\end{minipage}
\begin{minipage}[t]{0.50\linewidth}
{\coship}
\\
\end{minipage}
\begin{minipage}[t]{0.05\linewidth}
\hspace*{1em}
\end{minipage}
\begin{minipage}[t]{0.45\linewidth}
(ত) জাহাজের নাম
\end{minipage}
\begin{minipage}[t]{0.02\linewidth}
:
\end{minipage}
\begin{minipage}[t]{0.50\linewidth}
{\vessel}
\end{minipage}
\begin{minipage}[t]{0.05\linewidth}
\hspace*{1em}
\end{minipage}
\begin{minipage}[t]{0.45\linewidth}
\hspace*{1.8em}পালা নং বি/এল নং
\end{minipage}
\begin{minipage}[t]{0.02\linewidth}
\hspace{1em}
\end{minipage}
\begin{minipage}[t]{0.50\linewidth}
{\rotno}
\\
\end{minipage}
\begin{minipage}[t]{0.05\linewidth}
\hspace*{1em}
\end{minipage}
\begin{minipage}[t]{0.45\linewidth}
(থ) মেনিফিস্ট নং
\end{minipage}
\begin{minipage}[t]{0.02\linewidth}
:
\end{minipage}
\begin{minipage}[t]{0.50\linewidth}
{\rotno}
\\
\end{minipage}
\normalsize
\begin{minipage}[t]{0.05\linewidth}
% tin
০৩।
\end{minipage}
\begin{minipage}[t]{0.95\linewidth}
\underline{\textbf{শুল্কায়ন সেকশনের পর্যালোচনা:}}
\end{minipage}
\begin{minipage}[t]{0.05\linewidth}
\hspace{1em}
\end{minipage}
\begin{minipage}[t]{0.05\linewidth}
% tina
(ক)
\end{minipage}
\begin{minipage}[t]{0.90\linewidth}
\underline{\textbf{আমদানি দলিল পত্র যাচাই:}}
পণ্য চালান খালাসের জন্য নিম্নবর্ণিত দলিলাদিসহ বি/ই দাখিল করা
হয়েছে যা সংশ্লিষ্ট ব্যাংকের অথরাইজ কর্মকর্তা কর্তৃক সত্যায়িত।
পরীক্ষান্তে এসব কাগজপত্র যথাযথ আছে বলে প্রতীয়মান হয়।
\\
(১) এল.সি এবং এল.সি.এ ফরম।
\\
(২) ইনভয়েস।
\\
(৩) প্যাকিং লিস্ট।
\\
(৪) বি/এল।
\\
(৫) কান্ট্রি অব অরিজিন সনদ।
\\
\end{minipage}
\begin{minipage}[t]{0.05\linewidth}
\hspace{1em}
\end{minipage}
\begin{minipage}[t]{0.05\linewidth}
% tinb
(খ)
\end{minipage}
\begin{minipage}[t]{0.90\linewidth}
\underline{\textbf{আমদানি যোগ্যতা যাচাই:}}
প্রচলিত আমদানিনীতি আদেশ পর্যালোচনায় দেখা যায় যে, পণ্যগুলি অবাধে আমদানিযোগ্য।
আলোচ্য চালানের ক্ষেত্রে আমদানিনীতি আদেশের প্রযোজ্য অন্যান্য শর্ত (কান্ট্রি অব অরিজিন, রেজিঃ
সার্টিফিকেট ইত্যাদি) প্রতিপালিত হয়েছে।
\\
\end{minipage}
\begin{minipage}[t]{0.05\linewidth}
\hspace{1em}
\end{minipage}
\begin{minipage}[t]{0.05\linewidth}
% tinc
(গ)
\end{minipage}
\begin{minipage}[t]{0.90\linewidth}
\underline{\textbf{কায়িক পরীক্ষার প্রতিবেদন পর্যালোচনা:}}
আলোচ্য পণ্য চালানটি ১০০\% কায়িক পরীক্ষার জন্য নির্বাচিত
নয়।
\\
\end{minipage}
\begin{minipage}[t]{0.05\linewidth}
\hspace{1em}
\end{minipage}
\begin{minipage}[t]{0.05\linewidth}
% tind
(ঘ)
\end{minipage}
\begin{minipage}[t]{0.90\linewidth}
\underline{\textbf{রাসায়নিক পরীক্ষা সংক্রান্ত মন্তব্য:}}
প্রযোজ্য নয়।
\\
\end{minipage}
\begin{minipage}[t]{0.05\linewidth}
\hspace{1em}
\end{minipage}
\begin{minipage}[t]{0.05\linewidth}
% tine
(ঙ)
\end{minipage}
\begin{minipage}[t]{0.90\linewidth}
\underline{\textbf{এইচ.এস.কোড সঠিকতা যাচাই:}}
ইনভয়েস প্রত্যায়িত এইচ.এস.কোড দি কাস্টমস্ এ্যাক্ট ১৯৬৯ এর FIRST SCHEDULE ও
EXPLANATORY NOTES প্রচলিত এসআরও/স্থায়ী আদেশ ইত্যাদির আলোকে পরীক্ষা করা হলো।
প্রত্যায়িত এইচ.এস.কোড যথাযথ আছে।
\\
\end{minipage}
\begin{minipage}[t]{0.05\linewidth}
\hspace{1em}
\end{minipage}
\begin{minipage}[t]{0.05\linewidth}
% tinf
(চ)
\end{minipage}
\begin{minipage}[t]{0.90\linewidth}
\underline{\textbf{রেয়াতী হার বা বিশেষ মওকুফ সংক্রান্ত:}}
\\
(১) জাতীয় রাজস্ব বোর্ড পত্র নং-০৮.০১.০০০০.০৬৮.১৮.০০৪.১৭/১৬৯(৩),
তারিখ-২৭/০৬/২০২১ খ্রিঃ এর মাধ্যমে আলোচ্য প্রতিষ্ঠান কর্তৃক উপকরণ ও যন্ত্রাংশ
আমদানির ক্ষেত্রে আমদানি পর্যায়ে আরোপনীয় সমুদয় মূল্য সংযোজন কর (আগাম কর ব্যতিত)
ও সম্পূরক শুল্ক (প্রযোজ্য ক্ষেত্রে) বোর্ডের সিদ্ধান্তক্রমে অব্যাহতি প্রদান করা হয়েছে।
\\
(২) মূসক-৪.৩ অনুযায়ি প্রতিষ্ঠানটি Refrigerator উৎপাদনকারি প্রতিষ্ঠান হওয়ায়
এসআরও-১১৪-আইন/২০২১/০৩/কাস্টমস, তারিখ: ২৪/০৫/২০২১ ইং এর
আওতায় রেয়াতী সুবিধা প্রাপ্ত।
\\
জাতীয় রাজস্ব বোর্ড পত্র এবং
এসআরও-১১৪-আইন/২০২১/০৩/কাস্টমস, তারিখ: ২৪/০৫/২০২১
এর আওতায় সিপিসি ৪০০০/৪০৯ প্রযোজ্য।
\\
\end{minipage}
\begin{minipage}[t]{0.05\linewidth}
\hspace{1em}
\end{minipage}
\begin{minipage}[t]{0.05\linewidth}
% ting
(ছ)
\end{minipage}
\begin{minipage}[t]{0.90\linewidth}
\underline{\textbf{অভিযোগ সংক্রান্ত:}} এ চালানে জাতীয় রাজস্ব বোর্ডের/গোয়েন্দা
সংস্থার অভিযোগ পাওয়া যায় নাই।
\\
\end{minipage}
\begin{minipage}[t]{0.05\linewidth}
\hspace{1em}
\end{minipage}
\begin{minipage}[t]{0.05\linewidth}
% tinh
(জ)
\end{minipage}
\begin{minipage}[t]{0.90\linewidth}
\underline{\textbf{ন্যায় নির্ণয় সংক্রান্ত:}} প্রযোজ্য নয়।
\\
\end{minipage}
\begin{minipage}[t]{0.05\linewidth}
\hspace{1em}
\end{minipage}
\begin{minipage}[t]{0.05\linewidth}
% tini
(ঝ)
\end{minipage}
\begin{minipage}[t]{0.90\linewidth}
\underline{\textbf{উল্লেখ করার মত প্রাসঙ্গিক অন্যান্য বিষয়:}} আমদানিকারক
মূসক ২.৩, ৪.৩ অন-লাইন আইআরসি কপি, ভ্যাট রিটার্ণ এবং ভ্যাট প্রত্যয়নপত্র দাখিল
করেছেন।
\\
\end{minipage}
\begin{minipage}[t]{0.05\linewidth}
% char
০৪।
\end{minipage}
\begin{minipage}[t]{0.95\linewidth}
\underline{\textbf{শুল্কায়ন সম্পর্কিত প্রস্তাব:}}
পণ্য চালান সংক্রান্ত আমদানি দলিলপত্র
ইনভয়েস পর্যালোচনা পূর্বক পণ্যের বর্ণনা, পরিমাণ, এইচ.এস.কোড, ঘোষিত মূল্য,
সমসাময়িককালের অনুরূপ পণ্যের একক মূল্যসহ নিম্নের ছকে শুল্কায়নের প্রস্তাব উপস্থাপন
করা হলো:
\end{minipage}
\scriptsize
\begin{minipage}[t]{1\linewidth}
{\taxtab}
\smallskip
\end{minipage}
\normalsize
\begin{minipage}[t]{0.05\linewidth}
% pach
০৫।
\end{minipage}
\begin{minipage}[t]{0.95\linewidth}
\underline{\textbf{শুল্কায়নযোগ্য মূল্য নিরুপন:}} ASYCUDA WORLD SYSTEM
-এ আলোচ্য পণ্য চালানের বি/ই দাখিলের পূর্ববর্তী ৯০ (নব্বই) দিনের
মূল্য তথ্য পর্যালোচনা করে হুবহু বাণিজ্যিক বর্ণনার পণ্যের যে শুল্কায়িত
মূল্য পাওয়া যায় তার সাথে সামঞ্জস্য রেখে উপরের ছকে
শুল্ক মূল্যায়ন (আমদানি পণ্যের মূল্য নির্ধারণ) বিধিমালা,
২০০০ (এসআরও নং- ৫৭/আইন/২০০০/১৮২১/শুল্ক, তারিখ: ২৩/০২/২০০০)
এর বিধি-৪ এর আওতায় ঘোষিত মূল্যে শুল্কায়ন প্রস্তাব রাখা হলো।
\\
\end{minipage}
\begin{minipage}[t]{0.05\linewidth}
% choi
০৬।
\end{minipage}
\begin{minipage}[t]{0.95\linewidth}
এমতাবস্থায়, নোট অনুচ্ছেদ-৪ এর ছকে
উল্লেখিত প্রস্তাবিত HSCODE ও মূল্যের
আলোকে শুল্কায়নের অনুমোদন দেয়া যেতে পারে।
\\
\\
সদয় অবগতি ও আদেশার্থে উপস্থাপন করা হলো।

\end{minipage}

\thispagestyle{laststyle}

\end{document}
