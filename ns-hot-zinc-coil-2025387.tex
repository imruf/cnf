\documentclass[12pt]{article}
\usepackage[legalpaper,
            lmargin=1in,rmargin=0.5in,
            bmargin=0.5in,tmargin=1in]{geometry}
\usepackage{fontspec}
\usepackage{titlesec}
\usepackage{multirow}
\usepackage[colorlinks=true,urlcolor=Blue]{hyperref}
\usepackage{graphicx}
\usepackage{array}
\usepackage{makecell}
\usepackage{fancyhdr}
\usepackage[none]{hyphenat}
\usepackage{longtable}
\usepackage[dvipsnames]{xcolor}
\usepackage[banglamainfont=Kalpurush,
            banglattfont=SolaimanLipi,
            % feature=1,
            % changecounternumbering=0
           ]{latexbangla}
% oncehsis imp file name
\newcommand{\eif}{নথি নং- এস \hspace{4em}/স্টাফ/কাস/\hspace{5em}/অংশ-}
\newcommand{\jdlf}{নথি নং- এস৫-২৪৫/স্টাফ/কাস/০৬-০৭/অংশ-২}
\newcommand{\jdwlf}{নথি নং- এস৫-১৭০/স্টাফ/কাস/১৪-১৫/অংশ-২}
\newcommand{\jealf}{নথি নং- এস-৫-৫৮৫/স্টাফ/কাস/২০১৩-২০১৪/অংশ-১২}
\newcommand{\jfclf}{নথি নং- এস৫-৫১৯/স্টাফ/কাস/২০-২১}
\newcommand{\jsmlf}{নথি নং- এস৫-৬৯১/স্টাফ/কাস/২০১৭-২০১৮/অংশ-০৩}
\newcommand{\jtrif}{নথি নং- এস-৫-১২১০/স্টাফ/কাস/১৬-১৭/অংশ-১}
\newcommand{\jwelf}{নথি নং- এস-৫-১১৪/স্টাফ/কাস/২০১৮-১৯/}
\newcommand{\scmlf}{নথি নং- এস-৫-১৬৩/স্টাফ/কাস/০৬-০৭/অংশ-৫}
\newcommand{\ssmlf}{নথি নং- এস-৫-১৪৪/স্টাফ/কাস/০৬-০৭/অংশ-১১}
\newcommand{\tdjf}{নথি নং- এস-৫-৪৪৩/স্টাফ/কাস/০৬-০৭/অংশ-}
\newcommand{\htff}{নথি নং- এস-৫-৬৮৪/স্টাফ/কাস/২০১৭-১৮/অংশ-৩}

% fileno
\newcommand{\filenou}{
\begin{minipage}[t]{0.60\linewidth}
\hspace{1em}
\end{minipage}
\begin{minipage}[t]{0.40\linewidth}
\underline{\fileno}
\end{minipage}
}


% new page
\newcommand{\nfpage}{
\newpage
\small
{\filenou}
}

% imp name
\newcommand{\ssml}{SHAMEEM SPINNING MILLS LTD.}
\newcommand{\ssmla}{SHAFIPUR, KALIAKAIR, GAZIPUR-1751, BANGLADESH.}
\newcommand{\scml}{SHAMEEM COMPOSITE MILLS LTD.}
\newcommand{\scmla}{SHAFIPUR, KALIAKAIR, GAZIPUR-1751, BANGLADESH}
\newcommand{\jsml}{JAMUNA SPINNING MILLS LTD.}
\newcommand{\jsmla}{SHAFIPUR, KALIAKAIR, GAZIPUR-1751, BANGLADESH.}
\newcommand{\jsmlaut}{JAMUNA SPINNING MILLS LTD. UNIT-2}
\newcommand{\jsmlauta}{BEJURA, SOUTH BEJURA, MADHABPUR, HOBIGONJ-3331, BANGLADESH.}
\newcommand{\jdl}{JAMUNA DENIMS LTD.}
\newcommand{\jdla}{SHAFIPUR, KALIAKAIR, GAZIPUR-1751, BANGLADESH.}
\newcommand{\jeal}{JAMUNA ELECTRONICS \& AUTOMOBILES LTD.}
\newcommand{\jeala}{SINABHA, KALIAKAIR PS, GAZIPUR-1750, BANGLADESH}
\newcommand{\jdwl}{JAMUNA DENIMS WEAVING LTD.}
\newcommand{\jdwla}{KASHIMPUR ROAD, JARUN, KONABARI PS, GAZIPUR-1751, BD}
\newcommand{\jfcl}{JAMUNA FAN AND CABLES LTD.}
\newcommand{\jfcla}{KASHIMPUR ROAD, JARUN, KONABARI PS, GAZIPUR-1751, BD}
\newcommand{\htf}{HOORAIN HTF LTD.}
\newcommand{\htfa}{BEJURA, SOUTH BEJURA, MADHABPUR, HOBIGONJ-3331, BANGLADESH.}
\newcommand{\tdj}{M/S THE DAILY JUGANTOR}
\newcommand{\tdja}{KA-244, KURIL, PROGATI SARANI, VATARA PS, DHAKA-1229, BANGLADESH}
\newcommand{\jtri}{JAMUNA TYRE AND RUBBER INDUSTRIES}
\newcommand{\jtria}{BEJURA, SOUTH BEJURA, MADHABPUR, HOBIGONJ}
\newcommand{\jwel}{JAMUNA WELDING ELECTRODE LTD.}
\newcommand{\jwela}{CHOYDANA, HAZIRPUKUR; GAZIPUR SADAR, GAZIPUR-1704; BANGLADESH}
\newcommand{\cbl}{CROWN BEVERAGE LIMITED}
\newcommand{\cbla}{SHAFIPUR, KALIAKAIR, GAZIPUR-1751, BANGLADESH}
\newcommand{\jpml}{JAMUNA PAPER MILLS LIMITED}
\newcommand{\jpmla}{BEJURA, SOUTH BEJURA MADHABPUR, HOBIGONJ, BANGLADESH}

% imp bin
\newcommand{\jealbin}{000146478-0103}
\newcommand{\tdjbin}{001960365-0101}
\newcommand{\jsmlbin}{000144425-0103}
\newcommand{\jfclbin}{001753334-0103}
\newcommand{\scmlbin}{000151754-0103}
\newcommand{\htfbin}{000146478-0103}
\newcommand{\cblbin}{000149506-0103}

% cnf name
\newcommand{\cnfn}{SHAMEEM SPINNING MILLS LTD.}
\newcommand{\cnfadd}{92, HIGH LEVEL ROAD, LALKHAN BAZAR, CHITTAGONG}
\newcommand{\cnfain}{301 08 3417}

% sros
\newcommand{\srofs}{এসআরও নং- ৫৭- আইন/২০০০/১৮২১/শুল্ক}
\newcommand{\srofsd}{তারিখ: ২৩/০২/২০০০ খ্রি:}
\newcommand{\srooot}{এসআরও নং- ১১৩- আইন/২০২১/০২/কাস্টমস}
\newcommand{\sroootd}{তারিখ: ২৪/০৫/২০২১ খ্রি:}
\newcommand{\srooof}{এসআরও নং- ১১৪- আইন/২০২১/০৩/কাস্টমস}
\newcommand{\srooofd}{তারিখ: ২৪/০৫/২০২১ খ্রি:}
\newcommand{\srootz}{এসআরও নং- ১২০- আইন/২০২১/০৯/কাস্টমস}
\newcommand{\srootzd}{তারিখ: ২৪/০৫/২০২১ খ্রিঃ}

\newcommand{\sroott}{এসআরও নং- ১৩৩- আইন/২০২১/২২/কাস্টমস}
\newcommand{\sroottd}{তারিখ: ২৪/০৫/২০২১ খ্রিঃ}

\newcommand{\srotsz}{এসআরও নং- ৩৬০- আইন/২০১৩/২৪৬৮/কাস্টমস}
\newcommand{\srotszd}{তারিখ: ২৫/১১/২০১৩ খ্রি:}

% cpcs
\newcommand{\cpcofs}{CPC- 4000/157}
\newcommand{\cpcost}{CPC- 4000/173}
\newcommand{\cpcttz}{CPC- 4000/220}
\newcommand{\cpcfzo}{CPC- 4000/401}
\newcommand{\cpcfzn}{CPC- 4000/409}
\newcommand{\cpcsso}{CPC- 4000/661}

% nbrl
\newcommand{\nbrosn}{জাতীয় রাজস্ব বোর্ডের পত্র নং- ০৮.০১.০০০০.০৬৮.১৮.০০৪.১৭/১৬৯}
\newcommand{\nbrosnd}{তারিখ: ২৭/০৬/২০২১ ইং}
\newcommand{\nbrosnt}{জাতীয় রাজস্ব বোর্ডের পত্র নং- ০৮.০১.০০০০.০৬৮.১৮.০০৪.১৭/১৬৯(৩)}
\newcommand{\nbrosntd}{তারিখ: ২৭/০৬/২০২১ খ্রি:}
\newcommand{\nbrfs}{জাতীয় রাজস্ব বোর্ডের পত্র নং- ০৮.০১.০০০০.০৩৪.০২.৩০১.১৯-৪৭}
\newcommand{\nbrfsd}{তারিখ: ১২/০৭/২০২১ খ্রি:}

% imp reg name
\newcommand{\ssmlreg}{২৩১৩৯৫০৬১৮০-এইচ, তারিখ: ০৯/০১/২০১৩}
\newcommand{\scmlreg}{৯৭০৪০১৭-এইচ তারিখ: ২১/০৪/১৯৯৭}
\newcommand{\jealreg}{এল-২৯৩০১৩০১৩৪৩১-এইচ, তারিখ: ০৯/০১/২০১৩}
\newcommand{\eireg}{\hspace{10em} তারিখ: \hspace{5em}}

% imp irc no
\newcommand{\jealirc}{260326120426720}
\newcommand{\scmlirc}{260326120041719}
\newcommand{\jfclirc}{260326120515020}

% san name
\newcommand{\maersk}{MAERSK BANGLADESH LIMITED}
\newcommand{\maerska}{58, AGRABAD COMMERCIAL AREA (3RD FLOOR), CHITTAGONG, 4100.}
\newcommand{\apl}{APL (BANGLADESH) PVT.LTD}
\newcommand{\apla}{PLOT NO. 30, 3RD FLOOR OF SURAIYA MANSION, AGRABAD, CHITTAGONG.}
\newcommand{\baridhi}{BARIDHI SHIPPING LINES LTD}
\newcommand{\baridhia}{3/F HRC BHABAN, 64-66 AGRABAD C/A, CHITTAGONG.}
\newcommand{\continentalbd}{CONTINENTAL TRADERS (BD) LIMITED}
\newcommand{\continentalbda}{73, AGRABAD C/A, CHITTAGONG.}
\newcommand{\continentaltr}{CONTINENTAL TRADERS (BD) LIMITED}
\newcommand{\continentaltra}{IQBAL BHABAN, AGRABAD C/A, CHITTAGONG.}
\newcommand{\gbx}{GBX LOGISTICS LTD}
\newcommand{\gbxa}{AYUB TRADE CENTER(1ST FLOOR), 1269/B, SK MUJIB ROAD, AGRABAD C/A, CHITTAGONG.}
\newcommand{\transmarine}{TRANSMARINE LOGISTICS LTD}
\newcommand{\transmarinea}{B.M.HEIGHTS(4TH FLOOR), 318, SK, MUJIB ROAD, AGRABAD C/A, CHITTAGONG.}
\newcommand{\trident}{TRIDENT SHIPPING LINE LTD}
\newcommand{\tridenta}{AKHTARUZZAMAN CENTER, 6TH FLOOR 21/22 AGRABAD, CHITTAGONG.}
\newcommand{\msc}{MSC MEDITERRANEAN SHIPP.CO.BD.LTD}
\newcommand{\msca}{IIUC TOWER, 4TH FLOOR, 1700/A SK.MUJIB ROAD, PLOT-09, AGRABAD, CHITTAGONG.}
\newcommand{\alviline}{ALVILINE BANGLADESH LIMITED}
\newcommand{\alvilinea}{78, AGRABAD C/A, MACCA MADINA TRADE CENTER, 9TH FLOOR, CHITTAGONG}
\newcommand{\ocean}{OCEAN NETWORK EXPRESS (BD) LTD}
\newcommand{\oceana}{IIUC TOWER (10TH FLOOR), 1700/A, PLOT-9, SK.MUJIB ROAD, AGRABAD C/A, CHITTAGONG}
\newcommand{\vega}{VEGA MARINE PVT LIMITED}
\newcommand{\vegaa}{DAAR-E SHAHIDI, 4TH FLOOR, 69 AGRABAD C/A, CTG}
\newcommand{\mega}{MEGATREND SHIPPING LINES LTD.}
\newcommand{\megaa}{MAKKAH MADINAH TRADE CENTER (16TH FLOOR), 78, AGRABAD, CHITTAGONG}
\newcommand{\famfa}{FAMFA SOLUTION LIMITED}
\newcommand{\famfaa}{BONANI, AGRABAD, CHITTAGONG}
\newcommand{\reliance}{RELIANCE SHIPPING SERVICES}
\newcommand{\reliancea}{34 AGRABAD C/A, CHITTAGONG 4100, BANGLADESH}
%\newcommand{\alvilinea}{}


% mujib logo
\newcommand{\my}{\includegraphics[height=3.2em]{pic/my.png}}

% slogan
\fancypagestyle{slogan}
{
\fancyhf{}
\renewcommand{\headrulewidth}{0pt}
% header
\lhead{
\framebox[1.1\width]{\footnotesize{``উন্নয়নের অক্সিজেন রাজস্ব''}}
}
\rhead{
\my
\\
\framebox[1.1\width]{\footnotesize{``জনকল্যানে রাজস্ব''}}
}
}

% customs
\newcommand{\tca}{The Customs Act, 1969}





\pagestyle{fancy}
\fancyhf{}
\renewcommand{\headrulewidth}{0pt}
% header
\chead{
\underline{{পৃষ্ঠা - \thepage}}
\\
}
\rhead{{\nfilenou}}
% footer
%\rfoot{চলমান পৃষ্ঠা-\thepage}

\fancypagestyle{laststyle}
{
   \fancyfoot[R]{}
}

\newcommand{\fileno}{নথি নং - ১০৬৯/এপি/সেকশন-৮(এ)/২০২১-২০২২}
\newcommand{\nfilenou}{\underline{\fileno}}

\newcommand{\product}{Refrigerator}
\newcommand{\good}{HOT ZINC COIL ETC.}
\newcommand{\pkg}{58 PKG=211849 KG}
\newcommand{\hscode}{7210.61.10, 7209.15.00, 7208.26.90, 7210.49.10}
\newcommand{\price}{US\$ 235,923.05}

\newcommand{\co}{CHINA}
\newcommand{\coship}{CHINA}

\newcommand{\vessel}{MARINE BIA}

\newcommand{\blno}{SHWWSE211118755}
\newcommand{\bldt}{17.11.21}

\newcommand{\beno}{C-2025387}
\newcommand{\bedt}{14.12.2021}
\newcommand{\menifest}{2021/5792}

\newcommand{\lcno}{0000088821020119}
\newcommand{\lcdt}{28.10.21}
\newcommand{\lcano}{363964}
\newcommand{\lcadt}{\lcdt}
\newcommand{\lienbank}{ISLAMI BANK BANGLADESH LIMITED}

\newcommand{\invno}{RFD20211014}
\newcommand{\invdt}{12.12.2021}

\newcommand{\impn}{\jeal}
\newcommand{\impadd}{\jeala}
\newcommand{\impbin}{\jealbin}
\newcommand{\impoldbin}{}

\newcommand{\crf}{NON CRF}
\newcommand{\crfdt}{}

\newcommand{\ircno}{\jealirc}
\newcommand{\ircrenewdt}{৩০/০৬/২০২২ ইং}
\newcommand{\musokr}{নভেম্বর-২১}

\newcommand{\taxtab}{
\begin{longtable}{|c|c|c|c|c|c|c|c|}
\hline
\textbf{
\makecell{
ক্রঃ \\ নং
}
}
&
\textbf{
\makecell{
পণ্যের বর্ণনা
}
}
&
\textbf{
\makecell{
পরিমাণ
}
}
&
\textbf{
\makecell{
ইনভয়েস
\\
ঘোষিত
\\
এইচএসকোড
\\
ও শুল্কহার
}
}
&
\textbf{
\makecell{
প্রকৃত
\\
এইচএসকোড
\\
ও শুল্কহার
}
}
&
\textbf{
\makecell{
ইনভয়েস
\\
প্রত্যায়িত
\\
একক মূল্য
\\
(US\$)
}
}
&
\textbf{
\makecell{
সাময়িক
\\
কালের
\\
অভিন্ন/অনুরুপ
\\
পণ্যের
\\
একক মূল্য
(US\$)
}
}
&
\textbf{
\makecell{
প্রস্তাবিত
\\
একক মূল্য
\\
(US\$)
}
} \\
\hline
% row
\makecell{
01
}
&
\makecell{
HOT ZINC COIL
\\
($0.30^*1200\textrm{MM}$)
\\
C/O. {\co}
}
&
\makecell{
17,410.00
\\
KGS
}
&
\makecell{
7210.61.10
\\
CD-10\%
\\
AIT-0.83\%
\\
CPC 4000/409
\\
NBR LETTER
\\
SRO 114/21
}
&
\makecell{
7210.61.10
\\
CD-10\%
\\
AIT-0.83\%
\\
CPC 4000/409
\\
NBR LETTER
\\
SRO 114/21
}
&
\makecell{
US\$
\\
1.12/KG
}
&
\makecell{
US\$
\\
1.11/KG
\\
DATA BASE
VALUE
\\
C-1724230
\\
DT:26.10.21
}
&
\makecell{
US\$
\\
1.12/KG
} \\
\hline
% row
\makecell{
02
}
&
\makecell{
COLD ROOLED SHEET
\\
($3.0^*1250^*2500$)
\\
C/O. {\co}
}
&
\makecell{
14,204.00
\\
KGS
}
&
\makecell{
7209.15.00
\\
CD-10\%
\\
AIT-0.83\%
\\
CPC 4000/661
\\
NBR LETTER
}
&
\makecell{
7209.15.00
\\
CD-10\%
\\
AIT-0.83\%
\\
CPC 4000/661
\\
NBR LETTER
}
&
\makecell{
US\$
\\
1.23/KG
}
&
\makecell{
US\$
\\
S/Q- 0.50/KG
\\
P/Q- 0.60/KG
\\
SECTION
VALUE
}
&
\makecell{
US\$
\\
1.23/KG
} \\
\hline
% row
\makecell{
03
}
&
\makecell{
PICKING PLATE
\\
($4^*1250^*2500\textrm{MM}$)
\\
C/O. {\co}
}
&
\makecell{
24,860.00
\\
KGS
}
&
\makecell{
7208.26.90
\\
CD-10\%
\\
AIT-0.83\%
\\
CPC 4000/661
\\
NBR LETTER
}
&
\makecell{
7208.26.90
\\
CD-10\%
\\
AIT-0.83\%
\\
CPC 4000/661
\\
NBR LETTER
}
&
\makecell{
US\$
\\
1.10/KG
}
&
\makecell{
US\$
\\
S/Q- 0.40/KG
\\
P/Q- 0.60/KG
\\
SECTION
VALUE
}
&
\makecell{
US\$
\\
1.10/KG
} \\
\hline
% row
\makecell{
04
}
&
\makecell{
HOT ZINC COIL
\\
($0.30^*1100\textrm{MM}$)
\\
C/O. {\co}
}
&
\makecell{
69,870.00
\\
KGS
}
&
\makecell{
7210.61.10
\\
CD-10\%
\\
AIT-0.83\%
\\
CPC 4000/409
\\
NBR LETTER
\\
SRO 114/21
}
&
\makecell{
7210.61.10
\\
CD-10\%
\\
AIT-0.83\%
\\
CPC 4000/409
\\
NBR LETTER
\\
SRO 114/21
US\$
\\
1.14/KG
}
&
\makecell{
US\$
\\
1.14/KG
}
&
\makecell{
US\$
\\
1.11/KG
\\
DATA BASE
VALUE
\\
C-1724230
\\
DT:26.10.21
}
&
\makecell{
US\$
\\
1.14/KG
} \\
\hline
% row
\makecell{
05
}
&
\makecell{
HOT ZINC COIL
\\
($1.00^*1250\textrm{MM}$)
\\
C/O. {\co}
}
&
\makecell{
56,325.00
\\
KGS
}
&
\makecell{
7210.49.10
\\
CD-10\%
\\
AIT-0.83\%
\\
CPC 4000/409
\\
NBR LETTER
\\
SRO 114/21
}
&
\makecell{
7210.49.10
\\
CD-10\%
\\
AIT-0.83\%
\\
CPC 4000/409
\\
NBR LETTER
\\
SRO 114/21
}
&
\makecell{
US\$
\\
1.08/KG
}
&
\makecell{
US\$
\\
S/Q-0.60/KG
\\
P/Q-0.85/KG
\\
SECTION
VALUE
}
&
\makecell{
US\$
\\
1.08/KG
} \\
\hline
% row
\makecell{
06
}
&
\makecell{
HOT ZINC COIL
\\
($1.2^*1250\textrm{MM}$)
\\
C/O. {\co}
}
&
\makecell{
22,150.00
\\
KGS
}
&
\makecell{
7210.49.10
\\
CD-10\%
\\
AIT-0.83\%
\\
CPC 4000/409
\\
NBR LETTER
\\
SRO 114/21
}
&
\makecell{
7210.49.10
\\
CD-10\%
\\
AIT-0.83\%
\\
CPC 4000/409
\\
NBR LETTER
\\
SRO 114/21
}
&
\makecell{
US\$
\\
1.08/KG
}
&
\makecell{
US\$
\\
S/Q-0.60/KG
\\
P/Q-0.85/KG
\\
SECTION
VALUE
}
&
\makecell{
US\$
\\
1.08/KG
} \\
\hline
% row
\makecell{
07
}
&
\makecell{
HOT ZINC COIL
\\
($2.0^*1250\textrm{MM}^*25000 \textrm{MM}$)
\\
C/O. {\co}
}
&
\makecell{
7,030.00
\\
KGS
}
&
\makecell{
7210.49.10
\\
CD-10\%
\\
AIT-0.83\%
\\
CPC 4000/409
\\
NBR LETTER
\\
SRO 114/21
}
&
\makecell{
7210.49.10
\\
CD-10\%
\\
AIT-0.83\%
\\
CPC 4000/409
\\
NBR LETTER
\\
SRO 114/21
}
&
\makecell{
US\$
\\
1.08/KG
}
&
\makecell{
US\$
\\
S/Q-0.60/KG
\\
P/Q-0.85/KG
\\
SECTION
VALUE
}
&
\makecell{
US\$
\\
1.08/KG
} \\
\hline
\end{longtable}
}

\begin{document}
\noindent
\begin{minipage}[t]{0.05\linewidth}
% ek
০১।
\end{minipage}
\begin{minipage}[t]{0.95\linewidth}
বি/ই রেজি: নং- {\beno}, তারিখ: {\bedt}
নথিভূক্ত করে
পরবর্তী কার্যক্রমের জন্য উপস্থাপন করা হলো।
\\
\\
\end{minipage}
\begin{minipage}[t]{0.05\linewidth}
\hspace*{0em}
\end{minipage}
\begin{minipage}[t]{0.05\linewidth}
সহকারী
\end{minipage}
\begin{minipage}[t]{0.37\linewidth}
\hspace{0em}
\end{minipage}
\begin{minipage}[t]{0.53\linewidth}
\textbf{শুল্কায়ন কর্মকর্তা}
\\
\end{minipage}
\begin{minipage}[t]{0.05\linewidth}
% dui
০২।
\end{minipage}
\begin{minipage}[t]{0.95\linewidth}
\underline{\textbf {আমদানিকৃত পণ্য চালানের
মৌলিক তথ্য:}}
\\
\end{minipage}
\footnotesize
\begin{minipage}[t]{0.05\linewidth}
\hspace*{1em}
\end{minipage}
\begin{minipage}[t]{0.40\linewidth}
(ক) বি/ই রেজি: নং ও তারিখ
\end{minipage}
\begin{minipage}[t]{0.02\linewidth}
:
\end{minipage}
\begin{minipage}[t]{0.53\linewidth}
\textbf{{\beno}} \hspace{2em} DT: {\bedt}
\\
\end{minipage}
\begin{minipage}[t]{0.05\linewidth}
\hspace*{1em}
\end{minipage}
\begin{minipage}[t]{0.40\linewidth}
(খ) আমদানিকারকের নাম, ঠিকানা
ও BIN নম্বর
\end{minipage}
\begin{minipage}[t]{0.02\linewidth}
:
\end{minipage}
\begin{minipage}[t]{0.53\linewidth}
\textbf{{\impn}}
\\
{\impadd}
\\
BIN NO. {\impbin}
\\
\end{minipage}
\begin{minipage}[t]{0.05\linewidth}
\hspace*{1em}
\end{minipage}
\begin{minipage}[t]{0.40\linewidth}
(গ) সিএন্ডএফ এজেন্টের নাম, ঠিকানা
ও AIN নম্বর
\end{minipage}
\begin{minipage}[t]{0.02\linewidth}
:
\end{minipage}
\begin{minipage}[t]{0.53\linewidth}
\textbf{{\cnfn}}
\\
{\cnfadd}
\\
AIN NO. {\cnfain}
\\
\end{minipage}
\begin{minipage}[t]{0.05\linewidth}
\hspace*{1em}
\end{minipage}
\begin{minipage}[t]{0.40\linewidth}
(ঘ) এল/সি নং ও তারিখ
\end{minipage}
\begin{minipage}[t]{0.02\linewidth}
:
\end{minipage}
\begin{minipage}[t]{0.53\linewidth}
{\lcno} \hspace{2em} DT: {\lcdt}
\\
\end{minipage}
\begin{minipage}[t]{0.05\linewidth}
\hspace*{1em}
\end{minipage}
\begin{minipage}[t]{0.40\linewidth}
(ঙ) লিয়েন ব্যাংকের নাম
\end{minipage}
\begin{minipage}[t]{0.02\linewidth}
:
\end{minipage}
\begin{minipage}[t]{0.53\linewidth}
{\lienbank}
\\
\end{minipage}
\begin{minipage}[t]{0.05\linewidth}
\hspace*{1em}
\end{minipage}
\begin{minipage}[t]{0.40\linewidth}
(চ) এলসিএ নং ও তারিখ
\end{minipage}
\begin{minipage}[t]{0.02\linewidth}
:
\end{minipage}
\begin{minipage}[t]{0.53\linewidth}
{\lcano} \hspace{2em} DT: {\lcadt}
\\
\end{minipage}
\begin{minipage}[t]{0.05\linewidth}
\hspace*{1em}
\end{minipage}
\begin{minipage}[t]{0.40\linewidth}
(ছ) বি/এল নং ও তারিখ
\end{minipage}
\begin{minipage}[t]{0.02\linewidth}
:
\end{minipage}
\begin{minipage}[t]{0.53\linewidth}
{\blno} \hspace{2em} DT: {\bldt}
\\
\end{minipage}
\begin{minipage}[t]{0.05\linewidth}
\hspace*{1em}
\end{minipage}
\begin{minipage}[t]{0.40\linewidth}
(জ) বাণিজ্যিক ইনভয়েস নং ও তারিখ
\end{minipage}
\begin{minipage}[t]{0.02\linewidth}
:
\end{minipage}
\begin{minipage}[t]{0.53\linewidth}
{\invno} \hspace{2em} DT: {\invdt}
\\
\end{minipage}
\begin{minipage}[t]{0.05\linewidth}
\hspace*{1em}
\end{minipage}
\begin{minipage}[t]{0.40\linewidth}
(ঝ) সিআরএফ নং ও ইস্যুর তারিখ
\end{minipage}
\begin{minipage}[t]{0.02\linewidth}
:
\end{minipage}
\begin{minipage}[t]{0.53\linewidth}
{\crf} \hspace{2em} {\crfdt}
\\
\end{minipage}
\begin{minipage}[t]{0.05\linewidth}
\hspace*{1em}
\end{minipage}
\begin{minipage}[t]{0.40\linewidth}
(ঞ) পণ্যের বিবরণ
\end{minipage}
\begin{minipage}[t]{0.02\linewidth}
:
\end{minipage}
\begin{minipage}[t]{0.53\linewidth}
{\good}
\\
\end{minipage}
\begin{minipage}[t]{0.05\linewidth}
\hspace*{1em}
\end{minipage}
\begin{minipage}[t]{0.40\linewidth}
(ট) পণ্যের পরিমাণ (একক সহ)
\end{minipage}
\begin{minipage}[t]{0.02\linewidth}
:
\end{minipage}
\begin{minipage}[t]{0.53\linewidth}
{\pkg}
\\
\end{minipage}
\begin{minipage}[t]{0.05\linewidth}
\hspace*{1em}
\end{minipage}
\begin{minipage}[t]{0.40\linewidth}
(ঠ) পণ্যের এইচ.এস.কোড
\end{minipage}
\begin{minipage}[t]{0.02\linewidth}
:
\end{minipage}
\begin{minipage}[t]{0.53\linewidth}
{\hscode}
\\
\end{minipage}
\begin{minipage}[t]{0.05\linewidth}
\hspace*{1em}
\end{minipage}
\begin{minipage}[t]{0.40\linewidth}
(ড) পণ্যের মূল্য (ইনভয়েস অনুযায়ী)
\end{minipage}
\begin{minipage}[t]{0.02\linewidth}
:
\end{minipage}
\begin{minipage}[t]{0.53\linewidth}
{\price}
\\
\end{minipage}
\begin{minipage}[t]{0.05\linewidth}
\hspace*{1em}
\end{minipage}
\begin{minipage}[t]{0.40\linewidth}
(ঢ) কান্ট্রি অব অরিজিন
\end{minipage}
\begin{minipage}[t]{0.02\linewidth}
:
\end{minipage}
\begin{minipage}[t]{0.53\linewidth}
{\co}
\\
\end{minipage}
\begin{minipage}[t]{0.05\linewidth}
\hspace*{1em}
\end{minipage}
\begin{minipage}[t]{0.40\linewidth}
(ণ) কান্ট্রি অব শিপমেন্ট
\end{minipage}
\begin{minipage}[t]{0.02\linewidth}
:
\end{minipage}
\begin{minipage}[t]{0.53\linewidth}
{\coship}
\\
\end{minipage}
\begin{minipage}[t]{0.05\linewidth}
\hspace*{1em}
\end{minipage}
\begin{minipage}[t]{0.40\linewidth}
(ত) জাহাজের নাম
\end{minipage}
\begin{minipage}[t]{0.02\linewidth}
:
\end{minipage}
\begin{minipage}[t]{0.53\linewidth}
{\vessel}
\end{minipage}
\begin{minipage}[t]{0.05\linewidth}
\hspace*{1em}
\end{minipage}
\begin{minipage}[t]{0.40\linewidth}
\hspace*{1.8em}পালা নং বি/এল নং
\end{minipage}
\begin{minipage}[t]{0.02\linewidth}
\hspace{1em}
\end{minipage}
\begin{minipage}[t]{0.53\linewidth}
{\menifest}, B/L {\blno}
\\
\end{minipage}
\begin{minipage}[t]{0.05\linewidth}
\hspace*{1em}
\end{minipage}
\begin{minipage}[t]{0.40\linewidth}
(থ) মেনিফিস্ট নং
\end{minipage}
\begin{minipage}[t]{0.02\linewidth}
:
\end{minipage}
\begin{minipage}[t]{0.53\linewidth}
{\menifest}
\\
\end{minipage}
\normalsize
\begin{minipage}[t]{0.05\linewidth}
% tin
০৩।
\end{minipage}
\begin{minipage}[t]{0.95\linewidth}
\underline{\textbf{শুল্কায়ন সেকশনের পর্যালোচনা:}}
\end{minipage}
\begin{minipage}[t]{0.05\linewidth}
\hspace{1em}
\end{minipage}
\begin{minipage}[t]{0.05\linewidth}
% tin
(ক)
\end{minipage}
\begin{minipage}[t]{0.90\linewidth}
\underline{\textbf{আমদানি দলিল পত্র যাচাই:}}
পণ্য চালান খালাসের জন্য নিম্নবর্ণিত দলিলাদিসহ বি/ই দাখিল করা
হয়েছে।
\\
(১) এল.সি এবং এল.সি.এ ফরম।
\\
(২) ইনভয়েস।
\\
(৩) প্যাকিং লিস্ট।
\\
(৪) বি/এল।
\\
(৫) কান্ট্রি অব অরিজিন সনদ।
\\
দলিলাদি পর্যালোচনায় এগুলো
সঠিক পাওয়া যায়।
\\
\end{minipage}
\begin{minipage}[t]{0.05\linewidth}
\hspace{1em}
\end{minipage}
\begin{minipage}[t]{0.05\linewidth}
% tin
(খ)
\end{minipage}
\begin{minipage}[t]{0.90\linewidth}
\underline{\textbf{আমদানি যোগ্যতা যাচাই:}}
প্রচলিত আমদানিনীতি আদেশ ২০১৫-২০১৮  পর্যালোচনা করে দেখা যায় যে, পণ্যগুলি অবাধে আমদানিযোগ্য।
আলোচ্য চালানের ক্ষেত্রে আমদানিনীতি আদেশের প্রযোজ্য অন্যান্য শর্ত (কান্ট্রি অব অরিজিন, রেজিঃ
সার্টিফিকেট ইত্যাদি) প্রতিপালিত হয়েছে।
\\
\end{minipage}
\begin{minipage}[t]{0.05\linewidth}
\hspace{1em}
\end{minipage}
\begin{minipage}[t]{0.05\linewidth}
% tin
(গ)
\end{minipage}
\begin{minipage}[t]{0.90\linewidth}
\underline{\textbf{কায়িক পরীক্ষার প্রতিবেদন পর্যালোচনা:}}
আলোচ্য পণ্যচালানটির ক্ষেত্রে প্রযোজ্য নয়।
\\
\end{minipage}
\begin{minipage}[t]{0.05\linewidth}
\hspace{1em}
\end{minipage}
\begin{minipage}[t]{0.05\linewidth}
% tin
(ঘ)
\end{minipage}
\begin{minipage}[t]{0.90\linewidth}
\underline{\textbf{রাসায়নিক পরীক্ষা সংক্রান্ত মন্তব্য:}}
আলোচ্য পণ্যচালানটির ক্ষেত্রে প্রযোজ্য নয়।
\\
\end{minipage}
\begin{minipage}[t]{0.05\linewidth}
\hspace{1em}
\end{minipage}
\begin{minipage}[t]{0.05\linewidth}
% tin
(ঙ)
\end{minipage}
\begin{minipage}[t]{0.90\linewidth}
\underline{\textbf{এইচ.এস.কোড সঠিকতা যাচাই:}}
আমদানিকারক কর্তৃক ঘোষিত এইচ.এস.কোড দি কাস্টমস্ এ্যাক্ট ১৯৬৯ এর FIRST SCHEDULE ও
EXPLANATORY NOTES প্রচলিত এসআরও/স্থায়ী আদেশ ইত্যাদির আলোকে পরীক্ষা করা হলো।
প্রত্যায়িত এইচ.এস.কোড যথাযথ আছে।
\\
\end{minipage}
\newpage
\noindent
\begin{minipage}[t]{0.05\linewidth}
\hspace{1em}
\end{minipage}
\begin{minipage}[t]{0.05\linewidth}
% tin
(চ)
\end{minipage}
\begin{minipage}[t]{0.90\linewidth}
\underline{\textbf{রেয়াতী হার বা বিশেষ মওকুফ সংক্রান্ত মন্তব্য:}}
\end{minipage}
%\footnotesize
\begin{minipage}[t]{0.1\linewidth}
\hspace{1em}
\end{minipage}
\begin{minipage}[t]{0.05\linewidth}
% tin
(১)
\end{minipage}
\begin{minipage}[t]{0.85\linewidth}
{\nbrosnt}, {\nbrosnd}।
\end{minipage}
\begin{minipage}[t]{0.1\linewidth}
\hspace{1em}
\end{minipage}
\begin{minipage}[t]{0.05\linewidth}
% tin
(২)
\end{minipage}
\begin{minipage}[t]{0.85\linewidth}
{\nbrfs}, {\nbrfsd}।
\end{minipage}
\begin{minipage}[t]{0.1\linewidth}
\hspace{1em}
\end{minipage}
\begin{minipage}[t]{0.05\linewidth}
% tin
(৩)
\end{minipage}
\begin{minipage}[t]{0.85\linewidth}
{\srooof}, {\srooofd}।
\\
\end{minipage}
\normalsize
\begin{minipage}[t]{0.05\linewidth}
\hspace{1em}
\end{minipage}
\begin{minipage}[t]{0.05\linewidth}
% tin
(ছ)
\end{minipage}
\begin{minipage}[t]{0.90\linewidth}
\underline{\textbf{অভিযোগ সংক্রান্ত:}} আলোচ্য পণ্যচালানে
গোপন সংবাদ দাতা, শুল্ক গোযেন্দা বা
জাতীয় রাজস্ব বোর্ডের কিংবা অন্য দপ্তর থেকে
কোন অভিযোগ পাওয়া যায় নাই।
\\
\end{minipage}
\begin{minipage}[t]{0.05\linewidth}
\hspace{1em}
\end{minipage}
\begin{minipage}[t]{0.05\linewidth}
% tin
(জ)
\end{minipage}
\begin{minipage}[t]{0.90\linewidth}
\underline{\textbf{ন্যায় নির্ণয় সংক্রান্ত:}} প্রযোজ্য নয়।
\\
\end{minipage}
\begin{minipage}[t]{0.05\linewidth}
% char
৪।
\end{minipage}
\begin{minipage}[t]{0.95\linewidth}
\underline{\textbf{প্রাসঙ্গিক বিষয়:}} আলোচ্য পণ্যচালানটি
জাতীয় রাজস্ব বোর্ডের পত্র এবং
{\srooof}, {\srooofd}
এর শর্ত মোতাবেক আমদানি করা হয়েছে।
উক্ত এসআরও মোতাবেক First Schedule
ভূক্ত পণ্যসমূহের মধ্যে উল্লিখিত H.S Code
এর বিপরীতে উক্ত এসআরও এর কলামে বর্ণিত
বিভিন্ন শিল্পের কাঁচামালসমূহকে, উহাদের উপর
আরোপনীয় আমদানি শুল্ক বা Custom Duty (CD),
যে পরিমাণে বর্ণিত এবং সম্পূরক শুল্ক বা
Supplementary Duty (SD),
যে পরিমাণে বর্ণিত হারের অতিরিক্ত হয় সেই পরিমাণ,
রেগুলেটরি ডিউটি হইতে, নিম্নবর্ণিত শর্তসাপেক্ষে,
অব্যাহতি প্রদান করা হয়েছে; শর্তসমূহ নিম্নরূপ-
\\
\end{minipage}
\begin{minipage}[t]{0.05\linewidth}
\hspace{1em}
\end{minipage}
\begin{minipage}[t]{0.05\linewidth}
% chare
(১)
\end{minipage}
\begin{minipage}[t]{0.9\linewidth}
 সংশ্লিষ্ট আমদানিকারককে
 Industrial IRC holder VAT compliant
 উৎপাদনকারি (Manufacturing Industry) হইতে
 হইবে;
 \\
\end{minipage}
\begin{minipage}[t]{0.1\linewidth}
\hspace{1em}
\end{minipage}
\begin{minipage}[t]{0.9\linewidth}
ব্যাখ্যা:
 \\
\end{minipage}
\begin{minipage}[t]{0.1\linewidth}
\hspace{1em}
\end{minipage}
\begin{minipage}[t]{0.9\linewidth}
(অ) Industrial IRC holder অর্থ
এইরূপ প্রতিষ্ঠান যাহার আমদানি ও রপ্তানি
প্রধান নিয়ন্ত্রকের দপ্তর হতে ইস্যুকৃত হালনাগাদ
শিল্প ভোক্তা (Industrial Consumer) আইআরসি
রহিয়াছে;
\\
\end{minipage}
\begin{minipage}[t]{0.1\linewidth}
\hspace{1em}
\end{minipage}
\begin{minipage}[t]{0.9\linewidth}
(আ) VAT Compliant অর্থ মূল্য
সংযোজন কর ও সম্পূরক শুল্ক আইন ২০১২
ও মূল্য সংযোজন কর ও সম্পূরক বিধিমালা ২০১৬
এর অধীন নিবন্ধিত নিয়মিত দাখিলপত্র (রিটার্ন) দাখিল
করে এইরূপ উৎপাদনকারী প্রতিষ্ঠান;
\\
\end{minipage}
\begin{minipage}[t]{0.05\linewidth}
\hspace{1em}
\end{minipage}
\begin{minipage}[t]{0.05\linewidth}
% chard
(২)
\end{minipage}
\begin{minipage}[t]{0.9\linewidth}
এই প্রজ্ঞাপনের উদ্দেশ্যপূরণকল্পে
pre-fabricated building বলিতে
World Customs Organization (WCO) কর্তৃক
প্রকাশিত Harmonized Commodity Description
and Coding System এর Explanatory Notes
Sixth Edition (2017) Volume-5 এর পৃষ্ঠা নং
XX-9406-1 এ বর্ণিত ব্যাখ্যা প্রাণিধানযোগ্য হইবে;
\\
\end{minipage}
\begin{minipage}[t]{0.05\linewidth}
\hspace{1em}
\end{minipage}
\begin{minipage}[t]{0.05\linewidth}
% chart
(৩)
\end{minipage}
\begin{minipage}[t]{0.9\linewidth}
এই প্রজ্ঞাপনের আওতায় পন্য আমদানি ও খালাসের লক্ষ্যে
প্রতিষ্ঠানটির হালনাগাদ নবায়নকৃত শিল্প ভোক্তা
আইআরসি রহিয়াছে এবং ১৩ ডিজিট সম্বলিত মূল্য
সংযোজন কর (মূসক) নিবন্ধিত প্রতিষ্ঠানটি বিবেচ্য
বিল অব এন্ট্রি দাখিলের অব্যবহিত পূর্ববর্তী মাসের
মূসক দাখিলপত্র (রিটার্ন) দাখিল করিয়াছে মর্মে
কাস্টমস  কম্পিউটার সিস্টেমে বা ভ্যাট অনলাইন সিস্টেমে
বা জাতীয় রাজস্ব বোর্ডের ওয়েবসাইটে বা ভ্যাট কমিশনারেট
এর ওয়েবসাইটে ইলেকট্রনিক্স তথ্য প্রদর্শিত থাকিতে হইবে;
\\
\end{minipage}
\begin{minipage}[t]{0.05\linewidth}
\hspace{1em}
\end{minipage}
\begin{minipage}[t]{0.05\linewidth}
% charc
(৪)
\end{minipage}
\begin{minipage}[t]{0.9\linewidth}
আমদানি ও শুল্কায়ন সংক্রান্ত অন্যান্য বিষয়সহ
দফা (১), (২) ও (৩) এর শর্ত পরিপালন করা হইয়াছে কিনা
তাহা সংশ্লিষ্ট শুল্কায়নকারী কর্মকর্তা কর্তৃক যাচাই করিয়া
নিশ্চিত হইতে হবে এবং শুল্কায়ন তত্ত্বাবধানকারী সংশ্লিষ্ট ডেপুটি
কমিশনারেট বা সহকারী কমিশনার পর্যায়ে শুল্কায়ন
অনুমোদন করাইতে হইবে;
\\
\end{minipage}
\begin{minipage}[t]{0.05\linewidth}
\hspace{1em}
\end{minipage}
\begin{minipage}[t]{0.05\linewidth}
% charp
(৫)
\end{minipage}
\begin{minipage}[t]{0.9\linewidth}
শুল্কায়ন তত্ত্বাবধানকারী সংশ্লিষ্ট ডেপুটি কমিশনার বা
সহকারী কমিশনার শুল্কায়নে বিলম্ব পরিহারের লক্ষ্যে
কাস্টমস কম্পিউটার সিস্টেম বা ভ্যাট
অনলাইন সিস্টেম বা জাতীয় রাজস্ব বোর্ড এর
ওয়েবসাইট বা ভ্যাট কমিশনারেট এর ওয়েবসাইট
হইতে সঠিকতা যাচাই করিবেন;
\\
\end{minipage}
\begin{minipage}[t]{0.05\linewidth}
\hspace{1em}
\end{minipage}
\begin{minipage}[t]{0.05\linewidth}
% charc
(৬)
\end{minipage}
\begin{minipage}[t]{0.9\linewidth}
প্রত্যেক মূসক কমিশনার কাস্টম হাউস বা
কাস্টমস স্টেশন কর্তৃপক্ষ অথবা জাতীয় রাজস্ব বোর্ডের
Customs Information System (CIS)
সেলের সহিত নিয়মিত যোগাযোগ রাখিয়া তাহার
প্রশাসনিক এলাকাধীন রেয়াতী সুবিধা ভোগকারী
আমদানিকারক-উৎপাদকদের একটি হালনাগাদ তালিকা
প্রণয়নপূর্বক প্রত্যেকের আমদানি তথ্য সংগ্রহ ও যাচাই
করিবেন;
\\
\end{minipage}
\begin{minipage}[t]{0.05\linewidth}
\hspace{1em}
\end{minipage}
\begin{minipage}[t]{0.05\linewidth}
% chars
(৭)
\end{minipage}
\begin{minipage}[t]{0.9\linewidth}
এই প্রজ্ঞাপনের আওতায় রেয়াতি হারে আমদানিকৃত
উপকরণ বা কাঁচামাল অনুযায়ী ব্যবহারপূর্বক মূসক
আইন ও মূসক বিধিমালা অনুযায়ী
যথাযথ পরিমাণ মূসক ও সম্পূরক শুল্ক (যদি থাকে) প্রদান
করা হইয়াছে বা হইতেছে কিনা তাহা সংশ্লিষ্ট মূসক
কমিশনারেট অডিট বা যাচাইপূর্বক নিশ্চিত করিবেন;
\\
\end{minipage}
\begin{minipage}[t]{0.05\linewidth}
\hspace{1em}
\end{minipage}
\begin{minipage}[t]{0.05\linewidth}
% chara
(৮)
\end{minipage}
\begin{minipage}[t]{0.9\linewidth}
কোন আমদানিকারক এই প্রজ্ঞাপনে প্রদত্ত রেয়াতী
সুবিধার অপব্যবহার করিয়াছেন বা করিতেছেন
মর্মে দফা (৭) এ উল্লিখিত কর্তৃপক্ষ হইতে সুনির্দিষ্ট
প্রামাণিক তথ্য সম্বলিত অভিযোগ পাওয়া গেলে সংশ্লিষ্ট
কাস্টম হাউস বা কাস্টমস স্টেশন কর্তৃপক্ষ উক্ত অভিযোগের
নিষ্পত্তি না হওয়া পর্যন্ত উক্ত আমদানিকারকগণকে
এই প্রজ্ঞাপনের অধীন রেয়াতি সুবিধা প্রদান স্থগিত রাখিবেন;
\\
\end{minipage}
\begin{minipage}[t]{0.05\linewidth}
\hspace{1em}
\end{minipage}
\begin{minipage}[t]{0.05\linewidth}
% charn
(৯)
\end{minipage}
\begin{minipage}[t]{0.9\linewidth}
বাংলাদেশ রপ্তানি প্রক্রিয়াকরণ অঞ্চল কর্তৃপক্ষ (BEPZA),
বাংলাদেশ অর্থনৈতিক অঞ্চল কর্তৃপক্ষ (BEZA),
বাংলাদেশ হাইটেক পার্ক কর্তৃপক্ষ অথবা আইন বলে প্রতিষ্ঠিত
অন্য যে কোনো কর্তৃপক্ষের নিয়ন্ত্রণাধীন শিল্প প্রতিষ্ঠান যাহাদের জন্য শিল্প ভোক্তা আইআরসি গ্রহণের বাধ্যবাধকতা নেই সেইরূপ
ক্ষেত্রে সংশ্লিষ্ট কর্তৃপক্ষ কর্তৃক ইস্যুকৃত আমদানি অনুমতিপত্র
দাখিল করিতে হইবে;
\\
এবং
\\
\end{minipage}
\begin{minipage}[t]{0.05\linewidth}
\hspace{1em}
\end{minipage}
\begin{minipage}[t]{0.05\linewidth}
% chard
(১০)
\end{minipage}
\begin{minipage}[t]{0.9\linewidth}
দফা (৮) এ বর্ণিত অভিযোগ চূড়ান্তভাবে প্রতিষ্ঠিত হইলে
অপব্যবহারজনিত রেয়াতের সহিত সংশ্লিষ্ট শুল্ক ও কর আদায়
ছাড়াও Customs Act, 1969 এর section 156 এর
sub-section (1) এর টেবিল এর Item 10A এর বিধান
অনুযায়ী ব্যবস্থা গ্রহণ করিতে হইবে।
\\
\end{minipage}
\begin{minipage}[t]{0.05\linewidth}
% char
০৫।
\end{minipage}
\begin{minipage}[t]{0.95\linewidth}
\underline{\textbf{এসআরও শর্ত পূরণের লক্ষ্যে
দাখিলকৃত দলিলাদি:}}
\end{minipage}
%\footnotesize
\begin{minipage}[t]{0.05\linewidth}
\hspace{0em}
\end{minipage}
\begin{minipage}[t]{0.05\linewidth}
% char
(১)
\end{minipage}
\begin{minipage}[t]{0.90\linewidth}
আলোচ্য আমদানিকারক হালনাগাদ Industrial IRC
দাখিল করেছেন, যার নং- {\ircno} এবং
{\ircrenewdt} তারিখ পর্যন্ত নবায়নকৃত আছে।
\end{minipage}
\begin{minipage}[t]{0.05\linewidth}
\hspace{0em}
\end{minipage}
\begin{minipage}[t]{0.05\linewidth}
% chard
(২)
\end{minipage}
\begin{minipage}[t]{0.90\linewidth}
আমদানিকারক প্রতিষ্ঠানটি শিল্প প্রতিষ্ঠান হিসেবে
মূসক-২.৩, মূসক-৪.৩ দাখিল করেছেন।
\end{minipage}
\begin{minipage}[t]{0.05\linewidth}
\hspace{0em}
\end{minipage}
\begin{minipage}[t]{0.05\linewidth}
% chard
(৩)
\end{minipage}
\begin{minipage}[t]{0.90\linewidth}
বি/ই দাখিলের অব্যবহিত
পূর্ববর্তী মাসের {\musokr} পর্যন্ত দাখিলপত্র
(রিটার্ন) দাখিল করেছেন, যা NBR এর ওয়েবসাইট
যাচাইয়ে সঠিক পাওয়া যায়।
\\
\end{minipage}
\begin{minipage}[t]{0.05\linewidth}
% choi
০৬।
\end{minipage}
\begin{minipage}[t]{0.95\linewidth}
\underline{\textbf{শুল্কায়ন সম্পর্কিত প্রস্তাব:}}
পণ্য চালান সংক্রান্ত আমদানি দলিলপত্র
ইনভয়েস পর্যালোচনা পূর্বক পণ্যের বর্ণনা, পরিমাণ, এইচ.এস.কোড, ঘোষিত মূল্য,
সমসাময়িককালের অভিন্ন/অনুরূপ পণ্যের একক মূল্যসহ নিম্নের ছকে শুল্কায়নের প্রস্তাব উপস্থাপন
করা হলো:
\end{minipage}
\scriptsize
\begin{minipage}{1\textwidth}
{\taxtab}
\vspace{5mm}
\end{minipage}
\normalsize
\begin{minipage}[t]{0.05\linewidth}
% choi
০৭।
\end{minipage}
\begin{minipage}[t]{0.95\linewidth}
\underline{\textbf{শুল্কায়নযোগ্য মূল্য নিরুপন:}} ASYCUDA WORLD SYSTEM
-এ আলোচ্য পণ্য চালানের বি/ই দাখিলের পূর্ববর্তী ৯০ (নব্বই) দিনের
মূল্য তথ্য পর্যালোচনা করে দেখা যায়, আমদানিকৃত
পণ্যের শুল্কায়িত সর্বনিম্ন যে বিনিময় মূল্য পাওয়া যায়, উক্ত মূল্য অপেক্ষা ঘোষিত মূল্য বেশি হওয়ায়
শুল্ক মূল্যায়ন (আমদানি পণ্যের মূল্য নির্ধারণ) বিধিমালা,
২০০০ (এসআরও নং- ৫৭/আইন/২০০০/১৮২১/শুল্ক, তারিখ: ২৩/০২/২০০০)
এর বিধি- ৪ অনুযায়ী আলোচ্য পণ্যচালানটি প্রস্তাবিত
মূল্যে শুল্কায়নযোগ্য।
\\
\end{minipage}
\begin{minipage}[t]{0.05\linewidth}
% sat
০৮।
\end{minipage}
\begin{minipage}[t]{0.95\linewidth}
\underline{\textbf{সার্বিক পর্যালোচনা পূর্বক নিম্নোক্ত প্রস্তাব উপস্থাপন করা হলো:}}
\end{minipage}
\begin{minipage}[t]{0.05\linewidth}
\hspace{0em}
\end{minipage}
% % \footnotesize
\begin{minipage}[t]{0.95\linewidth}
প্রস্তাব:
\end{minipage}
\begin{minipage}[t]{0.05\linewidth}
\hspace{0em}
\end{minipage}
\begin{minipage}[t]{0.05\linewidth}
(ক)
\end{minipage}
\begin{minipage}[t]{0.90\linewidth}
আমদানিকারক প্রতিষ্ঠান কর্তৃক আলোচ্য চালানের
ক্ষেত্রে
{\srooof}, {\srooofd}
এর শর্তাবলী (প্রযোজ্য ক্ষেত্রে) পরিপালিত হয়েছে
বিধায় পণ্যচালানটি {\srooof}, {\srooofd}
{\cpcfzn} -তে নোট অনুচ্ছেদ-৬ এর প্রকৃত H.S Code
ও প্রস্তাবিত মূল্যে শুল্কায়ন অনুমোদন দেয়া যেতে পারে।
\\
\end{minipage}
\begin{minipage}[t]{0.05\linewidth}
\hspace{0em}
\end{minipage}
\begin{minipage}[t]{0.05\linewidth}
(খ)
\end{minipage}
\begin{minipage}[t]{0.90\linewidth}
আমদানিকারক কমিশনার মহোদয় বরাবর
একটি আবেদন করেছেন। আবেদন পত্র নং নাই তারিখ: ৩০/০১/২০২২
নথির ডান পার্শ্বে রক্ষিত আছে, দয়া করে দেখা যেতে পারে।
\\
আবেদনে আমদানিকারক উল্লেখ করেন যে, তারা যমুনা
ব্র্যান্ডের ফ্রিজ উৎপাদন করেন। আলোচ্য বি/ই এর মাধ্যমে
আমদানিকৃত ৭ টি আইটেমের মধ্যে ১ ও ৪ নং আইটেম
HOT ZINC COIL 0.30*1200, 0.30*1100
এর মধ্যে
ZINC ও ALUMINIUM
রয়েছে।
খালাসকালে নমুনা উত্তোলন করে
ল্যাব টেস্টের যে ফলাফল আসবে
তা মেনে নিতে বাধ্য থাকবেন।
\\
\\
এমতাবস্থায় আমদানিকারকের
আবেদন বিবেচিত হলে খালাসকালে
বি/ই নং {\beno}, {\bedt}
এর আইটেম ১ এবং ৪ এর
নমুনা উত্তোলন করে
কাস্টম ল্যাবে প্রেরণ করা যেতে পারে।
\end{minipage}
\thispagestyle{laststyle}
\end{document}
