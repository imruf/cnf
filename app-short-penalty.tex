\documentclass[12pt]{article}
\usepackage[legalpaper,
            lmargin=1in,rmargin=0.5in,
            bmargin=0.5in,tmargin=1in,
            headheight=0.2in, includefoot]{geometry}
\usepackage{fontspec}
\usepackage{setspace}
\usepackage{titlesec}
\usepackage{multirow}
\usepackage[colorlinks=true,urlcolor=Blue]{hyperref}
\usepackage{graphicx}
\usepackage{array}
\usepackage{makecell}
\usepackage{fancyhdr}
\usepackage[none]{hyphenat}
\usepackage{longtable}
\usepackage[dvipsnames]{xcolor}
\usepackage[banglamainfont=Kalpurush,
            banglattfont=SolaimanLipi,
            % feature=1,
            % changecounternumbering=0
           ]{latexbangla}
% oncehsis imp file name
\newcommand{\eif}{নথি নং- এস \hspace{4em}/স্টাফ/কাস/\hspace{5em}/অংশ-}
\newcommand{\jdlf}{নথি নং- এস৫-২৪৫/স্টাফ/কাস/০৬-০৭/অংশ-২}
\newcommand{\jdwlf}{নথি নং- এস৫-১৭০/স্টাফ/কাস/১৪-১৫/অংশ-২}
\newcommand{\jealf}{নথি নং- এস-৫-৫৮৫/স্টাফ/কাস/২০১৩-২০১৪/অংশ-১২}
\newcommand{\jfclf}{নথি নং- এস৫-৫১৯/স্টাফ/কাস/২০-২১}
\newcommand{\jsmlf}{নথি নং- এস৫-৬৯১/স্টাফ/কাস/২০১৭-২০১৮/অংশ-০৩}
\newcommand{\jtrif}{নথি নং- এস-৫-১২১০/স্টাফ/কাস/১৬-১৭/অংশ-১}
\newcommand{\jwelf}{নথি নং- এস-৫-১১৪/স্টাফ/কাস/২০১৮-১৯/}
\newcommand{\scmlf}{নথি নং- এস-৫-১৬৩/স্টাফ/কাস/০৬-০৭/অংশ-৫}
\newcommand{\ssmlf}{নথি নং- এস-৫-১৪৪/স্টাফ/কাস/০৬-০৭/অংশ-১১}
\newcommand{\tdjf}{নথি নং- এস-৫-৪৪৩/স্টাফ/কাস/০৬-০৭/অংশ-}
\newcommand{\htff}{নথি নং- এস-৫-৬৮৪/স্টাফ/কাস/২০১৭-১৮/অংশ-৩}

% fileno
\newcommand{\filenou}{
\begin{minipage}[t]{0.60\linewidth}
\hspace{1em}
\end{minipage}
\begin{minipage}[t]{0.40\linewidth}
\underline{\fileno}
\end{minipage}
}


% new page
\newcommand{\nfpage}{
\newpage
\small
{\filenou}
}

% imp name
\newcommand{\ssml}{SHAMEEM SPINNING MILLS LTD.}
\newcommand{\ssmla}{SHAFIPUR, KALIAKAIR, GAZIPUR-1751, BANGLADESH.}
\newcommand{\scml}{SHAMEEM COMPOSITE MILLS LTD.}
\newcommand{\scmla}{SHAFIPUR, KALIAKAIR, GAZIPUR-1751, BANGLADESH}
\newcommand{\jsml}{JAMUNA SPINNING MILLS LTD.}
\newcommand{\jsmla}{SHAFIPUR, KALIAKAIR, GAZIPUR-1751, BANGLADESH.}
\newcommand{\jsmlaut}{JAMUNA SPINNING MILLS LTD. UNIT-2}
\newcommand{\jsmlauta}{BEJURA, SOUTH BEJURA, MADHABPUR, HOBIGONJ-3331, BANGLADESH.}
\newcommand{\jdl}{JAMUNA DENIMS LTD.}
\newcommand{\jdla}{SHAFIPUR, KALIAKAIR, GAZIPUR-1751, BANGLADESH.}
\newcommand{\jeal}{JAMUNA ELECTRONICS \& AUTOMOBILES LTD.}
\newcommand{\jeala}{SINABHA, KALIAKAIR PS, GAZIPUR-1750, BANGLADESH}
\newcommand{\jdwl}{JAMUNA DENIMS WEAVING LTD.}
\newcommand{\jdwla}{KASHIMPUR ROAD, JARUN, KONABARI PS, GAZIPUR-1751, BD}
\newcommand{\jfcl}{JAMUNA FAN AND CABLES LTD.}
\newcommand{\jfcla}{KASHIMPUR ROAD, JARUN, KONABARI PS, GAZIPUR-1751, BD}
\newcommand{\htf}{HOORAIN HTF LTD.}
\newcommand{\htfa}{BEJURA, SOUTH BEJURA, MADHABPUR, HOBIGONJ-3331, BANGLADESH.}
\newcommand{\tdj}{M/S THE DAILY JUGANTOR}
\newcommand{\tdja}{KA-244, KURIL, PROGATI SARANI, VATARA PS, DHAKA-1229, BANGLADESH}
\newcommand{\jtri}{JAMUNA TYRE AND RUBBER INDUSTRIES}
\newcommand{\jtria}{BEJURA, SOUTH BEJURA, MADHABPUR, HOBIGONJ}
\newcommand{\jwel}{JAMUNA WELDING ELECTRODE LTD.}
\newcommand{\jwela}{CHOYDANA, HAZIRPUKUR; GAZIPUR SADAR, GAZIPUR-1704; BANGLADESH}
\newcommand{\cbl}{CROWN BEVERAGE LIMITED}
\newcommand{\cbla}{SHAFIPUR, KALIAKAIR, GAZIPUR-1751, BANGLADESH}
\newcommand{\jpml}{JAMUNA PAPER MILLS LIMITED}
\newcommand{\jpmla}{BEJURA, SOUTH BEJURA MADHABPUR, HOBIGONJ, BANGLADESH}

% imp bin
\newcommand{\jealbin}{000146478-0103}
\newcommand{\tdjbin}{001960365-0101}
\newcommand{\jsmlbin}{000144425-0103}
\newcommand{\jfclbin}{001753334-0103}
\newcommand{\scmlbin}{000151754-0103}
\newcommand{\htfbin}{000146478-0103}
\newcommand{\cblbin}{000149506-0103}

% cnf name
\newcommand{\cnfn}{SHAMEEM SPINNING MILLS LTD.}
\newcommand{\cnfadd}{92, HIGH LEVEL ROAD, LALKHAN BAZAR, CHITTAGONG}
\newcommand{\cnfain}{301 08 3417}

% sros
\newcommand{\srofs}{এসআরও নং- ৫৭- আইন/২০০০/১৮২১/শুল্ক}
\newcommand{\srofsd}{তারিখ: ২৩/০২/২০০০ খ্রি:}
\newcommand{\srooot}{এসআরও নং- ১১৩- আইন/২০২১/০২/কাস্টমস}
\newcommand{\sroootd}{তারিখ: ২৪/০৫/২০২১ খ্রি:}
\newcommand{\srooof}{এসআরও নং- ১১৪- আইন/২০২১/০৩/কাস্টমস}
\newcommand{\srooofd}{তারিখ: ২৪/০৫/২০২১ খ্রি:}
\newcommand{\srootz}{এসআরও নং- ১২০- আইন/২০২১/০৯/কাস্টমস}
\newcommand{\srootzd}{তারিখ: ২৪/০৫/২০২১ খ্রিঃ}

\newcommand{\sroott}{এসআরও নং- ১৩৩- আইন/২০২১/২২/কাস্টমস}
\newcommand{\sroottd}{তারিখ: ২৪/০৫/২০২১ খ্রিঃ}

\newcommand{\srotsz}{এসআরও নং- ৩৬০- আইন/২০১৩/২৪৬৮/কাস্টমস}
\newcommand{\srotszd}{তারিখ: ২৫/১১/২০১৩ খ্রি:}

% cpcs
\newcommand{\cpcofs}{CPC- 4000/157}
\newcommand{\cpcost}{CPC- 4000/173}
\newcommand{\cpcttz}{CPC- 4000/220}
\newcommand{\cpcfzo}{CPC- 4000/401}
\newcommand{\cpcfzn}{CPC- 4000/409}
\newcommand{\cpcsso}{CPC- 4000/661}

% nbrl
\newcommand{\nbrosn}{জাতীয় রাজস্ব বোর্ডের পত্র নং- ০৮.০১.০০০০.০৬৮.১৮.০০৪.১৭/১৬৯}
\newcommand{\nbrosnd}{তারিখ: ২৭/০৬/২০২১ ইং}
\newcommand{\nbrosnt}{জাতীয় রাজস্ব বোর্ডের পত্র নং- ০৮.০১.০০০০.০৬৮.১৮.০০৪.১৭/১৬৯(৩)}
\newcommand{\nbrosntd}{তারিখ: ২৭/০৬/২০২১ খ্রি:}
\newcommand{\nbrfs}{জাতীয় রাজস্ব বোর্ডের পত্র নং- ০৮.০১.০০০০.০৩৪.০২.৩০১.১৯-৪৭}
\newcommand{\nbrfsd}{তারিখ: ১২/০৭/২০২১ খ্রি:}

% imp reg name
\newcommand{\ssmlreg}{২৩১৩৯৫০৬১৮০-এইচ, তারিখ: ০৯/০১/২০১৩}
\newcommand{\scmlreg}{৯৭০৪০১৭-এইচ তারিখ: ২১/০৪/১৯৯৭}
\newcommand{\jealreg}{এল-২৯৩০১৩০১৩৪৩১-এইচ, তারিখ: ০৯/০১/২০১৩}
\newcommand{\eireg}{\hspace{10em} তারিখ: \hspace{5em}}

% imp irc no
\newcommand{\jealirc}{260326120426720}
\newcommand{\scmlirc}{260326120041719}
\newcommand{\jfclirc}{260326120515020}

% san name
\newcommand{\maersk}{MAERSK BANGLADESH LIMITED}
\newcommand{\maerska}{58, AGRABAD COMMERCIAL AREA (3RD FLOOR), CHITTAGONG, 4100.}
\newcommand{\apl}{APL (BANGLADESH) PVT.LTD}
\newcommand{\apla}{PLOT NO. 30, 3RD FLOOR OF SURAIYA MANSION, AGRABAD, CHITTAGONG.}
\newcommand{\baridhi}{BARIDHI SHIPPING LINES LTD}
\newcommand{\baridhia}{3/F HRC BHABAN, 64-66 AGRABAD C/A, CHITTAGONG.}
\newcommand{\continentalbd}{CONTINENTAL TRADERS (BD) LIMITED}
\newcommand{\continentalbda}{73, AGRABAD C/A, CHITTAGONG.}
\newcommand{\continentaltr}{CONTINENTAL TRADERS (BD) LIMITED}
\newcommand{\continentaltra}{IQBAL BHABAN, AGRABAD C/A, CHITTAGONG.}
\newcommand{\gbx}{GBX LOGISTICS LTD}
\newcommand{\gbxa}{AYUB TRADE CENTER(1ST FLOOR), 1269/B, SK MUJIB ROAD, AGRABAD C/A, CHITTAGONG.}
\newcommand{\transmarine}{TRANSMARINE LOGISTICS LTD}
\newcommand{\transmarinea}{B.M.HEIGHTS(4TH FLOOR), 318, SK, MUJIB ROAD, AGRABAD C/A, CHITTAGONG.}
\newcommand{\trident}{TRIDENT SHIPPING LINE LTD}
\newcommand{\tridenta}{AKHTARUZZAMAN CENTER, 6TH FLOOR 21/22 AGRABAD, CHITTAGONG.}
\newcommand{\msc}{MSC MEDITERRANEAN SHIPP.CO.BD.LTD}
\newcommand{\msca}{IIUC TOWER, 4TH FLOOR, 1700/A SK.MUJIB ROAD, PLOT-09, AGRABAD, CHITTAGONG.}
\newcommand{\alviline}{ALVILINE BANGLADESH LIMITED}
\newcommand{\alvilinea}{78, AGRABAD C/A, MACCA MADINA TRADE CENTER, 9TH FLOOR, CHITTAGONG}
\newcommand{\ocean}{OCEAN NETWORK EXPRESS (BD) LTD}
\newcommand{\oceana}{IIUC TOWER (10TH FLOOR), 1700/A, PLOT-9, SK.MUJIB ROAD, AGRABAD C/A, CHITTAGONG}
\newcommand{\vega}{VEGA MARINE PVT LIMITED}
\newcommand{\vegaa}{DAAR-E SHAHIDI, 4TH FLOOR, 69 AGRABAD C/A, CTG}
\newcommand{\mega}{MEGATREND SHIPPING LINES LTD.}
\newcommand{\megaa}{MAKKAH MADINAH TRADE CENTER (16TH FLOOR), 78, AGRABAD, CHITTAGONG}
\newcommand{\famfa}{FAMFA SOLUTION LIMITED}
\newcommand{\famfaa}{BONANI, AGRABAD, CHITTAGONG}
\newcommand{\reliance}{RELIANCE SHIPPING SERVICES}
\newcommand{\reliancea}{34 AGRABAD C/A, CHITTAGONG 4100, BANGLADESH}
%\newcommand{\alvilinea}{}


% mujib logo
\newcommand{\my}{\includegraphics[height=3.2em]{pic/my.png}}

% slogan
\fancypagestyle{slogan}
{
\fancyhf{}
\renewcommand{\headrulewidth}{0pt}
% header
\lhead{
\framebox[1.1\width]{\footnotesize{``উন্নয়নের অক্সিজেন রাজস্ব''}}
}
\rhead{
\my
\\
\framebox[1.1\width]{\footnotesize{``জনকল্যানে রাজস্ব''}}
}
}

% customs
\newcommand{\tca}{The Customs Act, 1969}





\pagestyle{fancy}
\fancyhf{}
\renewcommand{\headrulewidth}{0pt}
% header
\chead{
\underline{{পৃষ্ঠা - \thepage}}
}
% footer
\rfoot{চলমান পৃষ্ঠা-\thepage}

\fancypagestyle{laststyle}
{
   \fancyfoot[R]{}
   \fancyfoot[L]{}
   \fancyfoot[C]{}
   \fancyhead[R]{}
   \fancyhead[L]{}
   \fancyhead[C]{}
}

\fancypagestyle{case}
{
\fancyhf{}
\renewcommand{\headrulewidth}{0pt}
% header
\lhead{
{\footnotesize{``উন্নয়নের অক্সিজেন রাজস্ব''}}
}
\rhead{
{\footnotesize{``জনকল্যানে রাজস্ব''}}
}
}


\newcommand{\fileno}{নথি নং - ২২৫২/এপি/সেকশন-৭(এ)/২০২১-২০২২}
\newcommand{\product}{JAMUNA FREEZER}
\newcommand{\good}{KRAFT SILICON PAPER, EPOXY WHITE POWDER}
\newcommand{\pkg}{11 PKG=8,763.00 KG}
\newcommand{\co}{CHINA}
\newcommand{\coship}{CHINA}
\newcommand{\vessel}{HAIAN WEST}
\newcommand{\rotno}{2021/6136}
\newcommand{\blno}{1ZSSHA21120956}
\newcommand{\bldt}{27.12.21}
\newcommand{\beno}{C-25420}
\newcommand{\bedt}{04.01.2022}
\newcommand{\lcno}{0000174121020359}
\newcommand{\lcdt}{13.12.21}
\newcommand{\lcano}{275993}
\newcommand{\lcadt}{\lcdt}
\newcommand{\lienbank}{MERCANTILE BANK LTD.}
\newcommand{\invno}{BST20212212201}
\newcommand{\invdt}{22.12.21}

\newcommand{\impn}{\jeal}
\newcommand{\impadd}{\jeala}
\newcommand{\impbin}{\jealbin}

\newcommand{\crf}{NON CRF}
\newcommand{\crfdt}{}
\newcommand{\ircno}{নং- ২৬০৩২৬১২০৪২৬৭২০}
\newcommand{\ircrenewdt}{৩০.০৬.২০২২}
\newcommand{\musokr}{}
\newcommand{\hscode}{84189910}
\newcommand{\price}{US\$ 21,522.30}
\newcommand{\menifest}{2021/6136}


\newcommand{\rodt}{\hspace{3em}-\hspace{3em}-২০২২ খ্রি:}

\begin{document}
\begin{center}
\textbf{\underline{সংক্ষিপ্ত ন্যায়-নির্ণয়ের আবেদন ফরম}}
\end{center}
\noindent
বরাবর
\\
\begin{minipage}[t]{0.06\linewidth}
\hspace{1em}
\end{minipage}
\begin{minipage}[t]{0.94\linewidth}
কমিশনার অব কাস্টমস
\\
অতিরিক্ত কমিশনার অব কাস্টমস
\\
যুগ্ম/ডেপুটি/সহকারী কমিশনার অব কাস্টমস
\\
রাজস্ব কর্মকর্তা/এস.পি.এস
\\
কাস্টম হাউজ,
চট্টগ্রাম।
\\
\\
\end{minipage}
\begin{minipage}[t]{0.06\linewidth}
বিষয়:
\end{minipage}
\begin{minipage}[t]{0.94\linewidth}
জাহাজ: {\vessel},
ROT NO. {\menifest},
B/L NO. {\blno},
DT: {\bldt}
-এর মাধ্যমে আমদানিকৃত পণ্য
{\good}
খালাসের নিমিত্তে
দাখিলকৃত
B/E NO: {\beno}, DT: {\bedt}।
\\
\\
\end{minipage}
মহোদয়,
\\
\hspace*{2.7em}আমি/আমরা এই মর্মে
ঘোষণা করছি যে, বিষয়ে উল্লেখিত আমদানিকৃত পণ্যচালান অসত্য
HS.Code
ঘোষণায়  আমদানি হওয়াতে
আমদানি নীতি আদেশ -এর ধারা ৩২ লঙ্ঘনের কারণে,
{\tca} -এর ধারা ১৫৬(১) এর টেবিলের দফা ১৪ অনুযায়ী
শাস্তিযোগ্য অপরাধ।
\\
\\
\hspace*{2.7em}
যেহেতু উল্লেখিত চালানের পণ্য আমাদের জরুরি প্রয়োজন
এবং যেহেতু আমরা পণ্যচালানটি দ্রুত খালাস করতে ইচ্ছুক
তাই আমি/আমরা {\tca} এর ধারা ১৮০ বলে কারণ দর্শানোর
নোটিশ জারীকরণ, ব্যক্তিগত শুনানি গ্রহণ ইত্যাদি ব্যতিরেকে
সংক্ষিপ্ত ন্যায়-নির্ণয়ের মাধ্যমে বিচার সম্পাদন করার আবেদন
জানাচ্ছি।
এ ক্ষেত্রে আপনি যে বিচারাদেশ জারী করবেন তা গ্রহণ করার
ব্যাপারে আমি/আমাদের সম্মতি জ্ঞাপন করছি।
\\
\\
\begin{minipage}[t]{0.55\linewidth}
\hspace{1em}
\end{minipage}
\begin{minipage}[t]{0.45\linewidth}
বিনীত নিবেদক
\\
\\
\\
\end{minipage}
\begin{minipage}[t]{0.55\linewidth}
তারিখ: {\rodt}
\end{minipage}
\begin{minipage}[t]{0.45\linewidth}
আমদানিকারক / ক্ষমতাপ্রাপ্ত ব্যক্তির স্বাক্ষর ও সীল
\\
\\
....................................................................
\\
\\
ঠিকানা- .........................................................
\\
....................................................................
\\
ফোন নম্বর-  ...................................................
\end{minipage}
\thispagestyle{empty}
\newpage
\setstretch{1.3}
\fontsize{10pt}{10pt}\selectfont
\begin{center}
গণপ্রজাতন্ত্রী বাংলাদেশ সরকার
\\
কাস্টম হাউস, চট্টগ্রাম।
\\
আদেশ নং ...................................................................................................................................... তারিখ: .........................
\\
আদেশ প্রদানকারী কর্মকর্তা জনাব মোঃ আল-আমিন, ডেপুটি কমিশনার অব কাস্টমস, কাস্টম হাউস, চট্টগ্রাম।
\\
\textbf{\underline{``মামলার সংক্ষিপ্ত বিবরণ''}}
\\
\end{center}
\begin{minipage}[t]{0.05\linewidth}
% ek
\hspace{0em}
\end{minipage}
\begin{minipage}[t]{0.95\linewidth}
মিথ্যা ঘোষণায় পণ্য আমদানি করার অপরাধ স্বীকার করে আমদানিকারকের পক্ষে
সিএন্ডএফ এজেন্টের প্রতিনিধি সংক্ষিপ্ত বিচারের আবেদন করেছেন।
তাঁর আবেদন গ্রহণ করা হলো। আদেশের এই কপিটি যার নামে জারী করা হয়েছে
তার নিজস্ব ব্যবহারের জন্য বিনামূল্যে প্রদান করা হলো।
\\
\end{minipage}
\begin{minipage}[t]{0.05\linewidth}
০২।
\end{minipage}
\begin{minipage}[t]{0.95\linewidth}
আমদানিকারক {\impn}, {\impadd}
এলসি নং-{\lcno}, DT:{\lcdt}
এর মাধ্যমে {\co}
হতে
(1) KRAFT SILICON PAPER,
(2) EPOXY WHITE POWDER
ঘোষণায় পণ্য আমদানি করে তা
খালাসের লক্ষ্যে আমাদানিকারকের
মনোনিত সিএন্ডএফ এজেন্ট
{\cnfn}, {\cnfadd} -এর
মাধ্যমে
B/E NO. {\beno}, DT: {\bedt}
দাখিল করেন।
নথি পর্যালোচনায় দেখা যায় যে,
ক্রমিক নং-২ এর পণ্য
EPOXY WHITE POWDER
ঘোষিত HS Code 84189910
(CD -25\%, AIT-0.83\% )
প্রদান করলেও
{\fsen}
অনুযায়ী উল্লিখিত
EPOXY WHITE POWDER
পণ্যের প্রকৃত HS Code 32089090
(CD -25\%, RD-3\%, AIT-0.83\% ) -এ
শ্রেণীবিন্যাসযোগ্য ও শুল্ক করাদি আদায়যোগ্য।
অর্থাৎ মিথ্যা ঘোষণায় পণ্য আমদানি করেছেন।
\end{minipage}
\begin{center}
\textbf{\underline{পর্যালোচনা}}
\end{center}
\begin{minipage}[t]{0.05\linewidth}
০৩।
\end{minipage}
\begin{minipage}[t]{0.95\linewidth}
পণ্য চালানের ইনভয়েস, প্যাকিং লিস্ট ও
শিপিং দলিলাদি দেখলাম।
ঘোষণা অনুযায়ী চালানের পণ্য
(1) KRAFT SILICON PAPER,
(2) EPOXY WHITE POWDER।
পণ্যচালানটি শতভাগ কায়িক পরীক্ষা করা হয়।
ক্রমিক নং-২ এর পণ্য
EPOXY WHITE POWDER
ঘোষিত HS Code 84189910
(CD -25\%, AIT-0.83\% )
প্রদান করলেও
{\fsen}
অনুযায়ী উল্লিখিত
EPOXY WHITE POWDER
পণ্যের প্রকৃত HS Code 32089090
(CD -25\%, RD-3\%, AIT-0.83\% ) -এ
শ্রেণীবিন্যাসযোগ্য ও শুল্ক করাদি আদায়যোগ্য।
আলোচ্য চালানের মোট
শুল্কায়নযোগ্য মূল্য
১৮,৫৬,৪০৪.৯৮ (আঠারো লক্ষ ছাপ্পান্ন হাজার চারশত চার টাকা আটানব্বই পয়সা) টাকা।
এবং মিথ্যা ঘোষণা প্রদান করায় রাজস্ব হানি/ঝুঁকির পরিমান
৪১,৮০৯.৮৫
(একচল্লিশ হাজার আটশত নয় টাকা পঁচাশি পয়সা) টাকা।
আলোচ্য ক্ষেত্রে আমদানিকারক মিথ্যা ঘোষণায়
পণ্য আমদানি করেছেন।
এর ফলে মিথ্যা ঘোষণায় পণ্য আমদানির আপরাধ
প্রমাণিত ও প্রতিষ্ঠিত।
যা {\tca} এর Section 16 এর
লঙ্ঘনসহ Section 32 মোতাবেক
অসত্য ঘোষণা যা উক্ত {\tca}
এর Section 156(1) এর Table এর
Clause 9(i) ও 14 মোতাবেক দন্ডনীয় অপরাধ।
আলোচ্য ক্ষেত্রে আমদানিকারকের পক্ষে সিএন্ডএফ
এজেন্টের প্রতিনিধি নিজেও অসত্য ঘোষণার অপরাধ
স্বীকার করেছেন।
\end{minipage}
\begin{center}
\textbf{\underline{আদেশ}}
\end{center}
\begin{minipage}[t]{0.05\linewidth}
০৪।
\end{minipage}
\begin{minipage}[t]{0.95\linewidth}
আমদানিকারক {\tca} এর Section 16
এর লঙ্ঘনসহ Section 32 মোতাবেক
অসত্য ঘোষণার বিষয়টি সন্দেহাতীতভাবে প্রমাণিত
হওয়ায় উক্ত {\tca} এর Section 156(1) এর
Table এর Clause 9(i) ও 14 এর বিধান-এ প্রদত্ত
ক্ষমতা বলে উক্ত পণ্যচালানটি
রাষ্ট্রের অনুকূলে বাজেয়াপ্তকরণসহ উক্ত আমদানিকারকের
উপর
৮৫,০০০ (পঁচাশি হাজার টাকা) টাকা
মাত্র অর্থদন্ড আরোপ করা হলো। তবে পণ্যগুলি অবাধে আমদানিযোগ্য
বিধায় {\tca} এর Section 181 ধারায় প্রদত্ত ক্ষমতা
বলে
৫০০০ (পাঁচ হাজার টাকা) টাকা মাত্র
বিমোচন জরিমানা আরোপ করা হলো।
চালানের পণ্য যথাযথ HS Code -এ শ্রেণিবিন্যাস ও যথাযথ
মূল্যে শুল্কায়নপূর্বক প্রযোজ্য শুল্ক করাদি, অর্থদন্ড, বিমোচন জরিমানা
ইত্যাদি পরিশোধ এবং প্রযোজ্য ক্ষেত্রে আমদানি নীতি আদেশের
শর্তাদি প্রতিপালন ও প্রযোজ্য আনুষ্ঠানিকতা পরিপালন সাপেক্ষে
আমদানিকারকের অনুকূলে পণ্য খালাস দেওয়ার আদেশ দিলাম।
উল্লেখ্য জরিমানার অর্থ শুল্ক করাদির অতিরিক্ত হিসাবে আদায়যোগ্য
হবে। এছাড়া সংশ্লিষ্ট সেকশনের রাজস্ব কর্মকর্তা/সহকারী রাজস্ব কর্মকর্তা
প্রযোজ্য ক্ষেত্রে পণ্যের বি/ই এর যাবতীয় Adjustment এর কার্যাদিও সম্পন্ন করবেন।
\\
\\
\\
\\
\\
\end{minipage}
\begin{minipage}[t]{0.05\linewidth}
প্রাপক:
\end{minipage}
\begin{minipage}[t]{0.65\linewidth}
{\impn}
\end{minipage}
\begin{minipage}[t]{0.30\linewidth}
\hspace{0em}
\end{minipage}
\begin{minipage}[t]{0.05\linewidth}
\hspace{0em}
\end{minipage}
\begin{minipage}[t]{0.65\linewidth}
{\impadd}
\end{minipage}
\begin{minipage}[t]{0.30\linewidth}
\begin{center}
(মোঃ আল-আমিন)
\\
ডেপুটি কমিশনার অব কাস্টমস
\\
তারিখ: {\rodt}
\end{center}
\end{minipage}
\newline
{\fileno}
\\
অনুলিপি অবগতি ও প্রয়োজনীয় কার্যকরী ব্যবস্থা গ্রহণের জন্য প্রেরণ করা হলো
\\
\begin{minipage}[t]{0.06\linewidth}
০২।
\end{minipage}
\begin{minipage}[t]{0.94\linewidth}
যুগ্ম কমিশনার (জেটি), কাস্টম হাউস, চট্টগ্রাম।
\end{minipage}
\begin{minipage}[t]{0.06\linewidth}
০৩।
\end{minipage}
\begin{minipage}[t]{0.94\linewidth}
সিএন্ডএফ এজেন্ট {\cnfn}।
\end{minipage}
\begin{minipage}[t]{0.06\linewidth}
০৪।
\end{minipage}
\begin{minipage}[t]{0.94\linewidth}
পিএটু কমিশনার, কাস্টম হাউস, চট্টগ্রাম (কমিশনার মহোদয়ের সদয় অবগতির জন্য)।
\end{minipage}
\begin{minipage}[t]{0.06\linewidth}
০৫।
\end{minipage}
\begin{minipage}[t]{0.94\linewidth}
অফিস কপি।
\\
\\
\\
\end{minipage}
\begin{minipage}[t]{0.70\linewidth}
\hspace{0em}
\end{minipage}
\begin{minipage}[t]{0.30\linewidth}
\begin{center}
(মোঃ আল-আমিন)
\\
ডেপুটি কমিশনার অব কাস্টমস
\end{center}
\end{minipage}
\thispagestyle{case}




\end{document}



\end{document}

